
%%%%%%%%%%%%%%%%%%%%%%% file typeinst.tex %%%%%%%%%%%%%%%%%%%%%%%%%
%
% This is the LaTeX source for the instructions to authors using
% the LaTeX document class 'llncs.cls' for contributions to
% the Lecture Notes in Computer Sciences series.
% http://www.springer.com/lncs       Springer Heidelberg 2006/05/04
%
% It may be used as a template for your own input - copy it
% to a new file with a new name and use it as the basis
% for your article.
%
% NB: the document class 'llncs' has its own and detailed documentation, see
% ftp://ftp.springer.de/data/pubftp/pub/tex/latex/llncs/latex2e/llncsdoc.pdf
%
%%%%%%%%%%%%%%%%%%%%%%%%%%%%%%%%%%%%%%%%%%%%%%%%%%%%%%%%%%%%%%%%%%%

 
\documentclass[runningheads,a4paper]{llncs}

\usepackage{amssymb}
\setcounter{tocdepth}{3}
\usepackage{graphicx}

\usepackage{url}
\urldef{\mailsa}\path|{\ldots,|
\urldef{\mailsb}\path|\ldots,|
\urldef{\mailsc}\path|\ldots}@\ldots.com|    
\newcommand{\keywords}[1]{\par\addvspace\baselineskip
\noindent\keywordname\enspace\ignorespaces#1}

\begin{document}

\mainmatter  % start of an individual contribution

% first the title is needed
\title{Web Services Composition in the presence of Non-Functional Properties: A
MDD Approach}

% a short form should be given in case it is too long for the running head
%\titlerunning{Lecture Notes in Computer Science: Authors' Instructions}

% the name(s) of the author(s) follow(s) next
%
% NB: Chinese authors should write their first names(s) in front of
% their surnames. This ensures that the names appear correctly in
% the running heads and the author index.
%
\author{Pl\'acido A. Souza Neto \and Martin A. Musicante \and\\%
%
 Genoveva Vargas-Solar\and Umberto Costa\and Val\'eria de Castro\\}
%
%\authorrunning{Lecture Notes in Computer Science: Authors' Instructions}
% (feature abused for this document to repeat the title also on left hand pages)

% the affiliations are given next; don't give your e-mail address
% unless you accept that it will be published
\institute{\ldots\\
\mailsa\\
\mailsb\\
\mailsc}

%
% NB: a more complex sample for affiliations and the mapping to the
% corresponding authors can be found in the file "llncs.dem"
% (search for the string "\mainmatter" where a contribution starts).
% "llncs.dem" accompanies the document class "llncs.cls".
%

%\toctitle{Lecture Notes in Computer Science}
%\tocauthor{Authors' Instructions}
\maketitle


\begin{abstract}
\ldots.
\keywords{\ldots}
\end{abstract}
 

\section{Introduction}
\ldots
 
\section{A methodology for web services com\-po\-si\-tions}

In this section we present $\pi$SOD-M. $\pi$SOD-M is a MDA (Model Driven
Architecture) based methodology. It provides a environment for building service compositions considering
their non-functional requirements. $\pi$SOD-M extends the SOD-M \cite{valeriaThesis} method by adding
the concept of \textit{Policy} \cite{Espinosa-OviedoVZC09,Espinosa-Oviedo2011a}
for representing NFR associated to service-based applications. $\pi$SOD-M also
proposes the generation of a set of models at different abstraction levels, as
well as transformations between these models.
  
$\pi$SOD-M's models represent both the cross-cutting aspects of the application
being modelled, as well as the constraints associated to services. The systems
cross-cutting concerns affect functional concerns, such as availability;
recovery; and persistence aspects. Constraints are restrictions that must be
respected during the execution of the application, for example the fact that a
service requires an authentication for executing system functions. 

$\pi$-SOD-M supports the construction of service-oriented applications that implement business processes.
Therefore, it proposes a development process based on the definition of models
(instances of the meta-modes) and transformations for semi-automatically
generating different levels of abstraction that describe a service-oriented
application from abstract platform independent views (CIM and PIM level) to
platform dependent views and the PSM and implementation levels. We extended the
Business and Service views of the SOD-M method \cite{CastroMV11}. The Business
view defines  the concepts for modeling a business process, while the Service
view defines the concepts  for designing services compositions. Our methodology
introduces concepts (e.g. NF-requirement, constraint, assertion, contract,
policy) in the Policy view for describing constraints associated to services and
non-functional properties associated to service processes.

\section{Modeling web services com\-po\-si\-tions with
$\pi$SOD-M}



\section{Measurement and evaluation of the $\pi$SOD-M methodology}

Based on \cite{basili:1985}, the measurement and evaluation process requires a
mechanism for determining what data is to be collected; why it is to be
collected; and how the collected data is to be interpreted. The purpose of the measurement and
evaluation flows from the needs of the organization; the need to better
understand resource utilization in order to improve cost estimation; or the need
to evaluate the benefits, advantages and drawbacks of a research project.

Aiming to collect data that help in an initial analysis of our methodology
through three examples described in this chapter, we will use the approach
proposed by Basili \cite{basili:1985} to extract these data and analyze if the goals of
our proposal have been achieved and how it is possible improve the $\pi$SOD-M
methodology. 
% 
% Com o objetivo de coletar dados que auxiliem em uma análise inicial da
% metodologia proposta através dos 3 exemplos descritos neste capítulo, iremos
% utilizar a abordagem proposta por Basili para extrair esses dados e analisar se
% os objetivos da nossa proposta foram alcançados e de que forma é possível
% melhorar a metodologia $\pi$SOD-M.

\paragraph{Generate a set of goals:} First we present the goals we have, when we
intend to measurement and evaluate the $\pi$SOD-M methodology. The goals are:

% Primeiramente apresentamos os objetivos que almejamos ao realizar essa medição e
% avaliação da metodologia. Os objetivos são.
% 
% Objetivos:

\begin{enumerate}
  \item  Verify if the identification of non-functional requirements at initial
  levels of development improves the modeling of reliable web services.
  
%   Verificar se a identificação de requisitos não-funcionais nos níveis iniciais de desenvolvimento melhora a modelagem de serviços web confiáveis.

  \item  Verify how to refine application constraints into specification code,
  and if it improves the software quality.
  
%   Verificar se é possível refinar restrições de qualidade em código de especificação.

  \item  Verify if it is possible to reuse code generated for requirements
non-functional.

% Verificar se é possível a reutilização de código gerado para requisitos
% não-funcionais.

  \item  Verify how the non-functional requirements are specified in the
  methodology.

\end{enumerate}

\paragraph{Set of questions of interest which quantify the goals:} Following the
approach proposed in Basili \cite{basili:1985}, the questions are related to each goal. Table \ref{tab7:questions} present the questions related to the goals of the $\pi$SOD-M evaluation analysis.

\begin{table}
\centering
\caption{$\pi$SOD-M's Goals and Questions}
\label{tab7:questions}

\begin{tabular}{|l|l|}

\hline 
\hline \textbf{Goals} & \textbf{Questions} \\

\hline 1. Verify if the definition of constraints at initial& 1. What are the
levels of modeling  \\ 

 the identification levels of development improves  & presented in the
methodology?  \\

   of non-functional requirements. &    And how the
  constraints are modelled?  \\
 
 &  \\
 
 &  2. What is the goal of each modeling level?\\

&  \\
 
 &  3. Which kind of artifact is generated at\\
 & each level of development?\\
\hline
2. Verify how to refine application constraints & 1. What kind of modelling
approach\\

 into contracts assertions, and if it facilitates &  is used in the
 methodology?
 \\
 
the identification of non-functional requirements. & \\

& 2. How are modeled the contract assertions?\\


\hline

3. Verify if it is possible to classify the contracts  &1. Which level of
development the contracts \\ 
generated from the constraints. &  are
generated? and Does the code\\ 
&is generated?\\
&\\
& 3. The code can be generated automatically \\
& or manual?\\

\hline

4. Verify how the non-functional requirements & 1. How are classified the
contracts? \\ 
could be classified from the contract definition. &  \\

\hline
\hline

\hline


\end{tabular}

\end{table}


\paragraph{Set of data metrics and distributions:} Still following the approach
proposed by Basili \cite{basili:1985}, we identify metrics with the objective of quantify the questions in order to generate data that can be
analyzed and evaluated in order to verify that the goals of the evaluation
were satisfied, or not. The metrics are listed by goal/question (for
example \#1.2. This metric is related to \textit{goal} 1, and \textit{question}
2 previously identified). The statements for measure the data are
shown in table \ref{tab8:metrics}.

\begin{table}
\centering
\caption{$\pi$SOD-M's Metrics/Distributions}
\label{tab8:metrics}

\begin{tabular}{|c|l|}

\hline 
\hline \textbf{Goal/Question} & \textbf{Metrics/Distributions} \\

\hline \#1.1 & 1. How many levels the methodology defines?  \\
& 2. How many models are generated at each level?\\
\hline \#1.2 & 1. How many restrictions were modeled (by level)?  \\
\hline \#1.3 & 1. How many artifacts were generated (by level)?  \\
\hline \#2.1 & 1. How long (in days) was spent for the modeling   \\
&  of each case study? \\
\hline \#2.2 & 1. How many contracts were modeled using the approach?  \\
\hline \#3.1 & 1. How many lines of code were generated for \\
& specify the contracts identified? \\

\hline \#3.2 & 1. What is the total of line of code generated? \\

\hline \#4.1 &  1. Have been any non-functional requirement identified?\\

 &  2. Which non-functional requirements were identified?\\

\hline
\hline

\hline


\end{tabular}

\end{table}


\paragraph{Collecting the data:} After the development of metrics and the types
of data to be collected, the extraction was performed considering the modeling
steps of $\pi$SOD-M for each case study presented in this chapter. Table
\ref{tab9:data} presents the results of the data collection.   

  
\paragraph{Evaluation of the data:} Tables \ref{tab8:metrics} and
\ref{tab9:data} present the questions and data generated from the evaluation of
each example. All examples were modeled in accordance with the proposed
methodology and yielded the following results.



\begin{table} 
\centering
\caption{Results: $\pi$SOD-M's Metrics/Distributions}
\label{tab9:data}

\begin{tabular}{|c|c|c|c|}
\hline 
\cline{2-4}
& \multicolumn{3}{ |c| }{$\pi$SOD-M} \\
\hline  
Metrics/Distributions & Example 1 & Example 2 & Example 3 
\\
\hline 
\hline \#1.1.1 & 4 & 4 & 4 \\
\hline \#1.1.2 & 1-1-1-1 & 2-2-2-2 & 3-1-1-1 \\
\hline \#1.2.1 & 5c-8a-6r-2co & 7c-9a-7r-5co & 4c-6a-6r-3co\\
\hline \#1.3.1 & 1 & 1 & 1 \\
\hline \#2.1.1 & 8 & 10 & 5\\
\hline \#2.2.1 & 2& 3& 3\\
\hline \#3.1.1 &9 & 37 & 20 \\
\hline \#3.2.1 & 29 & 56& 40\\
\hline \#4.1.1 & 2& 3& 3\\
%\hline \#4.1.2 & & & \\

\hline

\end{tabular}
\end{table}


The methodology defines 4 levels of modeling (\textit{\#1.1.1}), however the
case studies generated a different number of models ($\pi$-UseCase,
$\pi$-ServiceProcess, $\pi$-ServiceComposition e $\pi$-PEWS). In \textit{Example
1} was generated 1 model of each type, in \textit{Example 2} were generated two
models of each type, and in \textit{Example 3} were generated 3 $\pi$-UseCase
models and 1 referring to the others (\textit{\#1.1.2}). We found that this
variation has no involvement with the identification of the number of non-functional
requirements.


Regarding the types of restrictions modeled at each level, we find that
the number constraints identified implies an increase in the number of
lines of code. The  \textit{Example 1} set five \textit{constraints} in the
$\pi$-UseCase model, 8 \textit{assertions} in the $\pi$-ServiceProcess model, 6
\textit{rules} in the $\pi$-ServiceComposition model and 2 \textit{contracts} in
the $\pi$-PEWS model (\textit{5c-8a-6r-2Co} \footnote{5c =
5 constraints, assertions 8th = 8; 6r = 6 rules; 2Co contracts = 2}). This
example generated 29 lines of code specification. However, the \textit{Example
2} presents the following data \textit{7c-9a-7r-5co}, generating 56 lines of
specification code. It was observed that the number of restriction increases the
number of lines specification. The \textit{Example 3} presents the following
data, \textit{4c-6a-6r-3co}, generating 40 code lines specification
(\textit{\#1.2.1}, \textit{\#3.1.1} and \textit{\#3.2.1}).


According to the data analysis, we also found that the number of lines of code
for the definition of contracts increase according to the numbers of rules
identified (\textit{\#1.2.1}). 

The number of days spent for modeling also varied
according to the number of constraints. Thus, the time spent with
identification and modeling of the constraints implies a greater number of days
of development. \textit{Example 2} was modeled in 10 days (\textit{\#2.1.1}) and
this example showed a greater number of restrictions and non-functional
requirements. We assume that if these restrictions were not modeled since the initial stages, the time spent on
maintenance could be higher. \textit{Example 3} had the fewest
number of restrictions, and therefore took less time to be modeled. Thus the
number of non-functional requirements of an application implies the time spent
for the project development using $\pi$SOD-M (\textit{\#2.2.1} and
\textit{\#4.1.1}).


\section{Final remarks}

\bibliography{biblio} 
\bibliographystyle{plain}
\end{document}
