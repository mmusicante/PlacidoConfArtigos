\documentclass[preprint,12pt]{elsarticle}
\usepackage{geometry}
%\geometry{letterpaper}                   % ... or a4paper or a5paper or ...
\usepackage{graphicx}
\usepackage{amssymb}
\usepackage{epstopdf}
%% Use the option review to obtain double line spacing
%% \documentclass[preprint,review,12pt]{elsarticle}

%% Use the options 1p,twocolumn; 3p; 3p,twocolumn; 5p; or 5p,twocolumn
%% for a journal layout:
%% \documentclass[final,1p,times]{elsarticle}
%% \documentclass[final,1p,times,twocolumn]{elsarticle}
%% \documentclass[final,3p,times]{elsarticle}
%% \documentclass[final,3p,times,twocolumn]{elsarticle}
%% \documentclass[final,5p,times]{elsarticle}
%% \documentclass[final,5p,times,twocolumn]{elsarticle}

%% if you use PostScript figures in your article
%% use the graphics package for simple commands
%% \usepackage{graphics}
%% or use the graphicx package for more complicated commands
%% \usepackage{graphicx}
%% or use the epsfig package if you prefer to use the old commands
%% \usepackage{epsfig}

%% The amssymb package provides various useful mathematical symbols
\usepackage{amssymb}
%% The amsthm package provides extended theorem environments
%% \usepackage{amsthm}

%% The lineno packages adds line numbers. Start line numbering with
%% \begin{linenumbers}, end it with \end{linenumbers}. Or switch it on
%% for the whole article with \linenumbers after \end{frontmatter}.
%% \usepackage{lineno}

%% natbib.sty is loaded by default. However, natbib options can be
%% provided with \biboptions{...} command. Following options are
%% valid:

%%   round  -  round parentheses are used (default)
%%   square -  square brackets are used   [option]
%%   curly  -  curly braces are used      {option}
%%   angle  -  angle brackets are used    <option>
%%   semicolon  -  multiple citations separated by semi-colon
%%   colon  - same as semicolon, an earlier confusion
%%   comma  -  separated by comma
%%   numbers-  selects numerical citations
%%   super  -  numerical citations as superscripts
%%   sort   -  sorts multiple citations according to order in ref. list
%%   sort&compress   -  like sort, but also compresses numerical citations
%%   compress - compresses without sorting
%%
%% \biboptions{comma,round}

% \biboptions{}


\journal{Journal of Systems and Software}

%%% OUR MACROS %%%
\newcommand{\COMMENT}[1]{ }

\usepackage[usenames,dvipsnames]{xcolor}


\usepackage{amsmath}
\usepackage[thmmarks,amsmath]{ntheorem}

\newcommand{\openbox}{\leavevmode
  \hbox to.77778em{%
  \hfil\vrule
  \vbox to.675em{\hrule width.6em\vfil\hrule}%
  \vrule\hfil}}

\theoremstyle{plain}
\theoremheaderfont{\normalfont\bfseries}
\theorembodyfont{\normalfont}
\theoremseparator{}
\theoremindent0cm
\theoremnumbering{arabic}
\newtheorem{algo}{Algorithm}

\theoremstyle{plain}
%\theoremheaderfont{\normalfont\itshape}
\theoremheaderfont{\normalfont\bfseries}
\theorembodyfont{\normalfont}
\theoremseparator{}
\theoremindent0cm
\theoremnumbering{arabic}
\theoremsymbol{\ensuremath{\openbox}} 
\newtheorem{example}{Example}


\theoremstyle{plain}
\theoremheaderfont{\normalfont\bfseries}
\theorembodyfont{\normalfont}
\theoremseparator{.}
\theoremindent0cm
\theoremnumbering{arabic}
\theoremsymbol{\ensuremath{\Box}} 
\newtheorem{defi}{Definition}

\theoremstyle{plain} 
\theoremsymbol{\ensuremath{\Box}} 
\theoremseparator{.} 
\newtheorem{prop}{Property}

\def\FlyingPig{\textsl{FlyingPig}}

\newcounter{numberInTrivlist}

\newenvironment{numtrivlist}{\begin{list}{\rm \arabic{numberInTrivlist})} 
                                         {\usecounter{numberInTrivlist}
                                          \setlength{\leftmargin}{0pt}
                                          \setlength{\rightmargin}{0pt}
                                          \setlength{\itemindent}{12pt}
                                          \setlength{\listparindent}{0pt}}}
                            {\end{list}}

\newenvironment{itemizedTrivlist}{\begin{list}{\rm ~\hspace{2mm} $\bullet$\ } 
                                         {\setlength{\leftmargin}{0pt}
                                          \setlength{\rightmargin}{0pt}
                                          \setlength{\itemindent}{12pt}
                                          \setlength{\listparindent}{0pt}}}
                            {\end{list}}


\begin{document}

\begin{frontmatter}

%% Title, authors and addresses

%% use the tnoteref command within \title for footnotes;
%% use the tnotetext command for the associated footnote;
%% use the fnref command within \author or \address for footnotes;
%% use the fntext command for the associated footnote;
%% use the corref command within \author for corresponding author footnotes;
%% use the cortext command for the associated footnote;
%% use the ead command for the email address,
%% and the form \ead[url] for the home page:
%%
%% \title{Title\tnoteref{label1}}
%% \tnotetext[label1]{}
%% \author{Name\corref{cor1}\fnref{label2}}
%% \ead{email address}
%% \ead[url]{home page}
%% \fntext[label2]{}
%% \cortext[cor1]{}
%% \address{Address\fnref{label3}}
%% \fntext[label3]{}

\title{$\pi$SOD-M: A Methodology for Building Service-Oriented Applications in
the Presence of Non-Functional Properties}

%% use optional labels to link authors explicitly to addresses:
%% \author[label1,label2]{<author name>}
%% \address[label1]{<address>}
%% \address[label2]{<address>}


\author[inst1]{Khalid Belhajjame}
\author[inst2]{Valeria~de~Castro}
\author[inst3]{Umberto Souza da Costa}
\author[inst4]{Javier~A.~Espinosa-Oviedo}
\author[inst3]{Martin A. Musicante}
\author[inst5]{Pl\'acido A. Souza Neto}
\author[inst6,inst4]{Genoveva Vargas-Solar}
\author[inst4]{Jos\'e-Luis Zechinelli-Martini}


\address[inst1]{Universi\'e de Paris - Dauphine -- Paris, France}
\address[inst2]{Universidad Rey Juan Carlos -- M\'{o}stoles, Spain}
\address[inst3]{Federal University of Rio Grande do Norte -- Natal, Brazil}
\address[inst4]{Universidad de las Am\'ericas-Puebla, LAFMIA -- Cholula, Mexico}
\address[inst5]{Federal Technological Institute of Rio Grande do Norte -- Natal, Brazil}
\address[inst6]{CNRS, LIG-LAFMIA, Saint Martin d'H\`eres, France}


\begin{abstract}
This paper presents\ldots
\end{abstract}

\begin{keyword}
%% keywords here, in the form: keyword \sep keyword

%% MSC codes here, in the form: \MSC code \sep code
%% or \MSC[2008] code \sep code (2000 is the default)

\end{keyword}

\end{frontmatter}

%%
%% Start line numbering here if you want
%%
% \linenumbers

%% main text
%*********************************************************************************************************
\section{Introduction}
\label{sec:intro}

Model Driven Development (MDD) is a top-down approach for the development of software systems. 
The main ideas of MDD were originally proposed by the Object Management Group (OMG)~\cite{mda}, as a set of guidelines for the structuring of specifications.
The technique advocates for the use of \textit{models} to specify a software system at different levels of abstraction (called \textit{viewpoints}):

\begin{trivlist}
\item \textbf{Computation Independent Models (CIM):} This level focusses on the
environment of the system, as well as on its business and requirement specifications. 
This viewpoint represents the software system at its highest level of abstraction. 
At this moment of the development, the structure and system processing details are still unknown or undetermined. 
 
\item \textbf{Platform Independent Models (PIM):} This level focusses on the system functionality, hiding the details of any particular platform. 
The specification defines those parts of the system that do not change from one platform to another. 

\item \textbf{Platform Specific Models (PSM):} This level focusses on the functionality, in the context of a particular implementation platform.
Models at this level combine the platform-independent view with the specific aspects of the platform to implement the system.  
\end{trivlist}

Besides the notion of model at each level of abstraction, MDD requires the use of \textit{model transformations} within and between levels.
Intra-level transformations are used to provide a unified representation of concepts of a given level.
Inter-level transformation implement a refinement process between levels.
Transformations may be automatic or semi-automatic.

MDD techniques has been successfully used for the development of hardware and software systems~\cite{MDDvariosAqui}. 
In particular, we are interested in the application of MDD to the design and implementation of web service applications.

Service oriented computing~\cite{Papazoglou2007} is at the origin of an evolution in the field of software development~\cite{??}. 
An important challenge of service oriented development is  to ensure the alignment between the requirements imposed by the business logic and the IT systems actually developed.
(Moreover, IT systems need to evolve according to the business needs.)
Thus, organizations are seeking for mechanisms to bridge the gap between the systems developed and business needs~\cite{bell}. 
The literature stresses the need for methodologies and techniques for service oriented analysis and design, claiming that they are the cornerstone  in the development of meaningful service based applications~\cite{5}.  
In this context, we argue that the convergence of model-driven software development, service orientation and better techniques for documenting and improving business processes are key to make real the idea of rapid, accurate development of software that serves, rather than dictates the needs of its users~\cite{watson}. 

In Service-Oriented Computing, pre-existing services are
combined to produce applications and provide the business logic. 
The selection of services is usually guided by the \textit{functional} requirements of the application being developed~\cite{1,2,decastro1,PapazoglouH06}. 
(Functional properties of a computer system are characterized by the effect produced by the system when given a defined input.)
Functional properties are not the only crucial aspect in the software development process. 
Other properties need to be addressed to fit in the application with its context.
These other aspects are called Non-Functional Properties.

Non-functional aspects of the services, often expressed as requirements and constraints in general purpose methodologies, are not usually considered from the beginning of the (service) software process.
Most methods consider them only after the application has been implemented, in order to ensure some level of reliability (e.g., data privacy, exception handling, atomicity, data persistence). 
This leads to service based applications that are partly specified and, thereby, partly compliant with the requirements of the application.
Ideally, non-functional requirements should be considered along with all the stages of the software development. 
The adoption of non-functional specifications from the early states of development
can help the developer to produce applications that are capable of dealing with
the application context.

In this work, we are interested in the extension of the \textit{Service Oriented Development Method} (SOD-M)~\cite{decastro1}, to support non-functional aspects, from the early stages of software development.
SOD-M is aligned with the MDD directives and proposes models, practices and techniques for the development of service-based applications.
SOD-M does not provide support for the specification of non-functional requirements, such as
security, reliability, and efficiency. 

The main goals of our work are:
\begin{trivlist}
\item \textit{(i)} To define a NFR model including a set of concepts need for the modeling of NFR in service-oriented applications.
\item \textit{(ii)} To propose a methodology for supporting the construction of service-oriented applications, taking into account both functional and non-functional requirements;
\item \textit{(iii)} To improve the construction process by providing an abstract view of the application and ensure the conformance to its specification;
\item \textit{(iv)} To reduce the programming effort through the semi-automatic generation of  models for the application, to produce concrete implementations from high abstraction models;
\end{trivlist}

The rest of the paper is organized as follows. 
Section\dots


\begin{center}
\textsc{\underline{Version from CAISE submission}}
\end{center}


Service oriented computing is at the origin of an evolution in the field of software development. 
An important challenge of service oriented development is  to ensure the alignment between IT systems and the business logic.
%dealing thereby with the promise that IT systems can  evolve  according to the business needs. 
Thus, organizations are  seeking for mechanisms to deal with the gap between the systems developed and business needs \cite{bell}. The literature stresses the need for methodologies and techniques for service oriented analysis and design, claiming that they are the cornerstone  in the development of meaningful services' based applications \cite{5}.  In this context, some authors argue that the convergence of model-driven software development, service orientation and better techniques for documenting and improving business processes are the key to make real the idea of rapid, accurate development of software that serves, rather than dictates, software users' goals \cite{watson}. 

Service oriented development methodologies providing models, best practices, and reference architectures to build services' based applications mainly address  functional aspects \cite{1,2,decastro1,PapazoglouH06}.  Non-functional aspects concerning services' and application's "semantics", often expressed as requirements and constraints in general purpose methodologies, are not fully considered or they are added once the application has been implemented in order to ensure some level of reliability (e.g., data privacy, exception handling, atomicity, data persistence). This leads to services' based applications that are partially specified and that are thereby partially compliant with application requirements.

The objective of this work   is to model non-functional constraints and associate them to  services' based applications  early during the services' composition modeling phase. Therefore this paper presents $\pi$-SOD-M, a model-driven method  that extends the SOD-M  \cite{decastro1} for building reliable  services' based information systems (SIS). 
%SOD-M defines a process  starting with the  identification of business services through business modeling, and, by means of models' transformations it allows to obtain a services' composition model \cite{decastro1} and the executable code that implements it. 

Our work proposes to extend the SOD-M \cite{decastro1} method with  (i)  the notion of {\em A-Policy} \cite{Espinosa-Oviedo2011a} for representing non-functional constraints associated to services' based applications.  
%{\em A-policies} are used to express constraints which can be applied to all the services' composition  or to a particular service used for implementing it. They represent both systems' cross-cutting aspects (e.g., exception handling expressing what to do when a service is not available) and use constraints imposed by the services  (e.g., the fact that a service requires imposes an authentication protocol for executing a method). 
Our work  also (ii) defines the $\pi$-{\sc Pews}  meta-model \cite{Placido2010LTPD} providing guidelines for expressing the composition and the {\em A-policies}. Finally, our work (iii) defines model to model transformation rules for generating the $\pi$-{\sc Pews} model of a reliable services' composition starting from the extended services' composition model; and, model to text transformations for generating the corresponding implementation. As will be shown within our environment implementing these meta models and rules, one may represent both systems' cross-cutting aspects (e.g., exception handling for describing what to do when a service is not available, recovery, persistence aspects) and constraints associated to services, that must be respected for using them (e.g., the fact that a service requires an authentication protocol for executing a method). 

The remainder of the paper is organized as follows. Section \ref{sec:motivation} gives an overview of our approach. It describes a motivation example that integrates and synchronizes well-known social networks services namely Facebook, Twitter and, Spotify. Sections \ref{sec:piscm}, \ref{sec:pewsmetamodel}, and \ref{sec:mmrules} describe respectively the three key elements of our proposal, namelly the $\pi$-SCM and $\pi$-{\sc Pews} meta-models and the transformation rules that support the semi-automatic generation of reliable services' compositions.
%describes $\pi$-SOD-M method that enables the representation and association of {\em A-policies} to services' composition  thereby making them reliable. 
%
Section \ref{sec:implementation} describes implementation and validation issues.
Section \ref{sec:related} analyses related work concerning policy/contract based programming and, services' composition platforms. Section \ref{sec:conclusions} concludes the paper and discusses future work.









\section{Related Works}
\label{sec:relworks}
%Conceptos existentes en NFR y metodologías similares
While Functional Requirements establish \textit{what} is computed by an application, Non-Functional Requirements (NFRs) are concerned with \textit{how} the task is preformed.  
NFRs include aspects such as performance, authentication and quality constraints.
Non-functional properties are also referred to as constraints, quality attributes��, quality goals��, quality of service requirements and ��non-behavioural requirements�� \cite{StGr05}. 
In the case of service-based applications, non-functional requirements concern the application itself as well as its component services. 
%The majority of the works on NFPs focus on measuring in which extent a software system  fulfills the NFPs that it should satisfy \cite{BBKL78,FePf96,KiDa96,Lyu96,MuIO90}. 
Most research on NFRs focus on the evaluation of compliance by the software system. 

In~\cite{Babamir2010,Yeom2006} non-functional properties of web services are classified according to three points of view, namely,  
\textit{service level}, \textit{system level} and \textit{business level}.
In~\cite{Babamir2010} NFRs are denoted as \textit{quality constraints}, which are expressed as logic formulae.
In~\cite{Yeom2006} authors classify NFRs into \textit{category}, \textit{sub-category} and \textit{property}. Categories include \textit{business}, \textit{service} and \textit{system}.
Possible \textit{sub-categories} are \textit{security}, \textit{value} or  \textit{interoperability}. The work also defines a \textit{web service quality model}, which considers non-functional properties. 

In~\cite{XiaoCZBOLH08} the authors use the terms  
\textit{non-functional attributes}, \textit{composition mo\-del entity} and \textit{mo\-del entity}  to classify different concepts related to NFRs.
The notion of non-functional attribute is used to describe NFRs of the abstract process model. In the lower level, the composition is annotated with non-functional attributes.

D'Ambrogio~\cite{DAmbrogio06} uses the term \textit{quality category} to group similar \textit{quality characteristics}. 
\textit{Quality dimensions} are used to quantify an individual characteristic.
For instance, the quality category \textit{performance} groups characteristics such as
\textit{latency} and \textit{throughput}. 
The development process is based on MDA and the authors also present a WSDL extension for describing the QoS of web services. A catalog of \textit{QoS characteristics} is provided for the web service domain, including properties as \textit{availability}, \textit{reliability} and \textit{access control}. 

 
Schmeling et al.~\cite{SchmelingCM11} present an approach and a toolset for specifying and implementing web service compositions with support to several NFRs. The term \textit{non-functional concern} (NFC) is used to denote  NFRs. 
%Two aspects are considered: specification and realization.
\textit{Non-functional concern} is a general term used to describe non-functional requirements. 
For instance, \textit{security}, \textit{reliability}, \textit{transactional behavior} are non-functional requirements. 
A \textit{non-functional action} represents some behavior that implements \textit{non-functional attributes}. 
An example of \textit{non-functional action} is \textit{encryption}, which provides the implementation of the \textit{non-functional attribute} \textit{confidentiality}. 
Non-functional actions related to a common concern are grouped into \textit{non-functional activities}. 

Pastrana et al.~\cite{PastranaPK11} use the term \textit{contract} to describe non-functional requirements. 
\textit{Contracts} may have pre-conditions, post-conditions and invariants. 
Each contract defines \textit{assertions} associated with \textit{quality properties}. 
Each service may have as many associated \textit{contracts} as needed.

Chollet et al.~\cite{CholletL09} associate (non-functional) \textit{quality properties} to 
(functional) activities. The authors present a security meta-model for web service
composition. The NFRs considered are \textit{authentication}, \textit{integrity} and \textit{confidentiality}. 
Each NFR is associated with a service activity.


Ceri et al.\cite{CeriDMF07} uses the notions of \textit{policy}, \textit{rule}, \textit{condition} and \textit{action model} to specify NFRs.
Agarwal et al.~\cite{AgarwalLS09} associate \textit{service policies} to services. 
Each service may also have \textit{properties}, such as \textit{security} and \textit{reliability}. 
Ovaska et al.~\cite{OvaskaEHPA10} use the terms \textit{quality attribute}, \textit{category}, \textit{conceptual layer} and \textit{importance} to organize and classify NFRs.
Other authors do not define specific terms to refer to NFRs. 
They use terms such as \textit{attribute}~\cite{ZhangPSP05,BasinDL06,JeongCL09}, 
\textit{property}~\cite{Fabra2011}, 
\textit{factor}~\cite{MohantyRP10,GutierrezRF10}, 
\textit{characteristic}~\cite{DiamadopoulouMPS08}, 
\textit{quality level}~\cite{ModicaTV09}, and
\textit{value}~\cite{ThissenW06,BasinDL06}.


Despite of the different notations found in the literature for classifying NFRs, some non-functional requirements are frequently considered, such as \textit{security}, \textit{performance}, \textit{reliability}, \textit{usability}, and \textit{availability}.
However, distinct hierarchies and models are proposed for NFRs,  according to different points of view.
We have identified a number of approaches~\cite{DAmbrogio06,CholletL09,SchmelingCM11,BasinDL06,Fabra2011,OvaskaEHPA10} that use MDD (Model Driven Development) for designing and developing applications. 

Fabra \textit{et al.}~\cite{Fabra2011} also describes the importance of  MDD for service-oriented applications. This work  presents a complete development methodology, although this methodology is not centered on NFRs.
The authors in~\cite{ThissenW06,ZhangPSP05} use formal methods to define a service-based development process that takes NFRs into account. 
In~\cite{AgarwalLS09,PastranaPK11} ontologies are used to define and model NFRs, 
whereas in~\cite{XiaoCZBOLH08,GutierrezRF10} Business Process Modeling (BPM) is used for
system specification, including NFRs. 
The majority of the authors concentrate on the  modeling of service compositions, although a significant number of approaches is focused on the definition of NFR models.


In the method defined in~\cite{XiaoCZBOLH08}, tasks in the process model can be 
annotated with \textit{non-functional attributes} (NFAs). 
NFAs are defined apart  and are concerned with data items or tasks. 
NFAs for data considers \textit{value} and \textit{range}, whereas NFAs for tasks include \textit{cost}, \textit{time}, \textit{resources} and \textit{expressions}.

The proposal in~\cite{ThissenW06} presents steps for  selecting services 
by taking QoS information into account. The proposed steps are: 
\textit{(i)} identification of relevant QoS information; 
\textit{(ii)} identification of basic composition patterns and 
QoS aggregation rules for these patterns; and 
\textit{(iii)} definition of a selection mechanism of services. 
The QoS properties considered are \textit{performance}, \textit{cost}, \textit{reliability} and
\textit{availability}. 
  
Karunamurthy et al.~\cite{Karunamurthy2012787} use the term \textit{non-function parameters} to define NFRs, such as \textit{cost}, \textit{response time}, \textit{availability}, \textit{security}, \textit{reliability} and \textit{reputation}.  
The \textit{Non-Func\-tion\-al Specification Language} (NFSL) is proposed as a domain specific language (DSL) to express \textit{non-function parameters}.

Liu et al.~\cite{Liu20121080} use the term \textit{QoS parameter} to describe non-functional requirements such as \textit{cost}, \textit{execution duration}, \textit{accuracy}, \textit{security}, \textit{integrity}, \textit{availability} and \textit{reliability}.  
In the same way, Tran et al.~\cite{Tran2012531} use the term \textit{QoS policies} to classify similar non-functional requirements.

Li et al.~\cite{Li2013} associate \textit{dimensions} to  \textit{QoS parameters} to classify NFRs.  
For instance, the \textit{time} dimension is associated to the \textit{execution time} and \textit{communication time} parameters; the \textit{spatial} dimension is associated to the \textit{storage capacity} and \textit{message length} parameters; the \textit{reliability} dimension is associated to the \textit{availability} and \textit{reliability} parameters and the \textit{cost} dimension is associated to the \textit{service cost} parameter.
Rumpel et al.~\cite{Rumpel2012}  associate \textit{quality requirements} to  \textit{quality properties}. Quality requirements are intended to be specified as constraints. 

\bigskip
Most works agree on distinguishing three points of view, namely the point of view of the organization (or Business view), of the individual service providers (or Service view) and of the composition designer (or System view).
The Business view is concerned with the business logic (\textit{i.e.}, an abstract level of tasks, defined by the guidelines and constraints imposed by the organization).
Service and System views are concerned with the implementation of the software solution: The Service level is concerned with the building blocks of the application.
It may use web services provided by third party sources.
The System level is concerned with the coordination of services, to implement the business logic.


%\subsection{Modeling non-functional requirements in service-oriented applications}
%\label{sec:modeling}
%(comentar la revisión y presentar la clasificacion de conceptos y NFR model con los conceptos que proponemos)
%%(comentar la revisión y presentar la clasificacion de conceptos y NFR model con los conceptos que proponemos)


%\subsection{Methodologies for Service-oriented applications development including non-functional requirements}
%\label{sec:methodologies}
%(aqui propuestas relacionadas, analizar sus caracteristicas, si son MDD, si parte de alto nivel, etc.)
%%(aqui propuestas relacionadas, analizar sus caracteristicas, si son MDD, si parte de alto nivel, etc.)
% De Valeria: la explicación de las caracterísiticas de las propuestas relacionadas está ya en Related Works.




%\section{Background}
%\label{sec:SOD-M}
%%Presentacion de SOD:M
%
The methodology for supporting the construction of service-oriented applications, taking into account both functional and non-functional requirements presented in this work extends the \textit{Service Oriented Development Method} (SOD-M) proposed by De Castro et al.\cite{decastro1}.

SOD-M define a model-driven approach for development service-based applications. The model-driven method provided by SOD-M is based on the Model Driven Architecture (MDA) proposed by the OMG \cite{miller}. MDA defines a top-down approach for the development of software systems that uses the notion of \textit{model} to specify a software system at different levels of abstraction (called \textit{viewpoints}):

\begin{trivlist}
\item \textbf{Computation Independent Models (CIM):} This level focusses on the
environment of the system, as well as on its business and requirement specifications.
This viewpoint represents the software system at its highest level of abstraction.
At this moment of the development, the structure and system processing details are still unknown or undetermined.

\item \textbf{Platform Independent Models (PIM):} This level focusses on the system functionality, hiding the details of any particular platform.
The specification defines those parts of the system that do not change from one platform to another.

\item \textbf{Platform Specific Models (PSM):} This level focusses on the functionality, in the context of a particular implementation platform.
Models at this level combine the platform-independent view with the specific aspects of the platform to implement the system.
\end{trivlist}

Besides the notion of model at each level of abstraction, MDA propose the use of \textit{model transformations} within and between levels. Intra-level transformations are used to provide a unified representation of concepts of a given level. Inter-level transformation implement a refinement process between levels. Transformations may be automatic or semi-automatic.

According to MDA guidelines, SOD-M meta-models are organized into three levels: CIM (\textit{Computational Independent Models}), PIM (\textit{Platform Independent Models}) and PSM (\textit{Platform Specific Models}).

Two models are defined at the CIM level: \textit{value model}
and \textit{BPMN model}.
\textsc{Explain these two here --Martin.}

Models at the PIM level describe the structure of the application flow,
while, the PSM level provides transformations towards more specific platforms.
The PIM-level models proposed by SOD-M are:
\begin{trivlist}
\item \textit{use case}:
\item \textit{extended use case}:
\item \textit{service process}:
\item \textit{service composition}.
\end{trivlist}

SOD-M defines three meta-models at the PSM level: \textit{web service interface}, \textit{extended composition service} and \textit{business logic}.
These three levels have no support for describing non-functional requirements.

The SOD-M approach includes transformations between models:
\textit{CIM-to-PIM, PIM-to-PIM} and \textit{PIM-to-PSM} transformations. Given
an abstract model at the CIM level, it is possible to apply transformations for
generating a model of the PSM level. In this context, it is necessary to
follow the process activities described by the methodology.

SOD-M considers two points of view:
\textit{(i)} \textit{business}, focusing on the characteristics and requirements
of the organization, and \textit{(ii)} \textit{system requirements}, focusing on
features and processes to be implemented in order application requirements. In
this way, SOD-M aims to simplify the design of service-oriented applications, as
well as its implementation using current technologies. 

{\LARGE \color{red} Relocate the next subsections or delete them --Genoveva and Martin}



%..--..--..--..--..--..--..--..--..--..--..--..--..--..--..--..--..--..--..--..--..--..--..--..--..--..--..--..--..--..--..--..--..--..--..--..--..--..--
\subsection{Modeling a services' based application}
%..--..--..--..--..--..--..--..--..--..--..--..--..--..--..--..--..--..--..--..--..--..--..--..--..--..--..--..--..--..--..--..--..--..--..--..--..--..--
 Figure \ref{fig:sodm} shows  SOD-M that defines a service oriented approach   providing  a set of guidelines to build services' based information systems (SIS) \cite{decastro1,decastro2}.  Therefore, SOD-M proposes to use services as first-class objects for the whole process of the SIS  development and it  follows a Model Driven Architecture (MDA) \cite{miller}  approach. Extending from the highest level of abstraction of the MDA, SOD-M provides  a conceptual structure to: first, capture the system requirements and specification in high-level abstraction models (computation independent models, CIMÕs); next,  starting from such models build platform independent models (PIMÕs) specifying the system details; next transform such models into platform specific models (PSMÕs) that bundles the specification of the system with the details of the targeted platform; and finally, serialize such model into the working-code that implements the system. 
\begin{figure} [htpb]
\centering
\includegraphics[width=0.65\textwidth]{figs/SODM}
\caption{SOD-M development process}
\label{fig:sodm}
\end{figure} 
As shown in Figure \ref{fig:sodm}, the SOD-M model-driven process begins by building the high-level computational independent models and enables specific models for a service platform to be obtained as a result  \cite{decastro1}. Referring to the "To Publish Music" application, using SOD-M the designer starts defining an E3value model \footnote{The E3 value model is a business model that represents a business case %graphically as a set of value exchanges ($\nabla$$\triangle$) and value activities (rounded boxes) performed by business actors (squared boxes) 
and allows  to understand the environment in which the services' composition will be placed \cite{e3value}.}  at the CIM level and then the corresponding models of the PIM are generated leading to a services' composition model (SCM).
%SOD-M proposes a set of models, :  i) the three different MDA abstraction levels: CIM, PIM and PSM; and ii) SOD-M views: business and information system views. 
%Model-Driven Engineering (MDE) \cite{schmidt} and particularly MDA \footnote{Model Driven Architecture (MDA)  is the particular model-driven proposal defined by the Object Management Group (OMG).} 
%provide
%is an evolving approach to software development that deals with the provision of 
%models, transformations between them and code generators to address software development. 
%One of the main advantages of model-driven approaches is the provision of 
%It also provides a conceptual structure where the models used by business managers and analysts can be traced towards more detailed models used by software developers.  

%Now, consider that besides the services' composition that represents the order in which the services are called for implementing the application "To Publish Music" it is necessary to model  other requirements that represent the (i) conditions imposed by services for being contacted, for example the fact the Facebook and Twitter require authentication protocol in order to call their methods for updating the wall; (ii) the conditions stemming from the business rules of the application logic, for example the fact that the walls in Facebook and Twitter must show the same song title and if this is not possible then none of them is updated. 

%..--..--..--..--..--..--..--..--..--..--..--..--..--..--..--..--..--..--..--..--..--..--..--..--..--..--..--..--..--..--..--..--..--..--..--..--..--..--
\subsection{Modeling non-functional constraints of services' based applications}
%..--..--..--..--..--..--..--..--..--..--..--..--..--..--..--..--..--..--..--..--..--..--..--..--..--..--..--..--..--..--..--..--..--..--..--..--..--..--
Adding non-functional requirements and services constraints in the services' composition is a complex task that implies programming  protocols for instance authentication protocols to call a service in our example, and atomicity (exception handling and recovery) for ensuring a true synchronization of the results produced by the service methods calls.
% song title disseminated in the walls of the user's Facebook and Twitter accounts. 

Service oriented computing promotes ease of information systems' construction thanks, for instance, to services' reuse. Yet, this is not applied to non-functional constraints as the ones described previously, because they do not follow in general the same service oriented principle and because they are often not fully considered in the specification process of existing services' oriented development methods. Rather, they   are either supposed to be ensured by the underlying execution platform, or they are programmed through ad-hoc protocols. Besides,  they are partially or rarely methodologically derived from the application specification, and they are added once the code has been implemented. In consequence, the resulting application does not fully preserve the compliance and reuse expectations provided by the service oriented computing methods.

%..--..--..--..--..--..--..--..--..--..--..--..--..--..--..--..--..--..--..--..--..--..--..--..--..--..--..--..--..--..--..--..--..--..--..--..--..--..--
%\subsection{$\pi$-SOD-M}
%..--..--..--..--..--..--..--..--..--..--..--..--..--..--..--..--..--..--..--..--..--..--..--..--..--..--..--..--..--..--..--..--..--..--..--..--..--..--

%
%Our proposal, called $\pi$-SOD-M, extends the services' composition model of the SOD-M method  with the notion of {\em A-Policy} \cite{Espinosa-Oviedo2011a} for representing services' composition constraints. 
%
Our work extends SOD-M for building applications by modeling the application logic and its associated non-functional constraints and thereby ensuring the generation of reliable services' composition. 
%This method is described in the following sections.
In order to do, our work organizes non-functional constraints into 
three layers representing: application modeling, services composition and services. The service composition layer serves as an integration layer between the services layer that exports methods and has associated constraints and characteristics; and the application layer that expresses requirements.
At the application layer NFP can refer to business rules and values. A value NFR expresses constraints about the way data and functions can be accessed and executed. For example accessing methods under security protocols.

Business NFP at the service layer concerns properties that are associated to services and defines how to call their exported operations (business properties). For example, response time, storage capacity (e.g., Dropbox service provides 5Giga free storage). Value constraints concern more on the conditions in which services can be used. For example, accessing to a function within an authentica-tion protocol.

Finally, at the service composition layer gives an abstract view of the kind of properties exported by services that can be combined for providing NFP for a composition. For example, confidentiality, authentication, privacy and access control can provide security at the service composition layer.

As a first step in our approach,  we started modeling non-functional constraints at the PSM level. Thus, in this paper we  propose the $\pi$-SCM, the services' composition meta-model extended with {\em A-policies} for modeling non-functional constraints (highlighted in Figure  \ref{fig:sodm} and described in Section \ref{sec:piscm}).  $\pi$-SOD-M defines the $\pi$-{\sc Pews}  meta-model providing guidelines for expressing the services' composition and the {\em A-policies} (see Section \ref{sec:pewsmetamodel}), and also defines model to model transformation rules for generating  $\pi$-{\sc Pews} models starting from $\pi$-SCM models that will support executable code generation (see Section \ref{sec:mmrules}). Finally, our work defines model to text transformation rules for generating the program that implements both the services' composition and the associated {\em A-policies} and that is executed by an adapted engine (see Section \ref{sec:implementation}).



\section{Modeling reliable service compositions with $\pi$-SOD-M}\label{sec:motivation}
%(Aquí lo dejaría con la estructura que teníamos para Caise, presentamos el framework y posteriormente comentamos modelos y transformaciones)

%(Aquí lo dejaría con la estructura que teníamos para Caise, presentamos el framework y posteriormente comentamos modelos y transformaciones)

\subsection{$\pi$-SOD-M meta-models}\label{sec:pisodmmetamodels}
%3.1. pi-SoD-M metamodels (presentar los metamodels)
%3.1. pi-SoD-M metamodels (presentar los metamodels)


The {\em A-policy} based service composition meta-model (Figure \ref{fig:e-scomposition-metamodel})
provides classes to represent workflows\footnote{Workflows will be transformed into implemented service compositions.}.
The meta-model identifies {\sc Business Collaborators}\footnote{We use {\sc capitals} for referring to meta-model classes.} and the {\sc Actions} they perform. 
Instances of this meta-model are UML activity diagrams. 
Figure~\ref{fig:e-scomposition-metamodel} show the modeling elements.
Those classes pictured as coloured boxes deal with non-functional properties.
The classes pictured with a white background are those of SOD-M.


In the meta-model of Figure~\ref{fig:e-scomposition-metamodel}:
\begin{itemizedTrivlist}
\item A {\sc Business Collaborator} element represents those entities that collaborate in the business processes by performing some of the required actions. 
They are graphically presented as a partition in the activity diagram. 
A collaborator can be either internal or external to the system. 
When the collaborator of the business is external to the system, the attribute {\sf IsExternal}\footnote{We use the {\sf sans serif} font for referring to classes defined using a meta-model.} of the collaborator is set to \textbf{true}.

\item {\sc Action}s, a kind of {\sc ExecutableNode}, are represented in the model as an activity. 
Each action represents some type of transformation or processing. 
There are two types of actions: i) a WebService (attribute Type is {\sf WS}); and ii) a simple operation that is not supported by a Web Service, called an {\sc ActivityOperation} (attribute Type is {\sc AOP}).
\begin{figure}[t]
\centering
\includegraphics[width=1.0\textwidth]{figs/E-service-composition-metamodel}
\caption{{\em A-policy} based service composition meta-model ($\pi$-SCM)}
\label{fig:e-scomposition-metamodel}
\end{figure}

\item The {\sc ServiceActivity} element is a composite activity that must be carried out as part of a business service and is composed by one or more executable nodes.

\item In order to represent constraints associated to services compositions, we extended the SOD-M service composition model with two concepts: {\sc Rule} and {\sc A-policy} (see blue elements in the $\pi$-SCM meta-model in Figure \ref{fig:e-scomposition-metamodel}).
We model non-functional constraints by using the notion of {\em A-policy}~\cite{Espinosa-Oviedo2011a,CIC:eovszmc09c}.
An {\em A-policy} is formed by attributes and rules. 
Intuitively, the conditions of each rule will be checked.
In case of no compliance, the actions defined by the rule will be performed.
The {\sc Rule} element represents an event - condition - action rule where the {\sc Event} part represents the moment in which a constraint  will be evaluated.
An {\em A-policy} defines variables and operations that can be shared by the rules and that can be used for expressing their Event and Condition parts. 
\end{itemizedTrivlist}

\begin{figure}[t]%[htpb]
\centering
\includegraphics[width=0.95\textwidth]{figs/e-composition-model}

{\color{red}\LARGE PLACIDO: Please change the names of the boxes in accordance to the explanation --Martin}

\caption{Service composition model for the ``To publish music'' business service.}
\label{fig:servicecompositionmodel}
\end{figure}

\begin{example}[To Publish Music]
To illustrate the use of the $\pi$-SCM meta-model, we define a model for the ``To Publish Music'' scenario (Figure \ref{fig:servicecompositionmodel}). 
In this model, there are three external business collaborators ({\em Spotify, Twitter} and {\em Facebook} \footnote{We use {\em italics} to refer to concrete values of the classes of a model that are derived from the classes of a meta-model.}). 
The model also shows the business process of the application that consists of three service activities: {\em Listen Music}, {\em Public Music} and {\em Confirmation}. 
Note that  the activity {\em Publish Music} calls the actions of two service collaborators namely {\em Facebook} and {\em Twitter}.
Both {\em Facebook} and {\em Twitter} services require authentication protocols in order to execute methods that will read and update the user space. 
%A call to such services must be part of the authentication protocol required by these services.
In the example we  associate two authentication policies, one for the open authentication protocol, represented by the class {\sf\small OAuthPolicy} at {\em Twitter}, that will be associated to the activity  {\sf\small UpdateTwitter} (see Figure \ref{fig:servicecompositionmodel}). 
In the same way, the {\em Facebook} class {\sf\small HTTPAuthPolicy}, for the http authentication protocol will be associated to the activity {\sf\small UpdateFacebook}.

{\sf\small OAuthPolicy} will implement the open authentication protocol.
The {\em A-policy} {\sf\small OAuthPolicy} has a variable {\sf\small Token} that will be used to store the authentication token provided by the service.
This variable is imported through the library {\sf\small OAuthPolicy.Token}. 
The A-policy {\sf\small OAuthPolicy} defines two rules, both can be triggered by events of type {\sf\small ActivityPrepared}: (R$_1$): If no token has been associated to the variable {\sf\small token}, then a token is obtained ; and (R$_2$): if the token has expired, then it is renewed. 
Notice that the code in the actions profits from the imported {\sf\small OAuthPolicy.Token} for transparently obtaining or renewing a token from a third party.

{\sf\small HTTPAuthPolicy} implements the HTTP-Auth protocol. 
The A-policy imports an http protocol library and it has two variables {\sf\small username} and {\sf\small password}.  
The event of type {\sf\small ActivityPrepared} is the triggering event of the rule {\sf\small R$_1$}. 
On the notification of an event of that type, a credential is obtained using the username and password. 
\hfill\openbox
\end{example}

{\color{magenta} Aqui\dots  }

Thanks to rules and policies  it is possible to model and associate non-functional properties to services' compositions  and then generate the code. For example, the atomic integration of information retrieved from different social network services, automatic generation of an integrated view of the operations executed in different social networks or for providing security in the communication channel when the payment service is called.

Back to the  definition process of a SIS, once the {\em A-policy} based services' composition model has been defined, then it can be transformed into a model (i.e., $\pi$-PEWS model) that can support then executable code generation. The following Section describes the $\pi$-PEWS  meta-model that supports this representation. 


%..--..--..--..--..--..--..--..--..--..--..--..--..--..--..--..--..--..--..--..--..--..--..--..--..--..--..--..--..--..--..--..--..--..--..--..--..--..--
\subsubsection{$\pi$-{\sc Pews}  meta-model}\label{sec:pewsmetamodel}
%..--..--..--..--..--..--..--..--..--..--..--..--..--..--..--..--..--..--..--..--..--..--..--..--..--..--..--..--..--..--..--..--..--..--..--..--..--..--
The idea of the $\pi$-{\sc Pews} meta-model is based on the services' composition approach provided by the language PEWS\cite{BaAM06,Placido2010LTPD} (\textit{Path Expressions for Web Services}), a programming language that lets the service designer  combine the methods or subprograms that
implement each operation of a service, in order to achieve the desired application logic. Figure \ref{fig:metamodel} presents the $\pi$-{\sc Pews} meta-model
consisting of  classes representing:
\begin{itemize}
\item A services' composition: {\sc Namespace} representing the interface exported by a service, {\sc Operation} that represents a call to a service method, {\sc CompositeOperation}, and  {\sc Operator} for representing a services' composition and {\sc Path} representing a services' composition.
A {\sc Path} can be an {\sc Operation} or a {\sc Compound Operation}
denoted by an identifier. A {\sc Compound Operation} is defined using an  {\sc Operator}  that can be represent  sequential ($\ . \ $) and parallel ($\ \| \ $) composition of services,
 choice ($\ + \ $) among services,
the sequential ($*$) and parallel ($\{\dots\}$) repetition of an operation or the conditional execution of an operation ($[C]S$).

\item {\em A-Policies} that can be associated to a services' composition:  {\sc A-Policy}, {\sc Rule}, {\sc Event}, {\sc Condition}, {\sc Action}, {\sc State}, and {\sc Scope}.
\end{itemize}
%
\begin{figure}
\centering
\includegraphics[width=0.80\textwidth]{figs/PEWSMetamodel}
\caption{$\pi$-{\sc Pews} Metamodel}
\label{fig:metamodel}
\end{figure}

As shown in the diagram an {\sc A-Policy} is applied to a {\sc Scope} that can be either an {\sc Operation} (e.g., an authentication protocol associated to a method exported by a service),  an {\sc Operator} (e.g., a temporal constraint associated to a sequence of operators, the authorized delay between reading a song title in Spotify and updating the walls must be less then 30 seconds), and a {\sc Path} (e.g., executing the walls' update under a strict atomicity protocol -- all or noting).  It groups a set of ECA rules, each rule having a classic semantics, i.e, {\em when an event of type E occurs if  condition C is verified then execute the action A}.  Thus, an {\em A-policy} represents a set of reactions to be possibly executed if one or several triggering events of its rules are notified.
\begin{itemize}
\item The class {\sc Scope} represents any element of a services' composition (i.e., operation, operator, path).
\item The class {\sc A-Policy} represents a recovery strategy implemented by ECA rules of the form {\sc Event} - {\sc Condition} - {\sc Action}. A {\em A-policy} has variables that represent the view of the execution state of its associated scope, that is required for executing the rules. The value of a variable is represented using the type {\sc Variable}. The class {\sc A-Policy} is specialized for defining specific constraints, for instance authentication {\em A-policies}.
\end{itemize}

%An authentication {\em A-policy} represents the situation where an invocation in
%an activity occurs until its sender and/or its recipient have been
%identified. Typically, authentication A-Policies ensure that the invocation of the activity will be done within an authentication protocol.
%

Given a $\pi$-SCM model of a specific services' based application (expressed according to the $\pi$-SCM meta-model), it is possible to generate its corresponding $\pi$-{\sc Pews} model thanks to transformation rules. The following Section describes the transformation rules between the $\pi$-SCM and $\pi$-{\sc Pews} meta-models of our method.





\subsection{$\pi$-SOD-M transformations}\label{sec:pisodmtransformations}
%3.2. pi-sod-M transformations (con la tablas del caise)

Figure \ref{fig:transformations} shows the transformation principle between the elements of the $\pi$-SCM meta-model used for representing the services' composition into the elements of the $\pi$-{\sc Pews} meta-model. There are two groups of rules: those that transform services' composition elements of the $\pi$-SCM to $\pi$-{\sc Pews} meta-models elements; and those that transform rules grouped by policies into {\em A-policy} types.

% _ . _ . _ . _ . _ . _ . _ . _ . _ . _ . _ . _ . _ . _ . _ . _ . _ . _ .
%\noindent

%{\bf\em 
\subsection{Transformation of the services' composition elements of the $\pi$-SCM to the $\pi$-{\sc Pews} elements}
% _ . _ . _ . _ . _ . _ . _ . _ . _ . _ . _ . _ . _ . _ . _ . _ . _ . _ .
A named action of the $\pi$-SCM represented by  {\sc\em Action} and {\sc\em Action:name} is transformed to a  named class {\sc Operation} with a corresponding attribute name {\sc Operation:name}. A  named service activity represented by the elements {\sc\em ServiceActivity}  and  {\sc\em ServiceActivity:name} of the $\pi$-SCM, are  transformed into a named operation of the $\pi$-{\sc Pews} represented by the elements  {\sc CompositeOperation} and {\sc CompositeOperation:name}. When more than one action is called, according to the following  composition patterns expressed using the operators {\sc\em merge, decision, fork and join} in the $\pi$-SCM the corresponding transformations, according to the PEWS operators presented above, are (see details in Figure \ref{fig:transformations}):
\begin{itemize}
\item   $op_1 . op_2$ if no {\sc\em ControlNode} is specified
\item ($op_1 \parallel op_2) . op_3$ if control nodes of type {\sc\em fork, join} are combined
 \item ($op_1 + op_2) . op_3$ if control nodes of type {\sc\em decision, merge} are combined
\end{itemize}

In the scenario "To Publish Music" the service activity {\sf PublishMusic} of the $\pi$-SC model specifies  calls to two {\sf Activitie}s of type {\em UpdateMusic}, respectively concerning the {\sf Business Service}s {\em Facebook} and {\em Twitter}. Given that no {\sf ConstrolNode} is specified by the $\pi$-SC model, the corresponding transformation is the expression that defines a {\sf Composite Operation} named {\em PublishSong} of the $\pi$-{\sc Pews} model of the form ({\sf PublishFacebook} $\parallel$ {\sf PublishTwitter}).
\begin{figure}
\centering{
%\includegraphics[width=0.80\textwidth]{figs/PI-SC-PI-P}}
\includegraphics[width=0.96\textwidth]{figs/Mapping-1}}
\caption{ $\pi$-SCM to $\pi$-{\sc Pews} transformation}
\label{fig:transformations}
\end{figure}

% _ . _ . _ . _ . _ . _ . _ . _ . _ . _ . _ . _ . _ . _ . _ . _ . _ . _ .
%\noindent

%{\bf\em 
\subsection{Transformation of rules grouped by A-policies   in the $\pi$-SCM to A-Policies of  $\pi$-{\sc Pews}}
% _ . _ . _ . _ . _ . _ . _ . _ . _ . _ . _ . _ . _ . _ . _ . _ . _ . _ .
The {\em A-policies} defined for the elements of the $\pi$-SCM are transformed into {\sc A-Policy} classes, named according to the names expressed in the source model. The transformation of the rules expressed in the $\pi$-SCM is guided by the event types associated to these rules.   The variables associated to an {\em A-policy} expressed in the $\pi$-SCM as {\sc\em $<$Variable:name, Variable:type$>$} are transformed into elements of type {\sc Variable} with attributes {\sc name} and {\sc type} directly specified from the elements {\sc\em  Variable:name} and {\sc\em Variable:type} of the $\pi$-SCM model.

As shown in Figure \ref{fig:transformations}, for an event of type {\sc\em Pre} the corresponding transformed rule is of type {\sc Precondition}; for an event of type {\sc\em Post} the corresponding transformed rule is of type {\sc Postcondition}; finally, for an event of type {\sc\em TimeRestriction} the corresponding transformed rule is of type {\sc Time}. 
The condition expression of a rule in the $\pi$-SCM ({\sc\em Rule:condition}) is transformed into a class {\sc\em Condition:expression} where the attributes of the expression are transformed into elements of type {\sc Attribute}.

%The attribute event of a rule  ({\sc\em Rule:event}) in the $\pi$-SCM is transformed into an {\sc Event Type} according to the rule type. 

%As shown in Figure \ref{fig:transformations}, the event type for a rule of type (i) {\sc Precondition} is {\sc ActivityPrepared}; (ii) {\sc Postcondition} is {\sc TermActivity}; (iii) {\sc TimeRestriction} is {\sc Temporal}. The {\sc\em Rule:Action} of a rule in the $\pi$-SCM is transformed into an {\sc Action:type}.

%
%Figure \ref{fig:p-scim} shows the  $\pi$-{\sc Pews} model for our example.
In the scenario "To Publish Music" the {\sf Policies} {\em OAuthPolicy} and {\em HTTPAuthPolicy} of the $\pi$-SCM model are transformed into {\em A-policies} of type {\sf Precondition} of the $\pi$-{\sc Pews} model of the scenario. Thus in both cases the events are of type {\sf ActivityPrepared}. These policies, as stated in the $\pi$-SCM model, are associated to {\sf Activities}. In the corresponding transformation they are associated to {\sf Operation}s {\em PublishFacebook} and {\em PublishTwitter}.
%\begin{figure}[htpb]
%\centering{
%\includegraphics[width=0.78\textwidth]{figs/modeloPEWS}}
%\caption{$\pi$-{\sc Pews} generated model fo the "To Publish Music" application}
%\label{fig:p-scim}
%\end{figure}

%Figure \ref{fig:pewsexpression} shows the correspondence between the model and the statements that implement it, with a schematic representation of the business process.
%\begin{figure}
%\centering{
%\includegraphics[width=0.85\textwidth]{figs/pews-expression}}
%\caption{Pews program implementing the "To Publish Music" application}
%\label{fig:pewsexpression}
%\end{figure}

%*********************************************************************************************************
\subsection{$\pi$-SOD-M Environment}\label{sec:implementation}

This section  describes the $\pi$-SOD-M development environment that implements the generation of {\em A-policies}' based services' compositions. For a given services' based application, the process  consists in generating the  code starting from a $\pi$-SCM modeling an application. Note that the services' composition model is not modeled from scratch, but it is the result of a general process defined by the $\pi$-SOD-M method in which a set of models are built following a service oriented approach \cite{decastro1}.

%We used the Eclipse Modeling Framework (EMF) to implement the whole model transformation process \footnote {The EMF project is a modeling framework and code generation facility for building tools and other applications based on a structured data model.}. From a model specification described in XMI, EMF provides tools and runtime support to produce a set of Java classes for the model, along with a set of adapter classes that enable viewing and command-based editing of the model, and a basic editor.
%In order to automate the transformation we specified  transformation rules using the ATL model transformation language Finally, in order to generate code we  used Acceleo \footnote{http://www.acceleo.org/pages/home/en}.

%%..--..--..--..--..--..--..--..--..--..--..--..--..--..--..--..--..--..--..--..--..--..--..--..--..--..--..--..--..--..--..--..--..--..--..--..--..--..--
%\subsection{$\pi$-SOD-M Development Environment}
%%..--..--..--..--..--..--..--..--..--..--..--..--..--..--..--..--..--..--..--..--..--..--..--..--..--..--..--..--..--..--..--..--..--..--..--..--..--..--

Figure \ref{fig:policymanager} depicts a general architecture of the $\pi$-SOD-M Development Environment showing the set of plug-ins  developed in order to implement it. The environment implements the abstract architecture shown in Figure \ref{fig:sodm}. Thus, it consists of plug-ins implementing the $\pi$-SCM and $\pi$-{\sc Pews} meta-models used for defining models specifying services' compositions and their associated policies; and ATL rules for transforming  PSM models (model to model transformation) and finally generating code (model to text transformation).
\begin{figure}[htpb]
	\begin{center}
		\includegraphics[width=0.60\textwidth]{figs/architecture}
	\end{center}
		\caption{$\pi$-SOD-M Development Environment}
   \label{fig:policymanager}
\end{figure}
\begin{itemize}
\item 	We  used the Eclipse Modeling Framework (EMF) \footnote {The EMF project is a modeling framework and code generation facility for building tools and other applications based on a structured data model.}   for implementing the meta-models  $\pi$- SCM and $\pi$-{\sc Pews}. Then, starting form these meta-models, we  developed the models' plug-ins needed to support the graphical representation of the $\pi$- SCM and $\pi$-{\sc Pews} models ($\pi$-ServiceCompostion Model and $\pi$-PEWS Model plug-ins).

\item	 We used  ATL \footnote{http://eclipse.org/atl/. An ATL program is basically a set of rules that define how source model elements are matched and navigated to create and initialize the elements of the target models.}
for  developing the mapping plug-in implementing the  mappings between models ($\pi$-ServiceComposition2$\pi$-PEWS Plug-in).

\item 	We  used Acceleo \footnote{http://www.acceleo.org/pages/home/en} for implementing  the code generation plug-in. We coded the pews.mt program  that implements the model to text transformation for generating executable code. It takes as input a $\pi$-PEWS model implementing a specific services' composition and it generates the code to be executed by the 
{\em A-policy} based services' composition execution environment. 

%\item Finally, we created a chain execution  to execute the model to text transformation.
\end{itemize}

%
As  shown in Figure \ref{fig:policymanager}, once an instance of a PEWS code is obtained starting form a particular $\pi$-services' composition model it can be executed over {\em A-policy} based services' composition execution environment  consisting of a composition engine and a {\em A-policy} manager.  The  {\em A-policy} manager  consists of three main components Manager, for scheduling the execution of rules, C-Evaluator and A-Executor respectively for evaluating rules' conditions and executing their actions. The {\em A-policy} Manager interacts with a composition engine thanks to a  message communication layer (MOM).


The composition engine manages the life cycle of the composition. Once a composition instance is activated, the engine schedules the composition activities according to the composition control flow.
Each activity is seen as the process where the service method call is executed.
The execution of an activity has four states: prepared, started, terminated, and failure.
The execution of the control flow (sequence, and/or split and join) can also be prepared, started, terminated and raise a failure.

At execution time, the evaluation of policies done by the {\em A-policy} manager must be synchronized with the execution of the services' composition (i.e., the execution of an activity or a control flow).  Policies associated to a scope are activated when the execution of its scope starts. A {\em A-policy} will have to be executed only if one or several of its rules is triggered. If several rules are triggered the {\em A-policy} manager first builds an execution plan that specifies the order in which such rules will be executed according to the strategies defined in the following section. 
%Once rules have been executed, the {\em A-policy} finishes its execution and returns to a sleeping state.
If rules belonging to several policies are triggered then policies are also ordered according to an execution plan. The execution of policies is out of the scope of this paper, the interested reader can refer to \cite{Espinosa-Oviedo2011a} for further details.
%The order of policies has implications on the global order of the rules to be executed.


%*********************************************************************************************************


\section{Applying $\pi$-SOD-M: The \FlyingPig\ use case}
\label{sec:flyingPig}
%4.1 Presentar los modelos para el caso y explicar las transformaciones
%4.2 Discussion. (o Lecciones aprendidas del caso)
We validated our methodology by developing a use-case concerning risk assessment for financial companies.
Our application is inspired on the ORCA System\footnote{The ORCA System is a trademark of GCP Global (www.gcpglobal.com).}.
Risk assessment is implemented by an interactive business process based on the exchange of a series of questionnaires intended to evaluate the risks implied the client's business practice.
For instance, the conditions and protocols used to perform confidential transactions, the physical security for accessing reserved areas such as computing server installations.
The information gathered by the questionnaires is used to determine whether there are risky practices within the business processes of the company, as well as to propose amends to these practices.
The ultimate goal of the risk assessment is to determine a degree of compliance to existing standards.
By analysing the questionnaires, ORCA detects risky practices, proposes solutions and triggers further assessment processes to ensure that the solutions have been implemented.

Our goal is to model a service based application (called \FlyingPig), for providing risk assessment as a service.
In order to provide this functionality, \FlyingPig\ would benefit from ORCA's legacy services: storage, assessment and data visualization functions.

In the following, we describe the results of applying $\Pi$-SODM to develop the \FlyingPig\ risk assessment system.
The models presented next were generated as a result of interacting with software developers at GCP Global.

\begin{figure}
\centering
\includegraphics[width=0.7\textwidth]{figs/3ValueModel.pdf}
\hspace*{5cm}\includegraphics[width=0.4\textwidth]{figs/3ValueKey.pdf}
\caption{E3value model for \FlyingPig.\label{fig:E3valuemodel}}
\end{figure}


\subsection{Computation-Independent Models (CIM)}

$\Pi$-SODM uses two models at the CIM level (see Section~\ref{sec:modelingWithPISODM}): The E3value~\cite{e3value} model and BPMN~\cite{BPMN}.
The former is a simple model just to identify the transference of value information between components of the system.
The BPMN model establishes which are the actors and main tasks of the application.

Figure~\ref{fig:E3valuemodel} shows the value model for the \FlyingPig\ application.
It is a business model that graphically represents a business case as a set of value exchanges ($\triangleright$ and $\triangleleft$) and value activities (rounded boxes) performed by business actors (squared boxes).

In our use-case, we identify two business actors: \textsl{ORCA} and \textsl{Broker}. 
Brokers are responsible for channelling requests for risk assessment of one or several companies. 
ORCA have two value activities which are services that provide an economical benefit: to \textsl{Identify Amendments} and the possibility to \textsl{Assess Risk Situation}. 
The values exchanged between ORCA and the brokers are: \textsl{Amendments} and \textsl{Evaluation Reports} which are value objects ([ \!\dots]) for the companies that need to have a risk assessment, as well as  \textsl{Questionnaire and Evidences} and the risk assessment \textsl{Fee} which are value objects for ORCA System.

The e3value defines \textit{dependency paths}, showing the value exchanges, which are triggered by the occurrence of an end-consumer need (in our case, the need of a risk assessment). 
A dependency path has a direction and consists of a sequence of linked dependency nodes.
A dependency path starts with a \textit{start stimulus} node and ends with an \textit{end stimulus} node (see Legend on Figure~\ref{fig:E3valuemodel}). 
Dependency paths may also contain \textsl{OR} and \textsl{AND} elements (both for initiate and join alternative and parallel paths).

The dependency path in Figure~\ref{fig:E3valuemodel} initiates with the need of assessment by a particular company. 
Once this need occurs, the value exchanges between ORCA and Broker are triggered. 
The client company will provide orca wit information (answers to a questionary), evidence (to support the information) and a fee (monetary value).
ORCA will provide amendments (recommendations to change practices) and an evaluation report. 

\begin{figure}[t]
\centering
\includegraphics[width=1.0\textwidth]{figs/BPMN_GCP.pdf}

{\color{red} Javier: Please change \underline{Asses} by \underline{Assess}. --M}
\caption{BPMN model for \FlyingPig.\label{fig:BPMNmodel}}
\end{figure}

The BPMN model is devised to better understand the process in which the value exchanges occur.
Figure~\ref{fig:BPMNmodel} shows the BPMN model\footnote{Details on BPMN (Business Process Management Notation) can be found in http://www.bpmn.org/.} for the \FlyingPig\ scenario. 
The model includes two pools representing the \textsl{ORCA} system and the \textsl{Brokers}. 
Brokers have two lanes, the client \textsl{Company} and a \textsl{User}. 
The user is a contact member of the company, who will coordinate the assessment process. 
This process will involve other members of the company as well.

The risk assessment process starts after a request from a company.
(This agrees with the value model, in which the start stimulus triggers the whole process.)
The request leads to the definition of a group of users that will answer questionnaires for evaluating risk.
Questionnaires are considered tasks that users will have to perform. 
Other tasks include amending a ``risky situation'' as well as producing evidence to show that a specific risk has been eliminated\footnote{Risky situations include from physical facts such as not having easy access to handicapped persons or having an unsecured access to the premises of the company, to more intangible ones, such as the use of an less-than-optimal protocol to access data on the company's computer server.}.

Once tasks are completed, they are stored and analysed to generate a list of un-compliant situations, associated to their corresponding \textit{calls for amendment}, in case there are any, or a report specifying a compliance level, incidents and a risk map.
During the process of analysing a questionnaire, the answers of some questions can trigger the generation of other questionnaires or amendments, that will become new tasks.  

Business processes have also associated rules and constraints that define their non functional requirements.
NFR represents the ``semantics'' and the conditions in which the tasks must be done.
In our example we have some constraints.

{\color{red}
Placido, Valeria: What can we say about non-functional requirements at the CIM level?
}



\begin{itemize}
\item {\color{magenta} Placido and Umberto: Could you please complete these items? --G \& M.}
\end{itemize}




\subsection{Platform-Independent Models (PIM)}

In Section~\ref{sec:modelingWithPISODM} we defined three models at the PIM level.
These models are built next, for the \FlyingPig\ scenario.

\begin{figure}[t]
\centering
\includegraphics[width=0.9\textwidth]{figs/UseCaseGeneral.png}

{\color{red} \raggedright
$\star$ ``Create a responsible'' should be changed for ``Designate user in charge''.

$\star$ ``requires concurrency for more than 200 users'' should be changed by ``Maximum number of users should be greater than 200''.
}
\caption{$\pi$-UseCase model for \FlyingPig.\label{fig:piUseCaseModel}}
\end{figure}


\paragraph{\underline{$\pi$-UseCase Model for \FlyingPig}}~

The $\pi$-UseCase model shown in Figure~\ref{fig:piUseCaseModel} describes the features and constraints for the \FlyingPig\ application. 
In this model, three actors are identified: \textit{Company}, \textit{User} and \textit{Channel-Broker}\footnote{\color{red} Can we change this for Broker ? --M.}. 
They are represented as stick figures.
In the context of \FlyingPig, Company is the actor asking for risk evaluation.
A Channel-Broker is the responsible for channelling the the evaluation process, assigning users to be in charge of tasks as well as delegating tasks. 
A User, in this model is an actor who answers questionnaires (with base on the actual facts about the Company).
The User also produces evidence to support facts and performs the necessary amendments to improve the results of the risk assessment.

Each actor is associated to one or more use cases (depicted as white ovals in Figure~\ref{fig:piUseCaseModel}). 
Use cases describe the main functionalities of the system.
The $\pi$-UseCase model for \FlyingPig\ defines six use cases. 

In our model, each use case may be associated to one or more (non-functional) constraints (depicted as coloured ovals in Figure~\ref{fig:piUseCaseModel}). 
The model defines three types of constraints: \textit{value} , \textit{business} or \textit{exceptional behavior}. 
Each constraint is identified by the word $<<$\textsf{constraint}$>>$ followed by its type.

In the case of \FlyingPig, the model counts seven constraints:
\begin{numtrivlist}
\item An acknowledgement is due less than 30 seconds after registering a task or demand for assessment. 
\item The system's infrastructure should be prepared to deal with, at least, 200 users. 
\item If the number of requests exceeds 200, \FlyingPig\ should make a load balance of the requests. 
\item The Channel-Broker must have their privileges verified \textit{before} the execution of the actions associated to the \textsf{designate user in charge} use case.
\item Users must have their privileges verified \textit{before} the execution of the actions associated to the \textsf{answer questionnaire and add evidences} use case.
\item All questionnaires need to be fully answered, in order to consider that a task is completed.
\item There is a time limit (in days) for each amendment required by the system.
\end{numtrivlist}

{\color{red} Anything else here? --M}

\begin{figure}
\centering
\includegraphics[width=1.0\textwidth]{figs/ServiceProcessGeneral.png}
\caption{$\pi$-ServiceProcess model for \FlyingPig.\label{fig:PiServiceProcessModel}}
\end{figure}

\paragraph{\underline{$\pi$-ServiceProcess Model for \FlyingPig}}~

The $\pi$-ServiceProcess model (Figure~\ref{fig:PiServiceProcessModel}) presents the workflow for \FlyingPig.
The actions in this model were obtained by applying the use case transformation rules described in Section~\ref{sec:pewsmetamodel}.

The Company, Broker-Channel and User actors are transformed into lanes that represent the business collaborators.
Use cases are transformed into \textit{actions} and are represented by white boxes.
The restrictions associated to each use case are transformed into \textit{assertions} (represented by coloured boxes) and may be decorated with pre- and post-conditions. 
The set of assertions related to a particular action form a \textit{contract} for this action. 

This model refines the concepts defined in the $\pi$-UseCase model and it is more suited to better describe the assertions, by grouping them into contracts.

\begin{figure}
\centering
\includegraphics[width=1.0\textwidth]{figs/ServiceCompositionGeneral.png}
\caption{$\pi$-ServiceComposition model for \FlyingPig.\label{fig:PiServiceCompositionModel}}
\end{figure}

\paragraph{\underline{$\pi$-ServiceProcess Model for \FlyingPig}}~

This model presented in Figure~\ref{fig:PiServiceCompositionModel} describes the interactions of the actions described in pi-ServiceProcess model with the external services. Services are provided through the ORCA system and the access is accomplished by FlyingPig interface. FlyingPig is the point of interaction between the actions described in the precess with the services offered by ORCA. FlyingPig provides an interface with five actions, they are: receive a request, generate new interface, create questionnaire, notify responsable e receive answers.. This interface makes the direct calls for the ORCA' services. The services offered by ORCA are: create space company, generate the questionnaire, analyze answer, data store, call for amendment and create report. These services are called according to the process execution.

With respect to non-functional requirements, the assertions grouped into contract in the pi-ServiceProcess model are transformed into policies in the pi-ServiceComposition model. The policies identified in this case study are: security, performance and conformability policies. Each policy is associated with specific the services and it is verified at the service execution time regarding the specific rule which the sercive is associated with..



\subsection{PSM}

{\color{magenta} Placido and Umberto: Could you please insert the $\pi$-PEWS model here? --G \& M.}


%%%%%%%%%%%%%%%%%%%%%%%%%%%%%%%%%%%%%%%%%%%%%%%%%%5
%% By Placido...
%%%%%%%%%%%%%%%%%%%%%%%%%%%%%%%%%%%%%%%%%%%%%%%%%%

Pi-UseCase Model description for FlyingPig

The piUseCase model shown in Figure  X models the use cases (features) and constrainsts for the FlyingPig application. 3 actors are identified in this model, the are: (I) Company, (II) User and the (III) Channel-Broker. The Company is the whom ask for risk evaluation, the (II) Channel-Broker is the responsible for managing a defining the evaluation flow, beyond create responsible for evaluating and delegate the tasks. This model also represents the (III) User actor which answers questionnaires for the assessment, including evidence and proceeding with necessary corrections when asked for this type of task.

Each actor is associated with use cases. Use cases represent the actions that the actors are responsible. Altogether, this model defines 6 (six) use cases. The use cases related to the Company's actor is (i) request assessment. The use cases related to Broker-Channel's actor are: (ii) create a responsable and (iii) delegate task; and the use cases related User's actor are: (iv) answer questionnaire and add evidences, (v) register task, and (vi) perform amendment.

In this model, each use case may be related with one or more constraints. The constraints can be of three types: value , business or exceptional behavior . This model as a whole has 7 constraints . The use case (i) request assessment has 2 constraints related with, the are from the value type. One is related to ( i.a) response time (confirmation) that should be up to 30 seconds , and the other ( i.b ) describes that the system should support a large number of requests from different companies . Moreover ( i.c ) if the number of requests exceeds 200 , FlyingPig should make a load balance of the requests. The use case create a responsable has a value constraint ( ii.d ) which defines that the Channel-Broker must have privileges verified before the execution to perform this action. This use case also ( ii.c ) is also related with the load balancing verification. The use case ( iii ) delegate task has no associated constraint . The use case answer questionnaire and add evidences have 2 constraints , they are: the User which performs this action ( iv.e ) must have privileges to answer the questionnaires and ( iv.f ) the time limit must be less than 15 days. The use case register task has 3 constraints associated . Two of them are the same constraints related with the request assessment use case, where (v.a) need a response time confirmation that should be up to 30 seconds, and the other ( v.b ) describes what the system should support a large number of requests for this action, once Users from various companies will be answering the questionnaire and performing tasks. The other constraint requires that ( v.g ) throughout the survey all the questions must be properly answered . Finally , the use case ( vi.f ) perform amendment must be executed in a set time limit in days, the same constraint related with the use case questionnaire answer and add evidences. Figure A presents the relation between the use case and its constraints.






TABLE A - RELATION BETWEEN USE CASE AND CONSTRAINT
use cases / constraints	(i) request assessment	(ii) create e responsable 	(iii) delegate task	(iv) answer questionnaire and add evidences	(v) register task	(vi) perform amendment
(a) confirmation must be received in up to 30sec	X				X	
(b) requires concurrency for more than 200 users	X				X	
(c) load balance if there is more than 200 requests	X	X				
(d) requires Broker’ privileges		X				
(e) requires User' privileges				X		
(f) all tasks must be done within a given time limit (e.g. maximum of 15 days)				X		X
(g) requires questionnaire completion					X	


Pi-ServiceProcess Model description for FlyingPig 

The pi-ServiceProcess model (Figure Y) represents the application execution flow of FlyingPig. In this model the action are the result of the use case transformation identified in pi-UseCase model according to their respective actors. The Company, Broker-Channel and User actors are transformed into rays that represent the business collaborators.

The restrictions associated with each use case are transformed into assertions. The set of assertions related to a particular action form the contract for this action. The process of FlyingPig executing begins with the action ( i ) request assessment performed by the Company. This action has one pre-condition regarding the response time and a post-condition concerning the load balance control for the server when it reaches 200 requests. Once the request is made by the Company for risk analysis, the Broker ( ii ) creates responsible for its analysis and ( iii ) delegate the necessary risk assessment. These actions are executed in sequence and has a pre-condition concerning the Broker's privilege. This can only create responsible and delegate the tasks if their authentication datas are correct and if and there was less than 3 requests. The tasks to be performed by the User can be ( iv ) answer questionnaire and add evidences; or ( v) perform amendments. Both the assertion has one pre-condition concerning the time limit for form submission. In this example, we use the limit of 15 days for form submission. If the User is performing the task of answer questionnaire and add evidences, this will have to enter with their authentication data. As postcondition, it is checked if all the answers were answered . The action of perform amendments has no additional verification. Finally, the user registers the tasks performed. This action has response time verification.

This model refines the concepts defined in the pi-UseCase model in order to better describe the assertions and groups them into contracts.


Pi-ServiceComposition Model description for FlyingPig 

This model (presented in Figure Z) describes the interactions of the actions described in pi-ServiceProcess model with the external services. Services are provided through the ORCA system and the access is accomplished by FlyingPig interface. FlyingPig is the point of interaction between the actions described in the precess with the services offered by ORCA. FlyingPig provides an interface with five actions, they are: receive a request, generate new interface, create questionnaire, notify responsable e receive answers.. This interface makes the direct calls for the ORCA' services. The services offered by ORCA are: create space company, generate the questionnaire, analyze answer, data store, call for amendment and create report. These services are called according to the process execution.

With respect to non-functional requirements, the assertions grouped into contract in the pi-ServiceProcess model are transformed into policies in the pi-ServiceComposition model. The policies identified in this case study are: security, performance and conformability policies. Each policy is associated with specific the services and it is verified at the service execution time regarding the specific rule which the sercive is associated with..





%%%%%%%%%%%%%%%%%%%%%%%%%%%%%%%%%%%%%%%%%%%%%%%%%%%%%
\subsection{Lessons Learned}

Through the example we underlined that every application implements functional aspects that describe its application logic.
Recall that an application logic refers to routines that perform the activities to reach the application objective.
Also there are non functional properties derived from NFR. They refer to strategies to be considered for the application execution like: security, isolation, adaptability, atomicity, and more.
These non functional properties must be ensured at execution time, and they are not completely defined within the application logic.

The challenge is to define them and to associate them with the application logic considering that different to existing solutions that suppose that it is possible to access the execution stat of all the components  of an application and that the application has complete control on them, in the case for service oriented applications  the components are autonomous services
API does not necessarily export information about methods dependency (e.g., in the REST protocol);
they do not share their state (stateless).

Given a set of services with their exported methods known in advance or provided by a  service directory, building services' based applications can be  a simple task that implies expressing an application logic as a services' composition. The challenge being  ensuring the compliance between the specification and the resulting application. Software engineering methods (e.g., \cite{1,2,decastro1,PapazoglouH06}) today can help to ensure this compliance, particularly when information systems include several sometimes complex business processes calling Web services or legacy applications exported as services.



%\appendix


%*********************************************************************************************************
\section{Conclusions and future work}\label{sec:conclusions}
We presented \pisodm, an model-driven method for designing and developing reliable service-based applications.
\pisodm extends a previously defined method (called SOD-M) to include Non-Functional Requirements.
These requirements are taken into account from the early stages of the software development process.
Non-functional constraints are related to business rules associated to the behavior of the application and, in the case of service-based applications, they are also concerned with constraints imposed by the services.

Our method includes two CIM-level models, three PIM-level models and one PSM-level model.
We implemented the meta-models on the Eclipse platform and we validated the approach by using an industrially inspired use case.

Our case study demonstrates the applicability of \pisodm.
This case study was developed together with our industrial partner, GCP Global.
The Company is using \pisodm for the development of their product.
The case study presented here is a simplified version of their application. 

%% The Appendices part is started with the command \appendix;
%% appendix sections are then done as normal sections
%% \appendix

%% \section{}
%% \label{}

%% References
%%
%% Following citation commands can be used in the body text:
%% Usage of \cite is as follows:
%%   \cite{key}          ==>>  [#]
%%   \cite[chap. 2]{key} ==>>  [#, chap. 2]
%%   \citet{key}         ==>>  Author [#]

%% References with bibTeX database:

\bibliographystyle{plain}
\bibliography{biblio}

%% Authors are advised to submit their bibtex database files. They are
%% requested to list a bibtex style file in the manuscript if they do
%% not want to use model1a-num-names.bst.

%% References without bibTeX database:

% \begin{thebibliography}{00}

%% \bibitem must have the following form:
%%   \bibitem{key}...
%%

% \bibitem{}

% \end{thebibliography}

\end{document}

%%
%% End of file `elsarticle-template-1a-num.tex'.
