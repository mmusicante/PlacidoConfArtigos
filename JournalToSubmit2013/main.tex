\documentclass[preprint,12pt]{elsarticle}
\usepackage{geometry}
%\geometry{letterpaper}                   % ... or a4paper or a5paper or ...
\usepackage{graphicx}
\usepackage{amssymb}
\usepackage{epstopdf} 
%% Use the option review to obtain double line spacing
%% \documentclass[preprint,review,12pt]{elsarticle}

%% Use the options 1p,twocolumn; 3p; 3p,twocolumn; 5p; or 5p,twocolumn
%% for a journal layout:
%% \documentclass[final,1p,times]{elsarticle}
%% \documentclass[final,1p,times,twocolumn]{elsarticle}
%% \documentclass[final,3p,times]{elsarticle}
%% \documentclass[final,3p,times,twocolumn]{elsarticle}
%% \documentclass[final,5p,times]{elsarticle}
%% \documentclass[final,5p,times,twocolumn]{elsarticle}

%% if you use PostScript figures in your article
%% use the graphics package for simple commands
%% \usepackage{graphics}
%% or use the graphicx package for more complicated commands
%% \usepackage{graphicx}
%% or use the epsfig package if you prefer to use the old commands
%% \usepackage{epsfig}

%% The amssymb package provides various useful mathematical symbols
\usepackage{amssymb}
%% The amsthm package provides extended theorem environments
%% \usepackage{amsthm}

%% The lineno packages adds line numbers. Start line numbering with
%% \begin{linenumbers}, end it with \end{linenumbers}. Or switch it on
%% for the whole article with \linenumbers after \end{frontmatter}.
%% \usepackage{lineno}

%% natbib.sty is loaded by default. However, natbib options can be
%% provided with \biboptions{...} command. Following options are
%% valid:

%%   round  -  round parentheses are used (default)
%%   square -  square brackets are used   [option]
%%   curly  -  curly braces are used      {option}
%%   angle  -  angle brackets are used    <option>
%%   semicolon  -  multiple citations separated by semi-colon
%%   colon  - same as semicolon, an earlier confusion
%%   comma  -  separated by comma
%%   numbers-  selects numerical citations
%%   super  -  numerical citations as superscripts
%%   sort   -  sorts multiple citations according to order in ref. list
%%   sort&compress   -  like sort, but also compresses numerical citations
%%   compress - compresses without sorting
%%
%% \biboptions{comma,round}

% \biboptions{}


\journal{Journal of Systems and Software}

%%% OUR MACROS %%%
\newcommand{\COMMENT}[1]{ } 

\begin{document}

\begin{frontmatter}

%% Title, authors and addresses

%% use the tnoteref command within \title for footnotes;
%% use the tnotetext command for the associated footnote;
%% use the fnref command within \author or \address for footnotes;
%% use the fntext command for the associated footnote;
%% use the corref command within \author for corresponding author footnotes;
%% use the cortext command for the associated footnote;
%% use the ead command for the email address,
%% and the form \ead[url] for the home page:
%%
%% \title{Title\tnoteref{label1}}
%% \tnotetext[label1]{}
%% \author{Name\corref{cor1}\fnref{label2}}
%% \ead{email address}
%% \ead[url]{home page}
%% \fntext[label2]{}
%% \cortext[cor1]{}
%% \address{Address\fnref{label3}}
%% \fntext[label3]{}

\title{$\pi$SOD-M: A Methodology for Building Service-Oriented Applications in
the Presence of Non-Functional Properties}

%% use optional labels to link authors explicitly to addresses:
%% \author[label1,label2]{<author name>}
%% \address[label1]{<address>}
%% \address[label2]{<address>} 
 
\author[inst1]{Pl\'acido A. Souza Neto}
\author[inst2]{Genoveva Vargas-Solar}
\author[inst3]{Martin A. Musicante}
\author[inst4]{Valeria~de~Castro}
\author[inst3]{Umberto Souza da Costa}

\address[inst1]{Federal Technological Institute of Rio Grande do Norte -- Natal, Brazil}

\address[inst2]{Universit\'e de Grenoble -- Saint Martin d'H\`{e}res, France}

\address[inst3]{Federal University of Rio Grande do Norte -- Natal, Brazil}

\address[inst4]{Universidad Rey Juan Carlos -- M\'{o}stoles, Spain}

\begin{abstract}
This paper presents\ldots
\end{abstract}

\begin{keyword}
%% keywords here, in the form: keyword \sep keyword

%% MSC codes here, in the form: \MSC code \sep code
%% or \MSC[2008] code \sep code (2000 is the default)

\end{keyword}

\end{frontmatter}

%%
%% Start line numbering here if you want
%%
% \linenumbers

%% main text
%*********************************************************************************************************
\section{Introduction}
\label{sec:intro}

Model Driven Development (MDD) is a top-down approach for the development of software systems. 
The main ideas of MDD were originally proposed by the Object Management Group (OMG)~\cite{mda}, as a set of guidelines for the structuring of specifications.
The technique advocates for the use of \textit{models} to specify a software system at different levels of abstraction (called \textit{viewpoints}):

\begin{trivlist}
\item \textbf{Computation Independent Models (CIM):} This level focusses on the
environment of the system, as well as on its business and requirement specifications. 
This viewpoint represents the software system at its highest level of abstraction. 
At this moment of the development, the structure and system processing details are still unknown or undetermined. 
 
\item \textbf{Platform Independent Models (PIM):} This level focusses on the system functionality, hiding the details of any particular platform. 
The specification defines those parts of the system that do not change from one platform to another. 

\item \textbf{Platform Specific Models (PSM):} This level focusses on the functionality, in the context of a particular implementation platform.
Models at this level combine the platform-independent view with the specific aspects of the platform to implement the system.  
\end{trivlist}

Besides the notion of model at each level of abstraction, MDD requires the use of \textit{model transformations} within and between levels.
Intra-level transformations are used to provide a unified representation of concepts of a given level.
Inter-level transformation implement a refinement process between levels.
Transformations may be automatic or semi-automatic.

MDD techniques has been successfully used for the development of hardware and software systems~\cite{MDDvariosAqui}. 
In particular, we are interested in the application of MDD to the design and implementation of web service applications.

Service oriented computing~\cite{Papazoglou2007} is at the origin of an evolution in the field of software development~\cite{??}. 
An important challenge of service oriented development is  to ensure the alignment between the requirements imposed by the business logic and the IT systems actually developed.
(Moreover, IT systems need to evolve according to the business needs.)
Thus, organizations are seeking for mechanisms to bridge the gap between the systems developed and business needs~\cite{bell}. 
The literature stresses the need for methodologies and techniques for service oriented analysis and design, claiming that they are the cornerstone  in the development of meaningful service based applications~\cite{5}.  
In this context, we argue that the convergence of model-driven software development, service orientation and better techniques for documenting and improving business processes are key to make real the idea of rapid, accurate development of software that serves, rather than dictates the needs of its users~\cite{watson}. 

In Service-Oriented Computing, pre-existing services are
combined to produce applications and provide the business logic. 
The selection of services is usually guided by the \textit{functional} requirements of the application being developed~\cite{1,2,decastro1,PapazoglouH06}. 
(Functional properties of a computer system are characterized by the effect produced by the system when given a defined input.)
Functional properties are not the only crucial aspect in the software development process. 
Other properties need to be addressed to fit in the application with its context.
These other aspects are called Non-Functional Properties.

Non-functional aspects of the services, often expressed as requirements and constraints in general purpose methodologies, are not usually considered from the beginning of the (service) software process.
Most methods consider them only after the application has been implemented, in order to ensure some level of reliability (e.g., data privacy, exception handling, atomicity, data persistence). 
This leads to service based applications that are partly specified and, thereby, partly compliant with the requirements of the application.
Ideally, non-functional requirements should be considered along with all the stages of the software development. 
The adoption of non-functional specifications from the early states of development
can help the developer to produce applications that are capable of dealing with
the application context.

In this work, we are interested in the extension of the \textit{Service Oriented Development Method} (SOD-M)~\cite{decastro1}, to support non-functional aspects, from the early stages of software development.
SOD-M is aligned with the MDD directives and proposes models, practices and techniques for the development of service-based applications.
SOD-M does not provide support for the specification of non-functional requirements, such as
security, reliability, and efficiency. 

The main goals of our work are:
\begin{trivlist}
\item \textit{(i)} To define a NFR model including a set of concepts need for the modeling of NFR in service-oriented applications.
\item \textit{(ii)} To propose a methodology for supporting the construction of service-oriented applications, taking into account both functional and non-functional requirements;
\item \textit{(iii)} To improve the construction process by providing an abstract view of the application and ensure the conformance to its specification;
\item \textit{(iv)} To reduce the programming effort through the semi-automatic generation of  models for the application, to produce concrete implementations from high abstraction models;
\end{trivlist}

The rest of the paper is organized as follows. 
Section\dots


\begin{center}
\textsc{\underline{Version from CAISE submission}}
\end{center}


Service oriented computing is at the origin of an evolution in the field of software development. 
An important challenge of service oriented development is  to ensure the alignment between IT systems and the business logic.
%dealing thereby with the promise that IT systems can  evolve  according to the business needs. 
Thus, organizations are  seeking for mechanisms to deal with the gap between the systems developed and business needs \cite{bell}. The literature stresses the need for methodologies and techniques for service oriented analysis and design, claiming that they are the cornerstone  in the development of meaningful services' based applications \cite{5}.  In this context, some authors argue that the convergence of model-driven software development, service orientation and better techniques for documenting and improving business processes are the key to make real the idea of rapid, accurate development of software that serves, rather than dictates, software users' goals \cite{watson}. 

Service oriented development methodologies providing models, best practices, and reference architectures to build services' based applications mainly address  functional aspects \cite{1,2,decastro1,PapazoglouH06}.  Non-functional aspects concerning services' and application's "semantics", often expressed as requirements and constraints in general purpose methodologies, are not fully considered or they are added once the application has been implemented in order to ensure some level of reliability (e.g., data privacy, exception handling, atomicity, data persistence). This leads to services' based applications that are partially specified and that are thereby partially compliant with application requirements.

The objective of this work   is to model non-functional constraints and associate them to  services' based applications  early during the services' composition modeling phase. Therefore this paper presents $\pi$-SOD-M, a model-driven method  that extends the SOD-M  \cite{decastro1} for building reliable  services' based information systems (SIS). 
%SOD-M defines a process  starting with the  identification of business services through business modeling, and, by means of models' transformations it allows to obtain a services' composition model \cite{decastro1} and the executable code that implements it. 

Our work proposes to extend the SOD-M \cite{decastro1} method with  (i)  the notion of {\em A-Policy} \cite{Espinosa-Oviedo2011a} for representing non-functional constraints associated to services' based applications.  
%{\em A-policies} are used to express constraints which can be applied to all the services' composition  or to a particular service used for implementing it. They represent both systems' cross-cutting aspects (e.g., exception handling expressing what to do when a service is not available) and use constraints imposed by the services  (e.g., the fact that a service requires imposes an authentication protocol for executing a method). 
Our work  also (ii) defines the $\pi$-{\sc Pews}  meta-model \cite{Placido2010LTPD} providing guidelines for expressing the composition and the {\em A-policies}. Finally, our work (iii) defines model to model transformation rules for generating the $\pi$-{\sc Pews} model of a reliable services' composition starting from the extended services' composition model; and, model to text transformations for generating the corresponding implementation. As will be shown within our environment implementing these meta models and rules, one may represent both systems' cross-cutting aspects (e.g., exception handling for describing what to do when a service is not available, recovery, persistence aspects) and constraints associated to services, that must be respected for using them (e.g., the fact that a service requires an authentication protocol for executing a method). 

The remainder of the paper is organized as follows. Section \ref{sec:motivation} gives an overview of our approach. It describes a motivation example that integrates and synchronizes well-known social networks services namely Facebook, Twitter and, Spotify. Sections \ref{sec:piscm}, \ref{sec:pewsmetamodel}, and \ref{sec:mmrules} describe respectively the three key elements of our proposal, namelly the $\pi$-SCM and $\pi$-{\sc Pews} meta-models and the transformation rules that support the semi-automatic generation of reliable services' compositions.
%describes $\pi$-SOD-M method that enables the representation and association of {\em A-policies} to services' composition  thereby making them reliable. 
%
Section \ref{sec:implementation} describes implementation and validation issues.
Section \ref{sec:related} analyses related work concerning policy/contract based programming and, services' composition platforms. Section \ref{sec:conclusions} concludes the paper and discusses future work.








%*********************************************************************************************************

%*********************************************************************************************************
\section{Related Works}
\label{sec:relworks}
%Conceptos existentes en NFR y metodologías similares
While Functional Requirements establish \textit{what} is computed by an application, Non-Functional Requirements (NFRs) are concerned with \textit{how} the task is preformed.  
NFRs include aspects such as performance, authentication and quality constraints.
Non-functional properties are also referred to as constraints, quality attributes��, quality goals��, quality of service requirements and ��non-behavioural requirements�� \cite{StGr05}. 
In the case of service-based applications, non-functional requirements concern the application itself as well as its component services. 
%The majority of the works on NFPs focus on measuring in which extent a software system  fulfills the NFPs that it should satisfy \cite{BBKL78,FePf96,KiDa96,Lyu96,MuIO90}. 
Most research on NFRs focus on the evaluation of compliance by the software system. 

In~\cite{Babamir2010,Yeom2006} non-functional properties of web services are classified according to three points of view, namely,  
\textit{service level}, \textit{system level} and \textit{business level}.
In~\cite{Babamir2010} NFRs are denoted as \textit{quality constraints}, which are expressed as logic formulae.
In~\cite{Yeom2006} authors classify NFRs into \textit{category}, \textit{sub-category} and \textit{property}. Categories include \textit{business}, \textit{service} and \textit{system}.
Possible \textit{sub-categories} are \textit{security}, \textit{value} or  \textit{interoperability}. The work also defines a \textit{web service quality model}, which considers non-functional properties. 

In~\cite{XiaoCZBOLH08} the authors use the terms  
\textit{non-functional attributes}, \textit{composition mo\-del entity} and \textit{mo\-del entity}  to classify different concepts related to NFRs.
The notion of non-functional attribute is used to describe NFRs of the abstract process model. In the lower level, the composition is annotated with non-functional attributes.

D'Ambrogio~\cite{DAmbrogio06} uses the term \textit{quality category} to group similar \textit{quality characteristics}. 
\textit{Quality dimensions} are used to quantify an individual characteristic.
For instance, the quality category \textit{performance} groups characteristics such as
\textit{latency} and \textit{throughput}. 
The development process is based on MDA and the authors also present a WSDL extension for describing the QoS of web services. A catalog of \textit{QoS characteristics} is provided for the web service domain, including properties as \textit{availability}, \textit{reliability} and \textit{access control}. 

 
Schmeling et al.~\cite{SchmelingCM11} present an approach and a toolset for specifying and implementing web service compositions with support to several NFRs. The term \textit{non-functional concern} (NFC) is used to denote  NFRs. 
%Two aspects are considered: specification and realization.
\textit{Non-functional concern} is a general term used to describe non-functional requirements. 
For instance, \textit{security}, \textit{reliability}, \textit{transactional behavior} are non-functional requirements. 
A \textit{non-functional action} represents some behavior that implements \textit{non-functional attributes}. 
An example of \textit{non-functional action} is \textit{encryption}, which provides the implementation of the \textit{non-functional attribute} \textit{confidentiality}. 
Non-functional actions related to a common concern are grouped into \textit{non-functional activities}. 

Pastrana et al.~\cite{PastranaPK11} use the term \textit{contract} to describe non-functional requirements. 
\textit{Contracts} may have pre-conditions, post-conditions and invariants. 
Each contract defines \textit{assertions} associated with \textit{quality properties}. 
Each service may have as many associated \textit{contracts} as needed.

Chollet et al.~\cite{CholletL09} associate (non-functional) \textit{quality properties} to 
(functional) activities. The authors present a security meta-model for web service
composition. The NFRs considered are \textit{authentication}, \textit{integrity} and \textit{confidentiality}. 
Each NFR is associated with a service activity.


Ceri et al.\cite{CeriDMF07} uses the notions of \textit{policy}, \textit{rule}, \textit{condition} and \textit{action model} to specify NFRs.
Agarwal et al.~\cite{AgarwalLS09} associate \textit{service policies} to services. 
Each service may also have \textit{properties}, such as \textit{security} and \textit{reliability}. 
Ovaska et al.~\cite{OvaskaEHPA10} use the terms \textit{quality attribute}, \textit{category}, \textit{conceptual layer} and \textit{importance} to organize and classify NFRs.
Other authors do not define specific terms to refer to NFRs. 
They use terms such as \textit{attribute}~\cite{ZhangPSP05,BasinDL06,JeongCL09}, 
\textit{property}~\cite{Fabra2011}, 
\textit{factor}~\cite{MohantyRP10,GutierrezRF10}, 
\textit{characteristic}~\cite{DiamadopoulouMPS08}, 
\textit{quality level}~\cite{ModicaTV09}, and
\textit{value}~\cite{ThissenW06,BasinDL06}.


Despite of the different notations found in the literature for classifying NFRs, some non-functional requirements are frequently considered, such as \textit{security}, \textit{performance}, \textit{reliability}, \textit{usability}, and \textit{availability}.
However, distinct hierarchies and models are proposed for NFRs,  according to different points of view.
We have identified a number of approaches~\cite{DAmbrogio06,CholletL09,SchmelingCM11,BasinDL06,Fabra2011,OvaskaEHPA10} that use MDD (Model Driven Development) for designing and developing applications. 

Fabra \textit{et al.}~\cite{Fabra2011} also describes the importance of  MDD for service-oriented applications. This work  presents a complete development methodology, although this methodology is not centered on NFRs.
The authors in~\cite{ThissenW06,ZhangPSP05} use formal methods to define a service-based development process that takes NFRs into account. 
In~\cite{AgarwalLS09,PastranaPK11} ontologies are used to define and model NFRs, 
whereas in~\cite{XiaoCZBOLH08,GutierrezRF10} Business Process Modeling (BPM) is used for
system specification, including NFRs. 
The majority of the authors concentrate on the  modeling of service compositions, although a significant number of approaches is focused on the definition of NFR models.


In the method defined in~\cite{XiaoCZBOLH08}, tasks in the process model can be 
annotated with \textit{non-functional attributes} (NFAs). 
NFAs are defined apart  and are concerned with data items or tasks. 
NFAs for data considers \textit{value} and \textit{range}, whereas NFAs for tasks include \textit{cost}, \textit{time}, \textit{resources} and \textit{expressions}.

The proposal in~\cite{ThissenW06} presents steps for  selecting services 
by taking QoS information into account. The proposed steps are: 
\textit{(i)} identification of relevant QoS information; 
\textit{(ii)} identification of basic composition patterns and 
QoS aggregation rules for these patterns; and 
\textit{(iii)} definition of a selection mechanism of services. 
The QoS properties considered are \textit{performance}, \textit{cost}, \textit{reliability} and
\textit{availability}. 
  
Karunamurthy et al.~\cite{Karunamurthy2012787} use the term \textit{non-function parameters} to define NFRs, such as \textit{cost}, \textit{response time}, \textit{availability}, \textit{security}, \textit{reliability} and \textit{reputation}.  
The \textit{Non-Func\-tion\-al Specification Language} (NFSL) is proposed as a domain specific language (DSL) to express \textit{non-function parameters}.

Liu et al.~\cite{Liu20121080} use the term \textit{QoS parameter} to describe non-functional requirements such as \textit{cost}, \textit{execution duration}, \textit{accuracy}, \textit{security}, \textit{integrity}, \textit{availability} and \textit{reliability}.  
In the same way, Tran et al.~\cite{Tran2012531} use the term \textit{QoS policies} to classify similar non-functional requirements.

Li et al.~\cite{Li2013} associate \textit{dimensions} to  \textit{QoS parameters} to classify NFRs.  
For instance, the \textit{time} dimension is associated to the \textit{execution time} and \textit{communication time} parameters; the \textit{spatial} dimension is associated to the \textit{storage capacity} and \textit{message length} parameters; the \textit{reliability} dimension is associated to the \textit{availability} and \textit{reliability} parameters and the \textit{cost} dimension is associated to the \textit{service cost} parameter.
Rumpel et al.~\cite{Rumpel2012}  associate \textit{quality requirements} to  \textit{quality properties}. Quality requirements are intended to be specified as constraints. 

\bigskip
Most works agree on distinguishing three points of view, namely the point of view of the organization (or Business view), of the individual service providers (or Service view) and of the composition designer (or System view).
The Business view is concerned with the business logic (\textit{i.e.}, an abstract level of tasks, defined by the guidelines and constraints imposed by the organization).
Service and System views are concerned with the implementation of the software solution: The Service level is concerned with the building blocks of the application.
It may use web services provided by third party sources.
The System level is concerned with the coordination of services, to implement the business logic.


\subsection{Modeling non-functional requirements in service-oriented applications}
\label{sec:modeling}
%(comentar la revisión y presentar la clasificacion de conceptos y NFR model con los conceptos que proponemos) 
%(comentar la revisión y presentar la clasificacion de conceptos y NFR model con los conceptos que proponemos)


\subsection{Methodologies for Service-oriented applications development including non-functional requirements}
\label{sec:methodologies}
%(aqui propuestas relacionadas, analizar sus caracteristicas, si son MDD, si parte de alto nivel, etc.)
%(aqui propuestas relacionadas, analizar sus caracteristicas, si son MDD, si parte de alto nivel, etc.)
% De Valeria: la explicación de las caracterísiticas de las propuestas relacionadas está ya en Related Works.




%*********************************************************************************************************

%*********************************************************************************************************
\section{Modeling reliable services' compositions with $\pi$-SOD-M}\label{sec:motivation}
%(Aquí lo dejaría con la estructura que teníamos para Caise, presentamos el framework y posteriormente comentamos modelos y transformaciones)

%(Aquí lo dejaría con la estructura que teníamos para Caise, presentamos el framework y posteriormente comentamos modelos y transformaciones)

\subsection{$\pi$-SOD-M meta-models}\label{sec:pisodmmetamodels}
%3.1. pi-SoD-M metamodels (presentar los metamodels)
%3.1. pi-SoD-M metamodels (presentar los metamodels)


The {\em A-policy} based service composition meta-model (Figure \ref{fig:e-scomposition-metamodel})
provides classes to represent workflows\footnote{Workflows will be transformed into implemented service compositions.}.
The meta-model identifies {\sc Business Collaborators}\footnote{We use {\sc capitals} for referring to meta-model classes.} and the {\sc Actions} they perform. 
Instances of this meta-model are UML activity diagrams. 
Figure~\ref{fig:e-scomposition-metamodel} show the modeling elements.
Those classes pictured as coloured boxes deal with non-functional properties.
The classes pictured with a white background are those of SOD-M.


In the meta-model of Figure~\ref{fig:e-scomposition-metamodel}:
\begin{itemizedTrivlist}
\item A {\sc Business Collaborator} element represents those entities that collaborate in the business processes by performing some of the required actions. 
They are graphically presented as a partition in the activity diagram. 
A collaborator can be either internal or external to the system. 
When the collaborator of the business is external to the system, the attribute {\sf IsExternal}\footnote{We use the {\sf sans serif} font for referring to classes defined using a meta-model.} of the collaborator is set to \textbf{true}.

\item {\sc Action}s, a kind of {\sc ExecutableNode}, are represented in the model as an activity. 
Each action represents some type of transformation or processing. 
There are two types of actions: i) a WebService (attribute Type is {\sf WS}); and ii) a simple operation that is not supported by a Web Service, called an {\sc ActivityOperation} (attribute Type is {\sc AOP}).
\begin{figure}[t]
\centering
\includegraphics[width=1.0\textwidth]{figs/E-service-composition-metamodel}
\caption{{\em A-policy} based service composition meta-model ($\pi$-SCM)}
\label{fig:e-scomposition-metamodel}
\end{figure}

\item The {\sc ServiceActivity} element is a composite activity that must be carried out as part of a business service and is composed by one or more executable nodes.

\item In order to represent constraints associated to services compositions, we extended the SOD-M service composition model with two concepts: {\sc Rule} and {\sc A-policy} (see blue elements in the $\pi$-SCM meta-model in Figure \ref{fig:e-scomposition-metamodel}).
We model non-functional constraints by using the notion of {\em A-policy}~\cite{Espinosa-Oviedo2011a,CIC:eovszmc09c}.
An {\em A-policy} is formed by attributes and rules. 
Intuitively, the conditions of each rule will be checked.
In case of no compliance, the actions defined by the rule will be performed.
The {\sc Rule} element represents an event - condition - action rule where the {\sc Event} part represents the moment in which a constraint  will be evaluated.
An {\em A-policy} defines variables and operations that can be shared by the rules and that can be used for expressing their Event and Condition parts. 
\end{itemizedTrivlist}

\begin{figure}[t]%[htpb]
\centering
\includegraphics[width=0.95\textwidth]{figs/e-composition-model}

{\color{red}\LARGE PLACIDO: Please change the names of the boxes in accordance to the explanation --Martin}

\caption{Service composition model for the ``To publish music'' business service.}
\label{fig:servicecompositionmodel}
\end{figure}

\begin{example}[To Publish Music]
To illustrate the use of the $\pi$-SCM meta-model, we define a model for the ``To Publish Music'' scenario (Figure \ref{fig:servicecompositionmodel}). 
In this model, there are three external business collaborators ({\em Spotify, Twitter} and {\em Facebook} \footnote{We use {\em italics} to refer to concrete values of the classes of a model that are derived from the classes of a meta-model.}). 
The model also shows the business process of the application that consists of three service activities: {\em Listen Music}, {\em Public Music} and {\em Confirmation}. 
Note that  the activity {\em Publish Music} calls the actions of two service collaborators namely {\em Facebook} and {\em Twitter}.
Both {\em Facebook} and {\em Twitter} services require authentication protocols in order to execute methods that will read and update the user space. 
%A call to such services must be part of the authentication protocol required by these services.
In the example we  associate two authentication policies, one for the open authentication protocol, represented by the class {\sf\small OAuthPolicy} at {\em Twitter}, that will be associated to the activity  {\sf\small UpdateTwitter} (see Figure \ref{fig:servicecompositionmodel}). 
In the same way, the {\em Facebook} class {\sf\small HTTPAuthPolicy}, for the http authentication protocol will be associated to the activity {\sf\small UpdateFacebook}.

{\sf\small OAuthPolicy} will implement the open authentication protocol.
The {\em A-policy} {\sf\small OAuthPolicy} has a variable {\sf\small Token} that will be used to store the authentication token provided by the service.
This variable is imported through the library {\sf\small OAuthPolicy.Token}. 
The A-policy {\sf\small OAuthPolicy} defines two rules, both can be triggered by events of type {\sf\small ActivityPrepared}: (R$_1$): If no token has been associated to the variable {\sf\small token}, then a token is obtained ; and (R$_2$): if the token has expired, then it is renewed. 
Notice that the code in the actions profits from the imported {\sf\small OAuthPolicy.Token} for transparently obtaining or renewing a token from a third party.

{\sf\small HTTPAuthPolicy} implements the HTTP-Auth protocol. 
The A-policy imports an http protocol library and it has two variables {\sf\small username} and {\sf\small password}.  
The event of type {\sf\small ActivityPrepared} is the triggering event of the rule {\sf\small R$_1$}. 
On the notification of an event of that type, a credential is obtained using the username and password. 
\hfill\openbox
\end{example}

{\color{magenta} Aqui\dots  }

Thanks to rules and policies  it is possible to model and associate non-functional properties to services' compositions  and then generate the code. For example, the atomic integration of information retrieved from different social network services, automatic generation of an integrated view of the operations executed in different social networks or for providing security in the communication channel when the payment service is called.

Back to the  definition process of a SIS, once the {\em A-policy} based services' composition model has been defined, then it can be transformed into a model (i.e., $\pi$-PEWS model) that can support then executable code generation. The following Section describes the $\pi$-PEWS  meta-model that supports this representation. 


%..--..--..--..--..--..--..--..--..--..--..--..--..--..--..--..--..--..--..--..--..--..--..--..--..--..--..--..--..--..--..--..--..--..--..--..--..--..--
\subsubsection{$\pi$-{\sc Pews}  meta-model}\label{sec:pewsmetamodel}
%..--..--..--..--..--..--..--..--..--..--..--..--..--..--..--..--..--..--..--..--..--..--..--..--..--..--..--..--..--..--..--..--..--..--..--..--..--..--
The idea of the $\pi$-{\sc Pews} meta-model is based on the services' composition approach provided by the language PEWS\cite{BaAM06,Placido2010LTPD} (\textit{Path Expressions for Web Services}), a programming language that lets the service designer  combine the methods or subprograms that
implement each operation of a service, in order to achieve the desired application logic. Figure \ref{fig:metamodel} presents the $\pi$-{\sc Pews} meta-model
consisting of  classes representing:
\begin{itemize}
\item A services' composition: {\sc Namespace} representing the interface exported by a service, {\sc Operation} that represents a call to a service method, {\sc CompositeOperation}, and  {\sc Operator} for representing a services' composition and {\sc Path} representing a services' composition.
A {\sc Path} can be an {\sc Operation} or a {\sc Compound Operation}
denoted by an identifier. A {\sc Compound Operation} is defined using an  {\sc Operator}  that can be represent  sequential ($\ . \ $) and parallel ($\ \| \ $) composition of services,
 choice ($\ + \ $) among services,
the sequential ($*$) and parallel ($\{\dots\}$) repetition of an operation or the conditional execution of an operation ($[C]S$).

\item {\em A-Policies} that can be associated to a services' composition:  {\sc A-Policy}, {\sc Rule}, {\sc Event}, {\sc Condition}, {\sc Action}, {\sc State}, and {\sc Scope}.
\end{itemize}
%
\begin{figure}
\centering
\includegraphics[width=0.80\textwidth]{figs/PEWSMetamodel}
\caption{$\pi$-{\sc Pews} Metamodel}
\label{fig:metamodel}
\end{figure}

As shown in the diagram an {\sc A-Policy} is applied to a {\sc Scope} that can be either an {\sc Operation} (e.g., an authentication protocol associated to a method exported by a service),  an {\sc Operator} (e.g., a temporal constraint associated to a sequence of operators, the authorized delay between reading a song title in Spotify and updating the walls must be less then 30 seconds), and a {\sc Path} (e.g., executing the walls' update under a strict atomicity protocol -- all or noting).  It groups a set of ECA rules, each rule having a classic semantics, i.e, {\em when an event of type E occurs if  condition C is verified then execute the action A}.  Thus, an {\em A-policy} represents a set of reactions to be possibly executed if one or several triggering events of its rules are notified.
\begin{itemize}
\item The class {\sc Scope} represents any element of a services' composition (i.e., operation, operator, path).
\item The class {\sc A-Policy} represents a recovery strategy implemented by ECA rules of the form {\sc Event} - {\sc Condition} - {\sc Action}. A {\em A-policy} has variables that represent the view of the execution state of its associated scope, that is required for executing the rules. The value of a variable is represented using the type {\sc Variable}. The class {\sc A-Policy} is specialized for defining specific constraints, for instance authentication {\em A-policies}.
\end{itemize}

%An authentication {\em A-policy} represents the situation where an invocation in
%an activity occurs until its sender and/or its recipient have been
%identified. Typically, authentication A-Policies ensure that the invocation of the activity will be done within an authentication protocol.
%

Given a $\pi$-SCM model of a specific services' based application (expressed according to the $\pi$-SCM meta-model), it is possible to generate its corresponding $\pi$-{\sc Pews} model thanks to transformation rules. The following Section describes the transformation rules between the $\pi$-SCM and $\pi$-{\sc Pews} meta-models of our method.





\subsection{$\pi$-SOD-M transformations}\label{sec:pisodmtransformations}
%3.2. pi-sod-M transformations (con la tablas del caise)

Figure \ref{fig:transformations} shows the transformation principle between the elements of the $\pi$-SCM meta-model used for representing the services' composition into the elements of the $\pi$-{\sc Pews} meta-model. There are two groups of rules: those that transform services' composition elements of the $\pi$-SCM to $\pi$-{\sc Pews} meta-models elements; and those that transform rules grouped by policies into {\em A-policy} types.

% _ . _ . _ . _ . _ . _ . _ . _ . _ . _ . _ . _ . _ . _ . _ . _ . _ . _ .
%\noindent

%{\bf\em 
\subsection{Transformation of the services' composition elements of the $\pi$-SCM to the $\pi$-{\sc Pews} elements}
% _ . _ . _ . _ . _ . _ . _ . _ . _ . _ . _ . _ . _ . _ . _ . _ . _ . _ .
A named action of the $\pi$-SCM represented by  {\sc\em Action} and {\sc\em Action:name} is transformed to a  named class {\sc Operation} with a corresponding attribute name {\sc Operation:name}. A  named service activity represented by the elements {\sc\em ServiceActivity}  and  {\sc\em ServiceActivity:name} of the $\pi$-SCM, are  transformed into a named operation of the $\pi$-{\sc Pews} represented by the elements  {\sc CompositeOperation} and {\sc CompositeOperation:name}. When more than one action is called, according to the following  composition patterns expressed using the operators {\sc\em merge, decision, fork and join} in the $\pi$-SCM the corresponding transformations, according to the PEWS operators presented above, are (see details in Figure \ref{fig:transformations}):
\begin{itemize}
\item   $op_1 . op_2$ if no {\sc\em ControlNode} is specified
\item ($op_1 \parallel op_2) . op_3$ if control nodes of type {\sc\em fork, join} are combined
 \item ($op_1 + op_2) . op_3$ if control nodes of type {\sc\em decision, merge} are combined
\end{itemize}

In the scenario "To Publish Music" the service activity {\sf PublishMusic} of the $\pi$-SC model specifies  calls to two {\sf Activitie}s of type {\em UpdateMusic}, respectively concerning the {\sf Business Service}s {\em Facebook} and {\em Twitter}. Given that no {\sf ConstrolNode} is specified by the $\pi$-SC model, the corresponding transformation is the expression that defines a {\sf Composite Operation} named {\em PublishSong} of the $\pi$-{\sc Pews} model of the form ({\sf PublishFacebook} $\parallel$ {\sf PublishTwitter}).
\begin{figure}
\centering{
%\includegraphics[width=0.80\textwidth]{figs/PI-SC-PI-P}}
\includegraphics[width=0.96\textwidth]{figs/Mapping-1}}
\caption{ $\pi$-SCM to $\pi$-{\sc Pews} transformation}
\label{fig:transformations}
\end{figure}

% _ . _ . _ . _ . _ . _ . _ . _ . _ . _ . _ . _ . _ . _ . _ . _ . _ . _ .
%\noindent

%{\bf\em 
\subsection{Transformation of rules grouped by A-policies   in the $\pi$-SCM to A-Policies of  $\pi$-{\sc Pews}}
% _ . _ . _ . _ . _ . _ . _ . _ . _ . _ . _ . _ . _ . _ . _ . _ . _ . _ .
The {\em A-policies} defined for the elements of the $\pi$-SCM are transformed into {\sc A-Policy} classes, named according to the names expressed in the source model. The transformation of the rules expressed in the $\pi$-SCM is guided by the event types associated to these rules.   The variables associated to an {\em A-policy} expressed in the $\pi$-SCM as {\sc\em $<$Variable:name, Variable:type$>$} are transformed into elements of type {\sc Variable} with attributes {\sc name} and {\sc type} directly specified from the elements {\sc\em  Variable:name} and {\sc\em Variable:type} of the $\pi$-SCM model.

As shown in Figure \ref{fig:transformations}, for an event of type {\sc\em Pre} the corresponding transformed rule is of type {\sc Precondition}; for an event of type {\sc\em Post} the corresponding transformed rule is of type {\sc Postcondition}; finally, for an event of type {\sc\em TimeRestriction} the corresponding transformed rule is of type {\sc Time}. 
The condition expression of a rule in the $\pi$-SCM ({\sc\em Rule:condition}) is transformed into a class {\sc\em Condition:expression} where the attributes of the expression are transformed into elements of type {\sc Attribute}.

%The attribute event of a rule  ({\sc\em Rule:event}) in the $\pi$-SCM is transformed into an {\sc Event Type} according to the rule type. 

%As shown in Figure \ref{fig:transformations}, the event type for a rule of type (i) {\sc Precondition} is {\sc ActivityPrepared}; (ii) {\sc Postcondition} is {\sc TermActivity}; (iii) {\sc TimeRestriction} is {\sc Temporal}. The {\sc\em Rule:Action} of a rule in the $\pi$-SCM is transformed into an {\sc Action:type}.

%
%Figure \ref{fig:p-scim} shows the  $\pi$-{\sc Pews} model for our example.
In the scenario "To Publish Music" the {\sf Policies} {\em OAuthPolicy} and {\em HTTPAuthPolicy} of the $\pi$-SCM model are transformed into {\em A-policies} of type {\sf Precondition} of the $\pi$-{\sc Pews} model of the scenario. Thus in both cases the events are of type {\sf ActivityPrepared}. These policies, as stated in the $\pi$-SCM model, are associated to {\sf Activities}. In the corresponding transformation they are associated to {\sf Operation}s {\em PublishFacebook} and {\em PublishTwitter}.
%\begin{figure}[htpb]
%\centering{
%\includegraphics[width=0.78\textwidth]{figs/modeloPEWS}}
%\caption{$\pi$-{\sc Pews} generated model fo the "To Publish Music" application}
%\label{fig:p-scim}
%\end{figure}

%Figure \ref{fig:pewsexpression} shows the correspondence between the model and the statements that implement it, with a schematic representation of the business process.
%\begin{figure}
%\centering{
%\includegraphics[width=0.85\textwidth]{figs/pews-expression}}
%\caption{Pews program implementing the "To Publish Music" application}
%\label{fig:pewsexpression}
%\end{figure}

%*********************************************************************************************************

\section{Applying $\pi$-SOD-M: a case study}
4.1 Presentar los modelos para el caso y explicar las transformaciones
4.2 Discussion. (o Lecciones aprendidas del caso)
\input{modelingCrimes}



%*********************************************************************************************************
\section{Conclusions and future work}\label{sec:conclusions}
We presented \pisodm, an model-driven method for designing and developing reliable service-based applications.
\pisodm extends a previously defined method (called SOD-M) to include Non-Functional Requirements.
These requirements are taken into account from the early stages of the software development process.
Non-functional constraints are related to business rules associated to the behavior of the application and, in the case of service-based applications, they are also concerned with constraints imposed by the services.

Our method includes two CIM-level models, three PIM-level models and one PSM-level model.
We implemented the meta-models on the Eclipse platform and we validated the approach by using an industrially inspired use case.

Our case study demonstrates the applicability of \pisodm.
This case study was developed together with our industrial partner, GCP Global.
The Company is using \pisodm for the development of their product.
The case study presented here is a simplified version of their application. 
%*********************************************************************************************************


\appendix

%*********************************************************************************************************
\section{Implementation issues}\label{sec:implementation}
\input{4-implementation}
%*********************************************************************************************************



%% The Appendices part is started with the command \appendix;
%% appendix sections are then done as normal sections
%% \appendix

%% \section{}
%% \label{}

%% References
%%
%% Following citation commands can be used in the body text:
%% Usage of \cite is as follows:
%%   \cite{key}          ==>>  [#]
%%   \cite[chap. 2]{key} ==>>  [#, chap. 2]
%%   \citet{key}         ==>>  Author [#]

%% References with bibTeX database:

\bibliographystyle{plain}
\bibliography{biblio}

%% Authors are advised to submit their bibtex database files. They are
%% requested to list a bibtex style file in the manuscript if they do
%% not want to use model1a-num-names.bst.

%% References without bibTeX database:

% \begin{thebibliography}{00}

%% \bibitem must have the following form:
%%   \bibitem{key}...
%%

% \bibitem{}

% \end{thebibliography}

\end{document}

%%
%% End of file `elsarticle-template-1a-num.tex'.
