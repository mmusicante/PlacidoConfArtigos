
This section  describes the $\pi$-SOD-M development environment that implements the generation of {\em A-policies}' based services' compositions. For a given services' based application, the process  consists in generating the  code starting from a $\pi$-SCM modeling an application. Note that the services' composition model is not modeled from scratch, but it is the result of a general process defined by the $\pi$-SOD-M method in which a set of models are built following a service oriented approach \cite{decastro1}.

%We used the Eclipse Modeling Framework (EMF) to implement the whole model transformation process \footnote {The EMF project is a modeling framework and code generation facility for building tools and other applications based on a structured data model.}. From a model specification described in XMI, EMF provides tools and runtime support to produce a set of Java classes for the model, along with a set of adapter classes that enable viewing and command-based editing of the model, and a basic editor.
%In order to automate the transformation we specified  transformation rules using the ATL model transformation language Finally, in order to generate code we  used Acceleo \footnote{http://www.acceleo.org/pages/home/en}.

%%..--..--..--..--..--..--..--..--..--..--..--..--..--..--..--..--..--..--..--..--..--..--..--..--..--..--..--..--..--..--..--..--..--..--..--..--..--..--
%\subsection{$\pi$-SOD-M Development Environment}
%%..--..--..--..--..--..--..--..--..--..--..--..--..--..--..--..--..--..--..--..--..--..--..--..--..--..--..--..--..--..--..--..--..--..--..--..--..--..--

Figure \ref{fig:policymanager} depicts a general architecture of the $\pi$-SOD-M Development Environment showing the set of plug-ins  developed in order to implement it. The environment implements the abstract architecture shown in Figure \ref{fig:sodm}. Thus, it consists of plug-ins implementing the $\pi$-SCM and $\pi$-{\sc Pews} meta-models used for defining models specifying services' compositions and their associated policies; and ATL rules for transforming  PSM models (model to model transformation) and finally generating code (model to text transformation).
\begin{figure}[htpb]
	\begin{center}
		\includegraphics[width=0.60\textwidth]{figs/architecture}
	\end{center}
		\caption{$\pi$-SOD-M Development Environment}
   \label{fig:policymanager}
\end{figure}
\begin{itemize}
\item 	We  used the Eclipse Modeling Framework (EMF) \footnote {The EMF project is a modeling framework and code generation facility for building tools and other applications based on a structured data model.}   for implementing the meta-models  $\pi$- SCM and $\pi$-{\sc Pews}. Then, starting form these meta-models, we  developed the models' plug-ins needed to support the graphical representation of the $\pi$- SCM and $\pi$-{\sc Pews} models ($\pi$-ServiceCompostion Model and $\pi$-PEWS Model plug-ins).

\item	 We used  ATL \footnote{http://eclipse.org/atl/. An ATL program is basically a set of rules that define how source model elements are matched and navigated to create and initialize the elements of the target models.}
for  developing the mapping plug-in implementing the  mappings between models ($\pi$-ServiceComposition2$\pi$-PEWS Plug-in).

\item 	We  used Acceleo \footnote{http://www.acceleo.org/pages/home/en} for implementing  the code generation plug-in. We coded the pews.mt program  that implements the model to text transformation for generating executable code. It takes as input a $\pi$-PEWS model implementing a specific services' composition and it generates the code to be executed by the 
{\em A-policy} based services' composition execution environment. 

%\item Finally, we created a chain execution  to execute the model to text transformation.
\end{itemize}

%
As  shown in Figure \ref{fig:policymanager}, once an instance of a PEWS code is obtained starting form a particular $\pi$-services' composition model it can be executed over {\em A-policy} based services' composition execution environment  consisting of a composition engine and a {\em A-policy} manager.  The  {\em A-policy} manager  consists of three main components Manager, for scheduling the execution of rules, C-Evaluator and A-Executor respectively for evaluating rules' conditions and executing their actions. The {\em A-policy} Manager interacts with a composition engine thanks to a  message communication layer (MOM).


The composition engine manages the life cycle of the composition. Once a composition instance is activated, the engine schedules the composition activities according to the composition control flow.
Each activity is seen as the process where the service method call is executed.
The execution of an activity has four states: prepared, started, terminated, and failure.
The execution of the control flow (sequence, and/or split and join) can also be prepared, started, terminated and raise a failure.

At execution time, the evaluation of policies done by the {\em A-policy} manager must be synchronized with the execution of the services' composition (i.e., the execution of an activity or a control flow).  Policies associated to a scope are activated when the execution of its scope starts. A {\em A-policy} will have to be executed only if one or several of its rules is triggered. If several rules are triggered the {\em A-policy} manager first builds an execution plan that specifies the order in which such rules will be executed according to the strategies defined in the following section. 
%Once rules have been executed, the {\em A-policy} finishes its execution and returns to a sleeping state.
If rules belonging to several policies are triggered then policies are also ordered according to an execution plan. The execution of policies is out of the scope of this paper, the interested reader can refer to \cite{Espinosa-Oviedo2011a} for further details.
%The order of policies has implications on the global order of the rules to be executed.

