This paper presented \pisodm for designing and developing reliable service-based applications. 
\pisodm is an MDA methodology that extends a previously defined method (called SOD-M) to include Non-Functional Requirements.
These requirements are taken into account from the initial stages of the software development process.
Non-functional constraints are related to business rules associated to the behavior of the application and, in the case of service-based applications, they are also concerned with constraints imposed by the services. 

Our methodology includes two CIM-level models, three PIM-level models and one PSM-level model. 
We implemented the meta-models on the Eclipse platform and we validated the approach by using an industrially inspired use case.

Our case study demonstrates the applicability of \pisodm.
The case study was developed together with our industrial partner, GCP Global.
The Company is using \pisodm for the development of their product.
The case study presented here is a simplified version of their application.