A non-functional property (NFP) specifies criteria about the behavior of a system. These criteria are related to the conditions in which they are executed and to its performance. Non-functional properties are also referred as “constraints”, “quality attributes”, “quality goals”, “quality of service requirements” and “non-behavioral requirements” \cite{StGr05}. In the case of service-based applications, non-functional requirements concern the application itself as well as its component services. 
The majority of the work on NFPs uses a product-oriented approach, which is concerned with measuring how much a software system is in accordance with the set of NFPs that it should satisfy \cite{BBKL78,FePf96,KiDa96,Lyu96,MuIO90}. 


These  approaches address the problem from four perspectives: (i) NFP expressed as quality of service properties (QoS attributes) that can be associated to service compositions; (ii) service matching for associating certain non-functional properties by composing specific services;  (iii) programming NFP as protocols that are asserted to service composition code; and, (iv) methodologies that provide tools for building service compositions with associated NFP. 

\subsection{Classifying Non-functional properties}
We conducted a systematic review \cite{placidoPhDThesis2012} in order to analyze approaches that consider non-functional properties for developing service-oriented applications. As a result this analysis proposes a classification of NFP as a result of a study concerning software methodologies for the construction of service-oriented systems. The classification is organized in three layers representing: application modeling, services composition and services. The service composition layer serves as an integration layer between the services layer that exports methods and has associated constraints and characteristics; and the application layer that expresses requirements (NFP). At the application layer NFP can refer to business rules (e.g., only the user can publish data on her/his wall) and values for example, the email address is a string containing an “@” and a “.”. A value NFR expresses constraints about the way data and functions can be ac-cessed and executed. 
Business NFP at the service layer concerns properties that are associated to services and defines how to call their exported operations (business properties).  Value constraints concern more on the conditions in which services can be used. For example, accessing to a function within an authentication protocol.
Finally, at the service composition layer gives an abstract view of the kind of properties exported by services that can be combined for providing NFP for a composition.  For example, confidentiality, authentication, privacy and access control can provide security at the service composition layer. 


\cite{ScCM10} proposes concepts and requirements for characterizing and analyzing exist-ing approaches addressing NFP for service coordinations. Therefore they propose:
\begin{itemize}
\item	A meta-model that introduces for characterizing NFP according to the entity to which they are associated: attribute, concern, action, and activity. 
\item	Six requirements for studying NFP definition approaches for service coordinations: NFP specifications, NFP actions specification, Web service subjects specification, non functional attributes execution order specification, composite Web service subjects specification, stateful non functional constraint specification.
\item	Seven requirements for studying NFP enforcement approaches: separation of con-cerns, transparent integration of functional and NFP, quantification, superimposi-tion, integration of NFP with distributed Web service, programming language in-dependence, Web service composition support.
\end{itemize}

NFP concepts are close to those defined by \cite{Souz12} but they are not organized into layers.  They correspond essentially to the service and service composition layers. The enforcement requirements describe the way NFP are weaved to service coordinations. Existing works respect all or a subset of these requirements. The chosen requirements have an impact in the way NFP are enforced at execution time.


%  Standards devoted for expressing non-functional properties for service-oriented applications are also analyzed in this section which needs to be addresses in the code generation phase. Finally exiting methodology for the development of service-oriented applications including non-functional properties are here depicted and compared with our approach.
%As a result of our study of related works, in this section we present a categorization of main concepts related the treatment if non-functional requirement in the field of service-oriented development.

\subsection{Specifying Non-functional properties}

Extensions to UML are proposed to express NFPs. NFPs can be integrated into the Class, Sequence, and Collaboration Diagrams \cite{CyPr04}. The performance engineering [DiSD02] annotates quantitative performance constraints to UML diagrams such as actor-use case communication, message in the sequence diagram, and state in the state machine diagram. The UML Profile for Modeling Quality of Service and Fault Tolerance Characteris-tics and Mechanisms \cite{Omg04} describes a set of QoS characteristics such as throughput, latency, integrity, dependability, among others. The characteristics can also be parameter-ized with for example the type of keys or types of encryption algorithms in case of the secu-rity characteristics. For all UML profiles based approaches it is generally possible to apply the stereotypes and its tagged values to arbitrary UML meta-classes, e.g. Class or Opera-tion. The ordering of stereotypes cannot be expressed but composition- related require-ments can theoretically be realized through behavioral diagrams like activity diagrams or state diagrams.

\cite{OrHe07} incorporate extra-functional properties into UML models by the use of UML stereotypes. For instance, there are stereotypes for encryption, decryption and logging which can be added to classes or operations that are representing services.
Design patterns help in documenting and communicating proven design solutions to recurring problems. Design patterns not only describe how to solve design problems but also document the designers’ intentions - why the solution is chosen over others and what tradeoffs are made. 

\cite{GrYu01} observe that non-functional properties  (NFPs) are pervasive in the descriptions of design patterns and propose their systematic treatment in descrip-tions of patterns and when applying patterns during design. 

Business rules provide means for expressing, managing and updating pieces of busi-ness domain knowledge independent of the rest of the application. According to the Busi-ness Rules Group \cite{Hay00}, a business rule is a statement that defines or constrains some aspect of the business. It is intended to assert business structure or to control the behavior of the business \cite{Hay00}. Business rules are usually expressed either as constraints or in the form if conditions then action. The conditions are also called rule premises. The business rule approach encompasses a collection of terms (definitions), facts (connection between terms) and rules (computation, constraints and conditional logic) \cite{Hall01}. Terms and Facts are statements that contain sensible business relevant observations, whereas rules are statements used to discover new information or guide decision-making. A business rule system is a system in which the rules are separated logically and perhaps physically from the other parts \cite{ChMe04}. \cite{ChMe04} investigate a hybrid approach, which combines business processes as known e.g., from BPEL with business rules \cite{Hay00}.

\cite{ChLa09} proposed a generative framework to address NFPs for web service composi-tions. As part of the framework they separate the orchestration meta-model from other non-functional meta-models. They propose a non-functional meta-model for security where concepts exist for requirements and how these requirements are ensured. The NFPs are then added to the composition model by adding annotations.
KAOS \cite{DaLF93,Lams01} is a goal-oriented framework for addressing functional requirements with a formal temporal logic that requirements engineer can use to precisely specify NFPs, such as performance, as part of attribute specification of functional requirements. 

The AO4BPEL \cite{CSHM06} deployment descriptor (DD) is used for specifying non-functional requirements for BPEL processes. First the service (one of logging, security, transactions and reliable messaging) is selected. Then the class (e.g. authentication or con-fidentiality for security) can be specified. The type (e.g. encryption or decryption for confi-dentiality) is then the concrete action that must be executed in order to enforce the re-quirement. Then a selector can be defined such as for a certain BPEL activity. Each selector can be mapped to a requirement. 

The W3C proposes a set of languages and protocols for specifying the NFP of a ser-vices’ coordination. A classification of NFP protocols, provided by the WS-* family protocols proposed by the World Wide Web Consortium (W3C). 
%These protocols suppose an underlying communication protocol. 
%
%The following sections describe representative examples of the WS-* protocol-based approaches.
%
 Standards in services' composition implement functional, non-functional constraints and communication aspects by combining different languages and protocols. WSDL and SOAP are languages used respectively for describing services' interfaces and message exchange protocols for calling methods exported by such services. For adding a transactional behaviour to a services' composition it is necessary to implement WS-Coordination, WS-Transaction, WS-BussinessActivity and WS-AtomicTransaction. The selection of the adequate protocols for adding a specific non-functional constraints to a services' composition (e.g., security, transactional behaviour and adaptability) is responsibility of a programmer. As a consequence, the development of an application based on a services' composition is a complex and a time-consuming process.
 % This is opposed to the philosophy of services that aims at facilitating the integration of distributed applications. 
 Other works, like \cite{Fauvet05} introduce a model for transactional services composition based on an advanced transactional model.\cite{BhiriGP05} proposes an approach that consists of a set of algorithms and rules to assist designers to compose transactional services. In \cite{Vidyasankar:2004} the model introduced in \cite{SchuldtABS02} is extended to web services for addressing atomicity.

As WS-* and similar approaches, our work enables the specification and programming of crosscutting aspects (i.e., atomicity, security, exception handling, persistence).
In contrast to these approaches, our work specifies policies for a services' composition in an orthogonal way. Besides, these approaches suppose that non-functional properties are implemented according a the knowledge that a programmer has of a specific application requirements but they are not derived in a methodological way, leading to ad-hoc solutions that can be difficult to reuse. In our approach, once defined the policies for a given application they can be reused and/or specialized for another one with the same requirements or that uses services that impose the same constraints.

\subsection{Non-functional properties in service compositions}

Several approaches adding non-functional properties to service compositions extend the underlying composition model. Non-functional properties can concern for example securty, atomicity, availability, response time.
\cite{CholletL09,SchmelingCM11}  associate non-functional  properties  to functional activities  modeled by a web service composition model. In \cite{chmelingCM11} 
non-functional propoerties are  are modeled independently and then associated to   a specific service composition represented in a BPMN diagram. 
In \cite{OvaskaEHPA10} non-functional properties concern  quality management and they are specified at design time. Quality requirements  are then associated to the components of an architecture that implements a service composition. 


Other approaches specify architectures or frameworks to manage and validate QoS attributes in service composition processes. 
\cite{XiaoCZBOLH08} proposes a framework to verify NFP compliance in composed services. Tasks in a BPEL process can be annotated with non-functional attributes (such as time, cost, resources, etc.) which are validated at design time. 
\cite{SchmelingCM11} uses a formal method  to choose best suited services instances according to a service composition pattern that includes some QoS attributes. 
 \cite{Babamir2010} proposes a framework for evaluating quality attributes of web services. By executing an algorithm, the framework allow to choose the best web services for any task in a process and also candidate web service to replace original ones in cases it cannot perform at runtime. \cite{DAmbrogio06,Liu20121080}  extend  WSDL and BPEL   for includin  QoS attributes that can be evaluated at runtime.
 \cite{Karunamurthy2012787} proposes an architecture in which web services are described with non-functional characteristics that are validated in the service composition process.

Other approaches \cite{JeongCL09,Kamalabad2012,MohantyRP10} use matching techniques for choosing services that can cope to non-functional properties required by a given application.
\cite{AgarwalLS09} defines a policy language to specificy constraints on functional  properties of Web services for facilitating automated discovery and selection of services. 
\cite{Yeom2006} define a quality driven selection of web services providing a QoS model.



%\cite{} works also on the selection of web service for use considering non-functional properties. The paper present web services quality prediction models, which take non-functional properties into account. 


Our proposal in contrast, focuses on the design of service-oriented application with non-functional properties modelled  for services and service compositions and then ensured at execution thanks to appropriate environments.

%We propose to model non-functional properties  represented in our work as constraints and policies but more general quality model such as the proposed by \cite{Goeb2011,Klass2009} could be taken in consideration in further works.

\subsection{Methodologies for Service-oriented applications development  including non-func\-tion\-al properties}

 \cite{Ramollari_asurvey,PapazoglouH06,FeuerlichtM05,soma,Arsanjani:2008} propose methodologies for building service-based applications that  mainly focus on modeling functional aspectos of service-based business processes. SOD-M \cite{decastro1} defines a model-driven approach for development service-based applications. 
Few service-oriented development methodologies    explicitly address  non functional properties modeling \cite{Tran2012531,GutierrezRF10}. \cite{GutierrezRF10} presents PWSSec process, a methodology that provides  activities, tasks, tools, security artifacts and organizational structures necessary to design a secure WS-based solution. 



Unlike methodologies and approaches providing best practices presented above, the main contribution of our proposal is that, integrated to a method that allows to model service-based applications starting from a high-level business modelling, and defining meta-models at different abstraction levels (CIM, PIM and PSM), it enables the design and development of services-based applications that can be reused and that are reliable. The definition of a model-driven approach that allows to model non-functional properties in the early stages of the software development process, and the definition of model transformations to go down in the development process adding more a more implementation details at each modelling stage is a key benefits of our proposal.

%\textit{Design by Contract} \cite{HL05TACoS} is an approach for specifying web services and verifying them through runtime checkers before they are deployed. A contract adds behavioral information to a service specification, that is, it specifies the conditions in which methods exported by a service can be called. Contracts are expressed using the language \textit{jmlrac} \cite{LeavensCCRC02}. The \textit{Contract Definition Language} (CDL) \cite{cdl2006} is a XML-based description language, for defining contracts for services. There are an associated architecture framework, design standards and a methodology, for developing applications using services.  A services' based application  specification is generated after several \cite{AbrialLNSS91} B-machines refinements that describe the services and their compositions.

%\cite{PapazoglouH06} proposes a methodology based on a SOA extension. This work defines a service oriented business process development methodology with phases for business process development. The whole life-cycle is based on six phases: planning, analysis and design, construction and testing, provisioning, deployment, and execution and monitoring.
%IBM proposes a methodology for the development of SOA solutions, called SOMA \cite{soma}. SOMA defines a life-cycle with seven phases: business modeling and transformation, solution management, identification,
%specification, realization, implementation and deployment monitoring and management.

%\cite{sommerville08} describes some key points for building services' based applications based on business process models that define the activities and information exchanged in a business processes. Activities in business process can be performed by services so that the model of business process
%represents a composition of services. It classifies services   as public utilities, business services or
%composition services. Software development that uses services involves creating programs for composing and configuring services to create new composite services. The service engineering process involves identifying services candidates, service interface and implementation definition, testing and deployment of each service.





%Besides the notion of model at each level of abstraction, MDA propose the use of \textit{model transformations} within and between levels. Intra-level transformations are used to provide a unified representation of concepts of a given level. Inter-level transformation implement a refinement process between levels. Transformations may be automatic or semi-automatic.


%The methodology for supporting the construction of service-oriented applications, taking into account both functional and non-functional requirements presented in this work extends the \textit{Service Oriented Development Method} (SOD-M) proposed by De Castro et al.\cite{decastro1}.
