
The methodology for supporting the construction of service-oriented applications, taking into account both functional and non-functional requirements presented in this work extends the \textit{Service Oriented Development Method} (SOD-M) proposed by De Castro et al.\cite{decastro1}.

SOD-M define a model-driven approach for development service-based applications. The model-driven method provided by SOD-M is based on the Model Driven Architecture (MDA) proposed by the OMG \cite{miller}. MDA defines a top-down approach for the development of software systems that uses the notion of \textit{model} to specify a software system at different levels of abstraction (called \textit{viewpoints}):

\begin{trivlist}
\item \textbf{Computation Independent Models (CIM):} This level focusses on the
environment of the system, as well as on its business and requirement specifications.
This viewpoint represents the software system at its highest level of abstraction.
At this moment of the development, the structure and system processing details are still unknown or undetermined.

\item \textbf{Platform Independent Models (PIM):} This level focusses on the system functionality, hiding the details of any particular platform.
The specification defines those parts of the system that do not change from one platform to another.

\item \textbf{Platform Specific Models (PSM):} This level focusses on the functionality, in the context of a particular implementation platform.
Models at this level combine the platform-independent view with the specific aspects of the platform to implement the system.
\end{trivlist}

Besides the notion of model at each level of abstraction, MDA propose the use of \textit{model transformations} within and between levels. Intra-level transformations are used to provide a unified representation of concepts of a given level. Inter-level transformation implement a refinement process between levels. Transformations may be automatic or semi-automatic.

According to MDA guidelines, SOD-M meta-models are organized into three levels: CIM (\textit{Computational Independent Models}), PIM (\textit{Platform Independent Models}) and PSM (\textit{Platform Specific Models}).

Two models are defined at the CIM level: \textit{value model}
and \textit{BPMN model}.
\textsc{Explain these two here --Martin.}

Models at the PIM level describe the structure of the application flow,
while, the PSM level provides transformations towards more specific platforms.
The PIM-level models proposed by SOD-M are:
\begin{trivlist}
\item \textit{use case}:
\item \textit{extended use case}:
\item \textit{service process}:
\item \textit{service composition}.
\end{trivlist}

SOD-M defines three meta-models at the PSM level: \textit{web service interface}, \textit{extended composition service} and \textit{business logic}.
These three levels have no support for describing non-functional requirements.

The SOD-M approach includes transformations between models:
\textit{CIM-to-PIM, PIM-to-PIM} and \textit{PIM-to-PSM} transformations. Given
an abstract model at the CIM level, it is possible to apply transformations for
generating a model of the PSM level. In this context, it is necessary to
follow the process activities described by the methodology.

SOD-M considers two points of view:
\textit{(i)} \textit{business}, focusing on the characteristics and requirements
of the organization, and \textit{(ii)} \textit{system requirements}, focusing on
features and processes to be implemented in order application requirements. In
this way, SOD-M aims to simplify the design of service-oriented applications, as
well as its implementation using current technologies. 