
The methodology for supporting the construction of service-oriented applications, taking into account both functional and non-functional requirements presented in this work extends the \textit{Service Oriented Development Method} (SOD-M) proposed by De Castro et al.\cite{decastro1}.

SOD-M define a model-driven approach for development service-based applications. The model-driven method provided by SOD-M is based on the Model Driven Architecture (MDA) proposed by the OMG \cite{miller}. MDA defines a top-down approach for the development of software systems that uses the notion of \textit{model} to specify a software system at different levels of abstraction (called \textit{viewpoints}):

\begin{trivlist}
\item \textbf{Computation Independent Models (CIM):} This level focusses on the
environment of the system, as well as on its business and requirement specifications.
This viewpoint represents the software system at its highest level of abstraction.
At this moment of the development, the structure and system processing details are still unknown or undetermined.

\item \textbf{Platform Independent Models (PIM):} This level focusses on the system functionality, hiding the details of any particular platform.
The specification defines those parts of the system that do not change from one platform to another.

\item \textbf{Platform Specific Models (PSM):} This level focusses on the functionality, in the context of a particular implementation platform.
Models at this level combine the platform-independent view with the specific aspects of the platform to implement the system.
\end{trivlist}

Besides the notion of model at each level of abstraction, MDA propose the use of \textit{model transformations} within and between levels. Intra-level transformations are used to provide a unified representation of concepts of a given level. Inter-level transformation implement a refinement process between levels. Transformations may be automatic or semi-automatic.

According to MDA guidelines, SOD-M meta-models are organized into three levels: CIM (\textit{Computational Independent Models}), PIM (\textit{Platform Independent Models}) and PSM (\textit{Platform Specific Models}).

Two models are defined at the CIM level: \textit{value model}
and \textit{BPMN model}.
\textsc{Explain these two here --Martin.}

Models at the PIM level describe the structure of the application flow,
while, the PSM level provides transformations towards more specific platforms.
The PIM-level models proposed by SOD-M are:
\begin{trivlist}
\item \textit{use case}:
\item \textit{extended use case}:
\item \textit{service process}:
\item \textit{service composition}.
\end{trivlist}

SOD-M defines three meta-models at the PSM level: \textit{web service interface}, \textit{extended composition service} and \textit{business logic}.
These three levels have no support for describing non-functional requirements.

The SOD-M approach includes transformations between models:
\textit{CIM-to-PIM, PIM-to-PIM} and \textit{PIM-to-PSM} transformations. Given
an abstract model at the CIM level, it is possible to apply transformations for
generating a model of the PSM level. In this context, it is necessary to
follow the process activities described by the methodology.

SOD-M considers two points of view:
\textit{(i)} \textit{business}, focusing on the characteristics and requirements
of the organization, and \textit{(ii)} \textit{system requirements}, focusing on
features and processes to be implemented in order application requirements. In
this way, SOD-M aims to simplify the design of service-oriented applications, as
well as its implementation using current technologies. 

{\LARGE \color{red} Relocate the next subsections or delete them --Genoveva and Martin}



%..--..--..--..--..--..--..--..--..--..--..--..--..--..--..--..--..--..--..--..--..--..--..--..--..--..--..--..--..--..--..--..--..--..--..--..--..--..--
\subsection{Modeling a services' based application}
%..--..--..--..--..--..--..--..--..--..--..--..--..--..--..--..--..--..--..--..--..--..--..--..--..--..--..--..--..--..--..--..--..--..--..--..--..--..--
 Figure \ref{fig:sodm} shows  SOD-M that defines a service oriented approach   providing  a set of guidelines to build services' based information systems (SIS) \cite{decastro1,decastro2}.  Therefore, SOD-M proposes to use services as first-class objects for the whole process of the SIS  development and it  follows a Model Driven Architecture (MDA) \cite{miller}  approach. Extending from the highest level of abstraction of the MDA, SOD-M provides  a conceptual structure to: first, capture the system requirements and specification in high-level abstraction models (computation independent models, CIMÕs); next,  starting from such models build platform independent models (PIMÕs) specifying the system details; next transform such models into platform specific models (PSMÕs) that bundles the specification of the system with the details of the targeted platform; and finally, serialize such model into the working-code that implements the system. 
\begin{figure} [htpb]
\centering
\includegraphics[width=0.65\textwidth]{figs/SODM}
\caption{SOD-M development process}
\label{fig:sodm}
\end{figure} 
As shown in Figure \ref{fig:sodm}, the SOD-M model-driven process begins by building the high-level computational independent models and enables specific models for a service platform to be obtained as a result  \cite{decastro1}. Referring to the "To Publish Music" application, using SOD-M the designer starts defining an E3value model \footnote{The E3 value model is a business model that represents a business case %graphically as a set of value exchanges ($\nabla$$\triangle$) and value activities (rounded boxes) performed by business actors (squared boxes) 
and allows  to understand the environment in which the services' composition will be placed \cite{e3value}.}  at the CIM level and then the corresponding models of the PIM are generated leading to a services' composition model (SCM).
%SOD-M proposes a set of models, :  i) the three different MDA abstraction levels: CIM, PIM and PSM; and ii) SOD-M views: business and information system views. 
%Model-Driven Engineering (MDE) \cite{schmidt} and particularly MDA \footnote{Model Driven Architecture (MDA)  is the particular model-driven proposal defined by the Object Management Group (OMG).} 
%provide
%is an evolving approach to software development that deals with the provision of 
%models, transformations between them and code generators to address software development. 
%One of the main advantages of model-driven approaches is the provision of 
%It also provides a conceptual structure where the models used by business managers and analysts can be traced towards more detailed models used by software developers.  

%Now, consider that besides the services' composition that represents the order in which the services are called for implementing the application "To Publish Music" it is necessary to model  other requirements that represent the (i) conditions imposed by services for being contacted, for example the fact the Facebook and Twitter require authentication protocol in order to call their methods for updating the wall; (ii) the conditions stemming from the business rules of the application logic, for example the fact that the walls in Facebook and Twitter must show the same song title and if this is not possible then none of them is updated. 

%..--..--..--..--..--..--..--..--..--..--..--..--..--..--..--..--..--..--..--..--..--..--..--..--..--..--..--..--..--..--..--..--..--..--..--..--..--..--
\subsection{Modeling non-functional constraints of services' based applications}
%..--..--..--..--..--..--..--..--..--..--..--..--..--..--..--..--..--..--..--..--..--..--..--..--..--..--..--..--..--..--..--..--..--..--..--..--..--..--
Adding non-functional requirements and services constraints in the services' composition is a complex task that implies programming  protocols for instance authentication protocols to call a service in our example, and atomicity (exception handling and recovery) for ensuring a true synchronization of the results produced by the service methods calls.
% song title disseminated in the walls of the user's Facebook and Twitter accounts. 

Service oriented computing promotes ease of information systems' construction thanks, for instance, to services' reuse. Yet, this is not applied to non-functional constraints as the ones described previously, because they do not follow in general the same service oriented principle and because they are often not fully considered in the specification process of existing services' oriented development methods. Rather, they   are either supposed to be ensured by the underlying execution platform, or they are programmed through ad-hoc protocols. Besides,  they are partially or rarely methodologically derived from the application specification, and they are added once the code has been implemented. In consequence, the resulting application does not fully preserve the compliance and reuse expectations provided by the service oriented computing methods.

%..--..--..--..--..--..--..--..--..--..--..--..--..--..--..--..--..--..--..--..--..--..--..--..--..--..--..--..--..--..--..--..--..--..--..--..--..--..--
%\subsection{$\pi$-SOD-M}
%..--..--..--..--..--..--..--..--..--..--..--..--..--..--..--..--..--..--..--..--..--..--..--..--..--..--..--..--..--..--..--..--..--..--..--..--..--..--

%
%Our proposal, called $\pi$-SOD-M, extends the services' composition model of the SOD-M method  with the notion of {\em A-Policy} \cite{Espinosa-Oviedo2011a} for representing services' composition constraints. 
%
Our work extends SOD-M for building applications by modeling the application logic and its associated non-functional constraints and thereby ensuring the generation of reliable services' composition. 
%This method is described in the following sections.
In order to do, our work organizes non-functional constraints into 
three layers representing: application modeling, services composition and services. The service composition layer serves as an integration layer between the services layer that exports methods and has associated constraints and characteristics; and the application layer that expresses requirements.
At the application layer NFP can refer to business rules and values. A value NFR expresses constraints about the way data and functions can be accessed and executed. For example accessing methods under security protocols.

Business NFP at the service layer concerns properties that are associated to services and defines how to call their exported operations (business properties). For example, response time, storage capacity (e.g., Dropbox service provides 5Giga free storage). Value constraints concern more on the conditions in which services can be used. For example, accessing to a function within an authentica-tion protocol.

Finally, at the service composition layer gives an abstract view of the kind of properties exported by services that can be combined for providing NFP for a composition. For example, confidentiality, authentication, privacy and access control can provide security at the service composition layer.

As a first step in our approach,  we started modeling non-functional constraints at the PSM level. Thus, in this paper we  propose the $\pi$-SCM, the services' composition meta-model extended with {\em A-policies} for modeling non-functional constraints (highlighted in Figure  \ref{fig:sodm} and described in Section \ref{sec:piscm}).  $\pi$-SOD-M defines the $\pi$-{\sc Pews}  meta-model providing guidelines for expressing the services' composition and the {\em A-policies} (see Section \ref{sec:pewsmetamodel}), and also defines model to model transformation rules for generating  $\pi$-{\sc Pews} models starting from $\pi$-SCM models that will support executable code generation (see Section \ref{sec:mmrules}). Finally, our work defines model to text transformation rules for generating the program that implements both the services' composition and the associated {\em A-policies} and that is executed by an adapted engine (see Section \ref{sec:implementation}).

