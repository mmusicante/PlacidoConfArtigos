This paper presented $\pi$-SOD-M for specifying and designing reliable service based applications. We model and associate policies to  services' based applications that represent both systems' cross-cutting aspects and use constraints stemming from the services used for implementing them.  We extended the SOD-M method, particularly the $\pi$-SCM (services' composition meta-model) and $\pi$-{\sc Pews} meta-models for representing both the application logic and its associated non-functional constraints and then generating its executable code. We implemented the meta-models on the Eclipse platform and we validated the approach using a use case that uses authentication  policies.

Non-functional constraints are related to business rules associated to the general "semantics" of the application and in the case of services' based applications, they also concern the use constraints imposed by the services. We are currently working on the definition of a method for explicitly expressing such properties in the early stages of the specification of services based applications. Having such business rules expressed and then translated and associated to the services' composition can help to ensure that the resulting application is compliant to the user requirements and also to the characteristics of the services it uses.

Programming non-functional properties is not an easy task, so we are defining a set of predefined {\em A-policy} types with the associated use rules for guiding the programmer when she associates them to a concrete application. {\em A-policy} type  that can also serve as patterns for programming or specializing the way non-functional constraints are programmed.

