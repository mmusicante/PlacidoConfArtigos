%(aqui propuestas relacionadas, analizar sus caracteristicas, si son MDD, si parte de alto nivel, etc.)

Model Driven Development (MDD)~\cite{mda} is a top-down approach for the development of software systems. 
MDD uses the notion of \textit{model} to specify a software system at different levels of abstraction (called \textit{viewpoints}):

\begin{trivlist}
\item \textbf{Computation Independent Models (CIM):} This level focusses on the
environment of the system, as well as on its business and requirement specifications. 
This viewpoint represents the software system at its highest level of abstraction. 
At this moment of the development, the structure and system processing details are still unknown or undetermined. 
 
\item \textbf{Platform Independent Models (PIM):} This level focusses on the system functionality, hiding the details of any particular platform. 
The specification defines those parts of the system that do not change from one platform to another. 

\item \textbf{Platform Specific Models (PSM):} This level focusses on the functionality, in the context of a particular implementation platform.
Models at this level combine the platform-independent view with the specific aspects of the platform to implement the system.  
\end{trivlist}

Besides the notion of model at each level of abstraction, MDD requires the use of \textit{model transformations} within and between levels.
Intra-level transformations are used to provide a unified representation of concepts of a given level.
Inter-level transformation implement a refinement process between levels.
Transformations may be automatic or semi-automatic.

