%(Aquí lo dejaría con la estructura que teníamos para Caise, presentamos el framework y posteriormente comentamos modelos y transformaciones)

In this section we present $\pi$SOD-M, an MDD based methodology for building service compositions with non-functional requirements. $\pi$SOD-M provides concepts for  modelling  non-functional requirements at the early stages of the development.
%The method uses the concept of \textit{Policy}~\cite{Espinosa-Oviedo2011a} to represent non-functional requirements.

\begin{figure}[h]
\centering
\includegraphics[width=1.0\textwidth]{figs/piSODM}
\caption{$\pi$SOD-M Overview.}
\label{fig:piSOD-M}
\end{figure}


As shown in Figure \ref{fig:piSOD-M}, according to MDA guidelines, $\pi$-SODM meta-models are organized into three levels: 
\begin{itemize}
\item CIM (\textit{Computational Independent Models}). Two models are defined at the CIM level: \textit{value model}
and \textit{BPMN model}. Referring to the "To Publish Music" application,  the designer starts defining an E3value model \footnote{The E3 value model is a business model that represents a business case %graphically as a set of value exchanges ($\nabla$$\triangle$) and value activities (rounded boxes) performed by business actors (squared boxes) 
and allows  to understand the environment in which the services' composition will be placed \cite{e3value}.}  at the CIM level and then the corresponding models of the PIM are generated leading to a services' composition model (SCM).

 \item PIM (\textit{Platform Independent Models}). Our method defines four meta-models: \textit{$\pi$-UseCase}, \textit{$\pi$-ServiceProcess}, and \textit{$\pi$-ServiceCom\-po\-si\-tion}.
%and \textit{$\pi$-PEWS}.
 
 The \textit{$\pi$-UseCase} meta-model describes functional and non-functional requirements.Non-functional requirements are defined as \textit{constraints} over actions and data.
 
The \textit{$\pi$-ServiceProcess} meta-model defines the concept of \textit{service contract} to represent constraints over data and actions 
%that must be performed upon certain conditions.
%The \textit{$\pi$-ServiceProcess} meta-model gathers the constraints
described in the \textit{$\pi$-UseCase}.
%model into contracts that are associated
%with services.

The \textit{$\pi$-ServiceComposition} meta-model provides the concept  \textit{Policy}
to integrate contracts with similar non-functional requirements.
For instance, security and privacy restrictions may be grouped into a security policy.
%\textit{$\pi$-ServiceComposition} models can be refined into PSMs.
 
  \item PSM (\textit{Platform Specific Models}). At the PSM level we have lower-level models that can be automatically translated into actual computer programs. The \textit{$\pi$-PEWS} meta-model provides concepts for modelling service compositions.

\end{itemize}


$\pi$SOD-M (Figure~\ref{fig:piSOD-M}) proposes the generation of a set of models at different levels of abstraction, as well as transformations between these models.
$\pi$SOD-M models represent both the functional aspects of the application as well as its non-functional constraints.
\textit{$\pi$-PEWS} models are textual descriptions of service compositions that can be translated into any service composition language, such as BPEL~\cite{bpel03} or PEWS~\cite{BaCAM05,Placido2010LTPD}.
This can be accomplished by defining: \textit{(i)} a model-to-model transformation, from a \textit{$\pi$-ServiceComposition} model to the corresponding PSM, and \textit{(ii)} a model-to-text transformation, from the this PSM to the composition language.

Extending from the highest level of abstraction of the MDA, $\pi$-SODM provides  a conceptual structure to: first, capture the system requirements and specification in high-level abstraction models (computation independent models, CIMÕs); next,  starting from such models build platform independent models (PIMÕs) specifying the system details; next transform such models into platform specific models (PSMÕs) that bundles the specification of the system with the details of the targeted platform; and finally, serialize such model into the working-code that implements the system. 


%Constraints are restrictions that must be verified during the execution of the application.
%An example of this is the requirement of the user's authentication for executing some system functions.
%


%Now, consider that besides the services' composition that represents the order in which the services are called for implementing the application "To Publish Music" it is necessary to model  other requirements that represent the (i) conditions imposed by services for being contacted, for example the fact the Facebook and Twitter require authentication protocol in order to call their methods for updating the wall; (ii) the conditions stemming from the business rules of the application logic, for example the fact that the walls in Facebook and Twitter must show the same song title and if this is not possible then none of them is updated. 
