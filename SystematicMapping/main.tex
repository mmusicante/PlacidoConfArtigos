\documentclass[preprint,12pt]{elsarticle}
\usepackage{geometry}
%\geometry{letterpaper}                   % ... or a4paper or a5paper or ...
\usepackage{graphicx}
\usepackage{xspace}
\usepackage{amssymb} 
\usepackage{epstopdf}
%% Use the option review to obtain double line spacing
%% \documentclass[preprint,review,12pt]{elsarticle}
   
%% Use the options 1p,twocolumn; 3p; 3p,twocolumn; 5p; or 5p,twocolumn
%% for a journal layout:
%% \documentclass[final,1p,times]{elsarticle}
%% \documentclass[final,1p,times,twocolumn]{elsarticle}
%% \documentclass[final,3p,times]{elsarticle}
%% \documentclass[final,3p,times,twocolumn]{elsarticle}
%% \documentclass[final,5p,times]{elsarticle}
%% \documentclass[final,5p,times,twocolumn]{elsarticle}

%% if you use PostScript figures in your article
%% use the graphics package for simple commands
%% \usepackage{graphics}
%% or use the graphicx package for more complicated commands
%% \usepackage{graphicx}
%% or use the epsfig package if you prefer to use the old commands
%% \usepackage{epsfig}

%% The amssymb package provides various useful mathematical symbols
\usepackage{amssymb}
%% The amsthm package provides extended theorem environments
%% \usepackage{amsthm}

%% The lineno packages adds line numbers. Start line numbering with
%% \begin{linenumbers}, end it with \end{linenumbers}. Or switch it on
%% for the whole article with \linenumbers after \end{frontmatter}.
%% \usepackage{lineno}

%% natbib.sty is loaded by default. However, natbib options can be
%% provided with \biboptions{...} command. Following options are
%% valid:

%%   round  -  round parentheses are used (default)
%%   square -  square brackets are used   [option]
%%   curly  -  curly braces are used      {option}
%%   angle  -  angle brackets are used    <option>
%%   semicolon  -  multiple citations separated by semi-colon
%%   colon  - same as semicolon, an earlier confusion
%%   comma  -  separated by comma
%%   numbers-  selects numerical citations
%%   super  -  numerical citations as superscripts
%%   sort   -  sorts multiple citations according to order in ref. list
%%   sort&compress   -  like sort, but also compresses numerical citations
%%   compress - compresses without sorting
%%
%% \biboptions{comma,round}

% \biboptions{}


\journal{Journal of Systems and Software}

%%% OUR MACROS %%%
\newcommand{\COMMENT}[1]{ }

\usepackage[usenames,dvipsnames]{xcolor}


\usepackage{amsmath}
\usepackage[thmmarks,amsmath]{ntheorem}

\newcommand{\openbox}{\leavevmode
  \hbox to.77778em{%
  \hfil\vrule
  \vbox to.675em{\hrule width.6em\vfil\hrule}%
  \vrule\hfil}}

\theoremstyle{plain}
\theoremheaderfont{\normalfont\bfseries}
\theorembodyfont{\normalfont}
\theoremseparator{}
\theoremindent0cm
\theoremnumbering{arabic}
\newtheorem{algo}{Algorithm}

\theoremstyle{plain}
%\theoremheaderfont{\normalfont\itshape}
\theoremheaderfont{\normalfont\bfseries}
\theorembodyfont{\normalfont}
\theoremseparator{}
\theoremindent0cm
\theoremnumbering{arabic}
\theoremsymbol{\ensuremath{\openbox}} 
\newtheorem{example}{Example}


\theoremstyle{plain}
\theoremheaderfont{\normalfont\bfseries}
\theorembodyfont{\normalfont}
\theoremseparator{.}
\theoremindent0cm
\theoremnumbering{arabic}
\theoremsymbol{\ensuremath{\Box}} 
\newtheorem{defi}{Definition}

\theoremstyle{plain} 
\theoremsymbol{\ensuremath{\Box}} 
\theoremseparator{.} 
\newtheorem{prop}{Property}

\def\FlyingPig{\textsl{FlyingPig}}

\newcounter{numberInTrivlist}

\newenvironment{numtrivlist}{\begin{list}{\rm \arabic{numberInTrivlist})} 
                                         {\usecounter{numberInTrivlist}
                                          \setlength{\leftmargin}{0pt}
                                          \setlength{\rightmargin}{0pt}
                                          \setlength{\itemindent}{12pt}
                                          \setlength{\listparindent}{0pt}}}
                            {\end{list}}

\newenvironment{itemizedTrivlist}{\begin{list}{\rm ~\hspace{2mm} $\bullet$\ } 
                                         {\setlength{\leftmargin}{0pt}
                                          \setlength{\rightmargin}{0pt}
                                          \setlength{\itemindent}{12pt}
                                          \setlength{\listparindent}{0pt}}}
                            {\end{list}}

\usepackage{listings}


\lstset{numbers=right, numbersep=5pt, numberstyle=\tiny, stepnumber=1,escapechar=\!,columns=fullflexible,
        morekeywords={procedure,let,for,do,if,then,else,add,choose,end,while,
        true,false,rise,exception,extend,resume,to,return,function}}

\newcommand{\pisodm}[0]{$\pi$SOD-M\xspace}

\begin{document}

\begin{frontmatter}

%% Title, authors and addresses

%% use the tnoteref command within \title for footnotes;
%% use the tnotetext command for the associated footnote;
%% use the fnref command within \author or \address for footnotes;
%% use the fntext command for the associated footnote;
%% use the corref command within \author for corresponding author footnotes;
%% use the cortext command for the associated footnote;
%% use the ead command for the email address,
%% and the form \ead[url] for the home page:
%%
%% \title{Title\tnoteref{label1}}
%% \tnotetext[label1]{}
%% \author{Name\corref{cor1}\fnref{label2}}
%% \ead{email address}
%% \ead[url]{home page}
%% \fntext[label2]{}
%% \cortext[cor1]{}
%% \address{Address\fnref{label3}}
%% \fntext[label3]{}

\title{Designing  Service-Oriented Applications in
the Presence of Non-Functional Properties: a mapping study}

%% use optional labels to link authors explicitly to addresses:
%% \author[label1,label2]{<author name>}
%% \address[label1]{<address>}
%% \address[label2]{<address>}



\author[inst3]{Umberto Souza da Costa}
\author[inst3]{Martin A. Musicante}
\author[inst5]{Pl\'acido A. Souza Neto}
\author[inst6,inst4]{Genoveva Vargas-Solar}




\address[inst3]{Universidade Federal do Rio Grande do Norte -- Natal, Brazil}
%\address[inst4]{Universidad de las Am\'ericas-Puebla, LAFMIA -- Cholula, Mexico}
\address[inst5]{Instituto Federal de Educa\c{c}\~{a}o, Ci\^{e}ncia e Tecnologia do Rio Grande do Norte -- Natal, Brazil}
\address[inst6]{CNRS, LIG-LAFMIA, Saint Martin d'H\`eres, France}


\begin{abstract}



\end{abstract}

\begin{keyword}
%% keywords here, in the form: keyword \sep keyword
MDA \sep Non-Functional Requirements \sep Service-based software process.

%% MSC codes here, in the form: \MSC code \sep code
%% or \MSC[2008] code \sep code (2000 is the default)

\end{keyword}

\end{frontmatter}

%%
%% Start line numbering here if you want
%%
% \linenumbers

%% main text
%*********************************************************************************************************
\section{Introduction}
\label{sec:intro}

Service oriented computing~\cite{Papazoglou2007} is at the origin of an evolution in the field of software development.
Service oriented methods advocates for the construction of software systems formed by the composition of heterogeneous, loosely coupled modules.
These modules (or services) communicate in order to achieve a common purpose.
 
An important challenge of service oriented development is  to ensure the alignment between the requirements imposed by the business logic and the IT systems actually being developed.
(Moreover, IT systems need to evolve according to the business needs.)
Thus, organizations are seeking for mechanisms to bridge the gap between the actually developed systems and their business needs~\cite{bell}. 
The literature stresses the need for methodologies and techniques for service oriented analysis and design, claiming that they are the cornerstone in the development of meaningful service based applications~\cite{5}.  

In Service-Oriented Computing, pre-existing services are
combined to produce applications and provide the business logic. 
The selection of services is usually guided by the \textit{functional} requirements of the application being developed~\cite{1,2,decastro1,PapazoglouH06}. 
(Functional properties of a computer system are characterized by the effect produced by the system when given a defined input.)
Functional properties are not the only crucial aspect in the software development process. 
Other properties need to be addressed to fit in the application with its context.
These other aspects are called Non-Functional Properties.

Non-functional aspects of the services, often expressed as requirements and constraints in general purpose methodologies, are not usually considered from the early phases of the (service) software process.
Most methods consider them only after the application has been implemented, in order to ensure some level of reliability (e.g., data privacy, exception handling, atomicity, data persistence). 
This leads to service based applications that are partly specified and, thereby, partly compliant with the requirements of the application.
Ideally, non-functional requirements should be considered along with all the stages of the software development. 
The adoption of non-functional specifications from the early states of development
can help the developer to produce applications that are capable of dealing with
the application context.

\bigskip

Model Driven Development (MDD) is a top-down approach for the development of software systems. 
The main ideas of MDD were originally proposed by the Object Management Group (OMG)~\cite{mda}, as a set of guidelines for the structuring of specifications.
The technique advocates for the use of \textit{models} to specify a software system at different levels of abstraction (called \textit{viewpoints}). 
These models are successively refined from abstract specifications into actual computer programs.

In this context, we argue that the convergence of model-driven software development, service orientation and better techniques for documenting and improving business processes are key to make real the idea of rapid, accurate development of software that serves, rather than dictates the needs of its users~\cite{watson}. 

In this work, we are interested in the extension of the \textit{Service Oriented Development Method} (SOD-M)~\cite{decastro1}, to support non-functional aspects, from the early stages of software development.
SOD-M is aligned with the MDD directives and proposes models, practices and techniques for the development of service-based applications.
SOD-M does not provide support for the specification of non-functional requirements, such as
security, reliability, and efficiency. 

The main goals of our work are:
\begin{trivlist}
\item \textit{(i)} To define a NFR model including a set of concepts need for the modeling of NFR in service-oriented applications.
\item \textit{(ii)} To propose a methodology for supporting the construction of service-oriented applications, taking into account both functional and non-functional requirements;
\item \textit{(iii)} To improve the construction process by providing an abstract view of the application and ensure the conformance to its specification;
\item \textit{(iv)} To reduce the programming effort through the semi-automatic generation of  models for the application, to produce concrete implementations from high abstraction models;
\end{trivlist}

The rest of the paper is organized as follows. 
Section\dots







\section{Background}
\label{sec:background}
%Conceptos existentes en NFR y metodolog��as similares

This section introduces the vocabulary and concepts related to nonfunctional properties of service based applications.

\subsection{Adding NFP to service compositions}

%\subsection{Problem space and solution space}

\subsection{Models, methodologies and environments}

\subsection{Related work}


%..--..--..--..--..--..--..--..--..--..--..--..--..--..--..--..--..--..--..--..--..--..--..--..--..--..--..--..--..--..--..--..--..--..--..--..--..--..--
\section{Mapping process}\label{sec:mappingprocess}



\subsection{Research questions}

\subsection{Search and screening of papers}

{\em ("non-functional" OR "non functional" OR quality OR NF OR QoS) AND (property OR requirement OR aspect OR attribute OR parameter OR concern OR constraint OR approach OR policy OR contract) AND ("web service" OR "service composition" OR "service based application" OR "service-based application")}

\subsection{Classification scheme and data extraction}
%..--..--..--..--..--..--..--..--..--..--..--..--..--..--..--..--..--..--..--..--..--..--..--..--..--..--..--..--..--..--..--..--..--..--..--..--..--..--

%..--..--..--..--..--..--..--..--..--..--..--..--..--..--..--..--..--..--..--..--..--..--..--..--..--..--..--..--..--..--..--..--..--..--..--..--..--..--
\section{Outcomes}
\label{sec:outcomes}

\subsection{RQ1: Which journals and conferences publish papers on NFR for web service based applications ?}

\subsection{RQ2: What are the most investigated aspects of NFR for web applications over the last years}

\subsection{RQ3 : Which development environments have been proposed for addressing NFR for web service based application ?}

\subsection{Discussion}
%..--..--..--..--..--..--..--..--..--..--..--..--..--..--..--..--..--..--..--..--..--..--..--..--..--..--..--..--..--..--..--..--..--..--..--..--..--..--



%*********************************************************************************************************
\section{Concluding remarks}\label{sec:conclusions}
This paper presented \pisodm for designing and developing reliable service-based applications. 
\pisodm is an MDA methodology that extends a previously defined method (called SOD-M) to include Non-Functional Requirements.
These requirements are taken into account from the initial stages of the software development process.
Non-functional constraints are related to business rules associated to the behavior of the application and, in the case of service-based applications, they are also concerned with constraints imposed by the services. 

Our methodology includes two CIM-level models, three PIM-level models and one PSM-level model. 
We implemented the meta-models on the Eclipse platform and we validated the approach by using an industrially inspired use case.

Our case study demonstrates the applicability of \pisodm.
The case study was developed together with our industrial partner, GCP Global.
The Company is using \pisodm for the development of their product.
The case study presented here is a simplified version of their application.


%% References with bibTeX database:

\bibliographystyle{plain}
\bibliography{biblio}


\end{document}

%%
%% End of file `elsarticle-template-1a-num.tex'.
