
This section introduces the vocabulary and concepts related to nonfunctional properties of service based applications.

\subsection{Adding NFP to service compositions}
In Service-Oriented Computing~\cite{Papazoglou2007}, pre-existing services are
combined to build an application business logic.
The selection of services is usually guided by the \textit{functional} requirements of the application being developed~\cite{2,decastro1,PapazoglouH06}\footnote{Functional properties of a computer system are characterized by the effect produced by the system when given a defined input.}.
An important challenge of service-oriented development is  to ensure the alignment between the functional requirements imposed by the business logic and the functions actually being developed.

Functional properties are not the only  aspect in the software development process.
Non-functional properties, such as data privacy, exception handling, atomicity  and, data persistence, need to be addressed  to fit in the application.
Adding non-functional properties and respecting services constraints while composing services is a complex task that implies programming  protocols for instance authentication protocols to call a service, and atomicity (exception handling and recovery) for ensuring a true synchronization of the results produced by the service methods calls.

Even if service-oriented computing benefits from reuse, this reuse is usually guided only by functional requirements. 
Ideally, non-functional requirements should be considered in every phase of the software development.
Yet,  they are partially or rarely methodologically derived from the specification, being usually added once the code has been implemented. 
In consequence, the development process does not fully preserve the compliance and reuse expectations provided by the service oriented computing methods.


The literature stresses the need for methodologies and techniques for service oriented analysis and design 
%since they are the cornerstone in the development of meaningful service based applications
~\cite{Papazoglou2007}. 
Existing approaches argue that the convergence of model-driven software development, service orientation,   and  business processes improvement are key for developing accurate  software~\cite{watson}. 
Model Driven Development (MDD)  for software systems is mainly characterized by the use of models as a product~\cite{Selic03}.
These models are successively refined from abstract specifications into actual computer programs.

%\subsection{Problem space and solution space}

\subsection{Models, methodologies and environments}

Non-functional properties of  services, often expressed as requirements and constraints in general purpose methodologies, are not always fully considered from the early phases of the (service) software process. Most methods integrate them only after the application has been implemented. This leads to service based applications that are partialy specified and, thereby, partialy compliant with the requirements of the application.


The adoption of non-functional specifications from the early states of development can help the developer to produce applications that  can deal with the application context.



\subsection{Related work}