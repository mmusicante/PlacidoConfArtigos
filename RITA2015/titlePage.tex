\documentclass[12pt,twoside]{article}
\usepackage{Jcst}
%Equations,theorems,lemmas,etc. are created using the traditional enviroment
%

\usepackage{xspace}

\usepackage{amsmath}
\usepackage[thmmarks,amsmath]{ntheorem}

\newcommand{\openbox}{\leavevmode
  \hbox to.77778em{%
  \hfil\vrule
  \vbox to.675em{\hrule width.6em\vfil\hrule}%
  \vrule\hfil}}

\theoremstyle{plain}
\theoremheaderfont{\normalfont\bfseries}
\theorembodyfont{\normalfont}
\theoremseparator{}
\theoremindent0cm
\theoremnumbering{arabic}
\newtheorem{algo}{Algorithm}

\theoremstyle{plain}
%\theoremheaderfont{\normalfont\itshape}
\theoremheaderfont{\normalfont\bfseries}
\theorembodyfont{\normalfont}
\theoremseparator{}
\theoremindent0cm
\theoremnumbering{arabic}
\theoremsymbol{\ensuremath{\openbox}} 
\newtheorem{example}{Example}


\newcommand{\pisodm}[0]{$\pi$SOD-M\xspace}
\def\FlyingPig{\textsl{FlyingPig}\xspace}

\newcounter{numberInTrivlist}

\newenvironment{numtrivlist}{\begin{list}{\rm \arabic{numberInTrivlist})}
                                         {\usecounter{numberInTrivlist}
                                          \setlength{\leftmargin}{0pt}
                                          \setlength{\rightmargin}{0pt}
                                          \setlength{\itemindent}{12pt}
                                          \setlength{\listparindent}{0pt}}}
                            {\end{list}}

\newenvironment{itemizedTrivlist}{\begin{list}{\rm ~\hspace{2mm} $\bullet$\ }
                                         {\setlength{\leftmargin}{0pt}
                                          \setlength{\rightmargin}{0pt}
                                          \setlength{\itemindent}{12pt}
                                          \setlength{\listparindent}{0pt}}}
                            {\end{list}}

\begin{document}

\setcounter{page}{1}

%%%%%%%%%%%%%%%%%%%%%%%%%%%%Set Page Head%%%%%%%%%%%%%%%%%%%%%%%%%%%%%%
\setpageinformation
%{Head of the first page} {Running head of odd pages}
{}
{ }{ }{}{Apr.}{2015}
{\pisodm: Building Service-Oriented Applications}
%%%%%%%%%%%%%%%%%%%%%%%%%%%%%%%%%%%%%%%%%%%%%%%%%%%%%%%%%%%%%%%%%%%%%%%
%===========================================================

%\begin{CJK}{GBK}{song}

\title{\pisodm: Building Service-Oriented Applications in the Presence of Non-Functional Requirements}

\author{Khalid Belhajjame$^1$, Valeria~de~Castro$^2$, Umberto~Souza~da~Costa$^3$, Javier~A.~Espinosa-Oviedo$^4$, Martin~A.~Musicante$^{3,*}$, Pl\'acido~A.~Souza~Neto$^5$, Genoveva~Vargas-Solar$^6$, \\ Jos\'e-Luis~Zechinelli-Martini$^4$}

\footnotetext{$^*$.Corresponding author.}
\footnotetext{$^1$.Universit\'e de Paris - Dauphine -- Paris, France.
\textsl{kbelhajj@googlemail.com}.
}
\footnotetext{$^2$.Universidad Rey Juan Carlos -- M\'{o}stoles, Spain.
\textsl{valeria.decastro@urjc.es}.
}
\footnotetext{$^3$.Universidade Federal do Rio Grande do Norte -- Natal, Brazil.
\textsl{\{umberto,mam\}@dimap.ufrn.br}.
}
\footnotetext{$^4$.Universidad de las Am\'ericas-Puebla, LAFMIA -- Cholula, Mexico.
\textsl{javiera.espinosa@gmail.com}, \textsl{joseluis.zechinelli@udlap.mx}.
}
\footnotetext{$^5$.Instituto Federal de Educa\c{c}\~{a}o, Ci\^{e}ncia e Tecnologia do Rio Grande do Norte -- Natal, Brazil.
\textsl{placido.neto@ifrn.edu.br}.
}
\footnotetext{$^6$.CNRS, LIG-LAFMIA, Saint Martin d'H\`eres, France.
\textsl{genoveva.vargas@imag.fr}.
}


%\maketitle


\begin{abstract}
Specifying non-functional requirements (NFRs) is a complex task, being usually dealt with on the later phases of the software process.
The late inclusion of NFRs in the development may compromise the quality of the deployed application.
This paper presents \pisodm, a method and associated tools that
\textit{(i)}  allows the early specification of non-functional requirements in a principled way: users are abstracted away from low level details;
\textit{(ii)} embraces the MDA philosophy, generating models (code) whenever possible.
The pro\-po\-sed solution has been utilized in the context of an industrial and real case study.
\end{abstract}

\keywords{MDA , Non-Functional Requirements, Service-based software process}

\begin{multicols}{2}
\normalsize

%% main text
%*********************************************************************************************************
\section{Introduction}
\label{sec:intro}

%Service oriented computing is at the origin of an evolution in the field of software development.
%Service oriented methods advocates for the construction of software systems formed by the composition of heterogeneous, loosely coupled modules.
%These modules (or services) communicate in order to achieve a common purpose.


%(Moreover, IT systems need to evolve according to the business needs.)
%Thus, organizations are seeking for mechanisms to bridge the gap between the actually developed systems and their business needs~\cite{bell}.


In Service-Oriented Computing~\cite{Papazoglou2007}, pre-ex\-isting services are
combined to build an application business logic.
The selection of services is usually guided by the \textit{functional}$^7$\footnote{$^7$.Functional properties of a computer system are characterized by the effect produced by the system when given a defined input.} requirements of the application being developed~\cite{2,decastro1,PapazoglouH06}.
An important challenge of service-o\-rien\-ted development is  to ensure the alignment between the functional requirements imposed by the business logic and the functions actually being developed.

Functional properties are not the only aspect to be considered in the software development process.
Non-functional properties, such as data privacy, exception handling, atomicity  and, data persistence, need to be addressed to fit in the application.
%Adding non-functional properties and respecting services constraints while composing services is a complex task that implies implementing protocols for instance authentication protocols to call a service, and atomicity (exception handling and recovery) for ensuring a true synchronization of the results produced by the service methods calls.
%Even if service-oriented computing benefits from reuse, this reuse is usually guided only by functional requirements.
%
%Furthermore,  often they are  not fully considered in the specification process of existing services' oriented development methods. Rather,
%They   are either ensured by the underlying execution platform, or they are programmed through ad-hoc protocols once the application has been programmed.
Ideally, Non-Functional Requirements (NFR) should be considered in every phase of the software development.
Yet, they are partially or rarely methodologically derived from the specification, being usually added once the code has been im\-ple\-men\-ted.
In the context of service oriented computing, the late consideration of non-functional requirements in the development process does not fully preserve the compliance and re\-use expectations promoted by the service oriented paradigm.


%Yet, non-functional properties of  services, often expressed as requirements and constraints in general purpose methodologies, are not usually considered from the early phases of the (service) software process.
%Most methods integrate them only after the application has been implemented.
%This leads to service based applications that are partialy specified and, thereby, partialy compliant with the requirements of the application.

% song title disseminated in the walls of the user's Facebook and Twitter accounts.





\end{multicols}
%\end{CJK}
\end{document}



