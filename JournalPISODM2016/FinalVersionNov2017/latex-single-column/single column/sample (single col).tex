%%%%%%%%%%%%%%%%%%%%%%
\documentclass{singlecol-new}
%%%%%%%%%%%%%%%%%%%%%%

\usepackage{natbib,stfloats}
\usepackage{mathrsfs}

\def\newblock{\hskip .11em plus .33em minus .07em}

\theoremstyle{TH}{
\newtheorem{lemma}{Lemma}
\newtheorem{theorem}[lemma]{Theorem}
\newtheorem{corrolary}[lemma]{Corrolary}
\newtheorem{conjecture}[lemma]{Conjecture}
\newtheorem{proposition}[lemma]{Proposition}
\newtheorem{claim}[lemma]{Claim}
\newtheorem{stheorem}[lemma]{Wrong Theorem}
\newtheorem{algorithm}{Algorithm}
}

\theoremstyle{THrm}{
\newtheorem{definition}{Definition}[section]
\newtheorem{question}{Question}[section]
\newtheorem{remark}{Remark}
\newtheorem{scheme}{Scheme}
}

\theoremstyle{THhit}{
\newtheorem{case}{Case}[section]
}

\makeatletter
\def\theequation{\arabic{equation}}

%\JOURNALNAME{\TEN{\it Int. J. System Control and Information
%Processing,
%Vol. \theVOL, No. \theISSUE, \thePUBYEAR\hfill\thepage}}%
%
%\def\BottomCatch{%
%\vskip -10pt
%\thispagestyle{empty}%
%\begin{table}[b]%
%\NINE\begin{tabular*}{\textwidth}{@{\extracolsep{\fill}}lcr@{}}%
%\\[-12pt]
%Copyright \copyright\ 2012 Inderscience Enterprises Ltd. & &%
%\end{tabular*}%
%\vskip -30pt%
%%%\vskip -35pt%
%\end{table}%
%}
\makeatother

%%%%%%%%%%%%%%%%%
\begin{document}%
%%%%%%%%%%%%%%%%%

\setcounter{page}{1}

\LRH{F. Wang et~al.}

\RRH{Metadata Based Management and Sharing of Distributed Biomedical
Data}

\VOL{x}

\ISSUE{x}

\PUBYEAR{xxxx}

\BottomCatch

%\CLline

\PUBYEAR{2012}

\subtitle{}

\title{Metadata Based Management and Sharing of Distributed Biomedical Data}

%
\authorA{Fusheng Wang}
%
\affA{Department of Biomedical Informatics,\\ Emory University,\\ Atlanta, GA, USA \\
Fax: +1 \qquad E-mail: fusheng.wang@emory.edu}
%
%
\authorB{Fusheng Wang}
\affB{Department of Biomedical Informatics,\\ Emory University,\\ Atlanta, GA, USA \\
Fax: +1 \qquad E-mail: fusheng.wang@emory.edu}
%

%
%\authorA{Fusheng Wang\footnote{Work done while working at Siemens Corporate Research.} }
%
%\affA{Department of Biomedical Informatics, Emory University
%\newline
%36 Eagle Row, Ste 589, Atlanta, GA 30322, USA}
%
%
%
%\authorB{\footnotesize Cristobal Vergara-Niedermayr\footnote{Work done while working at Siemens Corporate Research.}}
%\affB{Oracle \newline
 %New Jersey, USA}
%
%
\authorC{Peiya Liu}
\affC{Department of Integrated Data Systems, Siemens Corporate
Research \newline 755 College Road East, Princeton 08540, USA}

\begin{abstract}
Biomedical research becomes reliant on multi-disciplinary,
multi-institutional collaboration, and data sharing is becoming increasingly important for researchers to reuse experiments, pool expertise and validate approaches. However, there are many hurdles for data sharing, including the unwillingness to share,  lack of flexible data model for providing context information for shared data,  difficulty to share syntactically and semantically consistent data across distributed institutions, and expensive cost to provide tools to share the data.  In our work,  we develop a Web-based collaborative biomedical data sharing platform {\em SciPort} to support biomedical data sharing across distributed organizations. SciPort provides a generic metadata model for researchers to flexibly customize and organize the data. To enable convenient data sharing, SciPort provides a central server based data sharing architecture, where data can be shared by one click through publishing metadata to the central server. To enable consistent data sharing, SciPort provides collaborative distributed schema management across distributed sites. To enable semantic consistency for data sharing, SciPort provides semantic tagging through controlled vocabularies. SciPort is lightweight and can be easily deployed for building data sharing communities for biomedical research.
\end{abstract}

\KEYWORD{Metadata; Scientific Data Management; Data Sharing;  Data Integration; Computer Supported Collaborative Work.}

\REF{to this paper should be made as follows: Rodr\'{\i}guez
Bol\'{\i}var, M.P. and Sen\'{e}s Garc\'{\i}a, B. (xxxx) `The
corporate environmental disclosures on the internet: the case of
IBEX 35 Spanish companies', {\it International Journal of Metadata,
Semantics and Ontologies}, Vol. x, No. x, pp.xxx\textendash xxx.}

\begin{bio}
Manuel Pedro Rodr\'iguez Bol\'ivar received his PhD in Accounting at
the University of Granada. He is a Lecturer at the Department of
Accounting and Finance, University of Granada. His research
interests include issues related to conceptual frameworks of
accounting, diffusion of financial information on Internet, Balanced
Scorecard applications and environmental accounting. He is author of
a great deal of research studies published at national and
international journals, conference proceedings as well as book
chapters, one of which has been edited by Kluwer Academic
Publishers.\vs{9}

\noindent Bel\'en Sen\'es Garc\'ia received her PhD in Accounting at
the University of Granada. She is a Lecturer at the Department of
Accounting and Finance, University of Granada. Her research
interests are related to cultural, institutional and historic
accounting and in environmental accounting. She has published
research papers at national and international journals, conference
proceedings as well as chapters of books.\vs{8}

\noindent Both authors have published a book about environmental
accounting edited by the Institute of Accounting and Auditing,
Ministry of Economic Affairs, in Spain in October 2003.
\end{bio}


\maketitle


 \section{Introduction}

%The needs for data sharing
With increased complexity of scientific problems, biomedical
research is increasingly a collaborative effort across multiple
institutions and disciplines.  Data sharing is becoming critical for
validating approaches and ensuring that future research can build on
previous efforts and discoveries. As a result, data sharing is often
required by scientific funding agencies to share the data produced
in grant projects. For example,  the National Institutes of Health
(NIH) of US requires data sharing for NIH funded projects of
\$500,000 or more in direct costs in any one year.

To support large scale collaborative biomedical research, NIH
provides large-scale collaborative project awards  for a team of
independently funded investigators to synergize and integrate their
efforts, and the awards mandate the research results and data to be
shared \cite{nih03datasharing,nih08policy}.  The Network for
Translational Research (NTR): Optical Imaging in Multimodality
Platforms\cite{ntr} is one of such collaborative projects on the
development, optimization, and validation of imaging methods and
protocols for rapid translation to clinical environments.  It
requires not only managing the complex scientific research results,
but also sharing the data across hundreds of research collaborators.
As another example, Siemens Healthcare has research collaborations
with hundreds of research sites distributed across the US, each
providing Siemens marketing support by periodically delivering white
papers, case reports, clinic methods, clinic protocols,
state-of-the-art images, etc. In the past, there were no convenient
methods for research partners to share data with Siemens, and mostly
data were delivered through media such as emails, CDs and hard
copies. This made it very difficult to organize, query and integrate
the shared data.

Sharing biomedical research data is important but difficult
\cite{piwowar08culture,birnholtz03data}. One major difficulty is the
unwillingness to share, as investigators may restrict access to data
to maximize the professional development and economic benefit
\cite{bekelman03}. With increased awareness of this issue, the
government and funding agencies are developing  strategies and
policies to promote and enforce the sharing of data
\cite{piwowar08culture,dsic,cabigconnected}. While the social issue
of data sharing is challenging, this paper will mainly focus on the
development of a biomedical data sharing system which makes data
sharing becoming flexible and easy for humans on collaborative
research.

\subsection{Complexity of Biomedical Data}

Biomedical data can be in heterogeneous formats such as structured
data, standard based medical images such as DICOM \cite{dicom}, non
standard medical images such as optical images, raw equipment data,
spreadsheets, PDF files,  XML documents, and many others. Thus, the
data are  a mix of structured data (often deeply hierarchical) and
files. Meanwhile, the data structures of biomedical research data
can be very complex, ranging from different primitive data types, to
lists and tables, from flat structures to deep nested structures,
and so on.  It is often difficult for investigators to create a
sophisticated schema that can capture enough context information or
metadata information, if the data types do not have a public,
centralized, and well-recognized database. During our interviews
with more than a dozen biomedical research investigators, we found
that major investigators use spreadsheet like tools and manage their
data through operating system folder organized files. To reduce the
cost and complexity for data sharing, a data sharing system needs to
provide an extensible architecture that can be easily customized by
researchers with their own metadata models, even without programming
skills.

\subsection{Management of Data Sharing}

Even with willingness to share data, in many cases an investigator
would like to control when to share data, and who can access the
data. An investigator may want to keep unpublished data only to his
or her own group or collaborative partners, and selectively publish
a subset of data for sharing. An investigator may also want to stop
sharing certain data at any time, for example, when he or she finds
a quality problem of the data. An investigator may also have new
results added on existing shared data and want to keep the data
current. A manageable data sharing system with convenient operations
can potentially increase the willingness for users to share, as the
effort for data sharing can be much reduced.


\subsection{Data Sharing Architecture}

There are three major types of data sharing architectures: i)
Centralized, multiple datasets hosted at a single location in a
common schema. For example, the Cancer Genome Atlas (TCGA) Data
Portal \cite{tcga} manages genome related data; National Biomedical
Imaging Archive \cite{nbia} manages DICOM based medical research
data. ii) Federated, with a virtual view of physically separate
datasets. For example, Cancer Biomedical Informatics Grid
(caBIG{\small \textregistered})  \cite{cabig} is a Grid based data
federation infrastructure that supports a CQL query language across
distributed data sources.  iii) Distributed, physically and
virtually separate datasets.  Centralized approach is often limited
to common data types. Biomedical research, however,  generates
complex data and often new data types.  Distributed approach is
often difficult to retrieve, interpret and aggregate results, and
lacks data consistency between research sites.  While caGrid is
becoming widely used in biomedical research community, caGrid itself
has a complex infrastructure and the effort is significant.


Considering the high dynamic nature of biomedical research and the
need of  cost effective data sharing, we develop a hybrid
architecture which combines the benefits of the centralized approach
and the distributed approach. In this approach, a data sharing
central server is provided for multiple distributed data sources,
and stores only published metadata (not raw data or images with
large sizes) from distributed data sources. Users can have flexible
management of data sharing through publishing or unpublishing data
with  a simple operation. The metadata contain context information
of original data sources (including raw data) which are still
managed at distributed research sites. The central server provides
an integrated view of all shared data, and shared data can be easily
aggregated and retrieved from the central server. This architecture
not only is lightweight, but also provides support of consistent
data sharing through collaboratively managing schemas and
semantically tagging of data.

\subsection{Inconsistency of Shared Data}


Another difficulty on sharing data is the incompatible
representation of data between sharing partners. Each site may use
different schemas (e.g., data templates) and represent data in
different formats; and even if multiple partners choose to use the
same schema, the schema may evolve as research progresses. For
example, a field name may be changed or additional constraints may
be added in a schema. Data generated from different versions of a
schema  may not match when searched together. For example, if a
field named ``age'' is changed to ``patient\_age'',  these two field
names will lead to structurally different fields and can not be
searched together as a single field.   Semantic consistency is on
using the same vocabulary to represent knowledge. For example,  one
investigator may use ``Malignant Tumor'' to describe a lesion, and
another investigator  may use `cancer'.    While semantics
annotations can be added to data to enhance semantic
interoperability between shared data, there is a lack of integrated
tools and architecture that can support semantic annotations from
controlled vocabularies.

\subsection{Our Contributions}


To meet the requirements of a data sharing system and overcome the
challenges, we develop SciPort, a Web-based collaborative biomedical
data sharing platform to support collaborative biomedical research.
SciPort brings together the following essential components to
support biomedical data sharing:


\begin{itemize}

\item Generic biomedical experiment metadata model and XML based biomedical data management for research sites to flexibly customize and organize their data (Section
\ref{sec:datamanagementoverview});

\item Hybrid, lightweight data sharing architecture through a Central
Server, and data sharing can be conveniently managed by investigators (Section \ref{sec:sharing});

\item Easy sharing of schemas between distributed research partners (Section \ref{sec:sharingschema});

\item Collaborative management of schemas and their evolution between distributed research sites to maintain syntactic data consistency  (Section \ref{sec:collaborativeschema});

\item Semantic tagging of data to achieve semantically consistent data sharing (Section \ref{sec:tag}).
\end{itemize}


This paper has been significantly extended from previous work
\cite{wang08com}, which discussed preliminary work and the technique
aspect on distributed scientific data management. The work presents
here represents latest results and software development, and has a
major new focus on the human and sharing aspects of biomedical data.
The paper also covers new work on metadata modeling for data
sharing, collaborative tagging for data sharing, user experience on
real world deployment, and so on.



\section{Generic Biomedical Experiment Metadata Modeling and Management} \label{sec:datamanagementoverview}


As discussed in \cite{birnholtz03data}, metadata models serve as the
abstraction of data, and are essential for understanding shared
data. However, metadata  are often incomplete, and important context
information of data \cite{chin04context} is often ignored. However,
many domain specific applications or data models are specific to
certain types of data and limited to a small set of context
information.  To precisely capture as sufficient metadata and
context information as possible for biomedical data sharing,
researchers themselves should be also to customize their own
metadata models easily to meet their research need and data sharing
need.


SciPort was first developed as a biomedical data management system  with
metadata  modeling, authoring, management, viewing, searching and
exchange.

%SciPort takes an XML based approach for data modeling, schema representation, and storage  and queries.

\subsection{Generic Meta Data Modeling for Biomedical Data}

To meaningfuly represent each dataset, we develop a {\em SciPort document} model to represent the metadata and context information of the data.
A SciPort Document can represent both (nested) structured data, files and images.

A SciPort document includes several objects: i) Primitive Data
Types/Fields. Primitive data types are used to represent structural
data, including {\em integer}, {\em float}, {\em date}, {\em text},
and Web-based data types such as {\em textarea}, {\em radiobutton},
{\em checkbox}, {\em URL}, etc.; ii) File. Files can be linked to a
document through the file object; iii) \textit{Reference.}  A
reference type links to another SciPort document; iv) Group. A group
is similar to a table, which aggregates a collection of fields or
nested groups.  There can be multiple instances for a group, like
rows of a table; v)Category. A category relates a list of fields,
e.g., ``patient data'' category, ``experiment data'' category, etc.
Categories are used only at the top level of the content, and
categories are not nested.

Figure \ref{fig:sampledocument} shows an example SciPort document
that describes image annotations on top of medical images.  This
document captures generic information such as the tile, description,
author, creation date and modification date; patient information
such as the age and gender; annotation data such as a number of
annotated tumors marked up on top of the image, and the link to the
image file from which annotations were generated.  Note that in the
example, we can easily represent nested complex information such as
multiple tumors, multiple spatial coordinates and different  data
types such as images and files.

\begin{figure*}
\caption{A Sample SciPort Document}\label{fig:sampledocument}
%\centerline{\epsffile{E:/Inderscience/LATEX-FILES/IJMSO/F1.eps}}
\end{figure*}

\subsection{XML Based Implementation and the Benefits}

The hierarchical nature of the data model fits perfectly with the
tree based XML data model, and we take an XML based approach to
implement the data model -- we call it \textit{SciPort Exchange
Document}.  Users can also easily define their own schemas which are
internally represented with an XML-based schema definition language.
Besides, To provide an intuitive way to present data,  we also
develop a hierarchical model based on XML to organize biomedical
data, thus documents can be quickly browsed and identified through
the hierarchy. For example, we can define a hierarchy with levels of
``site''$\rightarrow$  ``patient'' $\rightarrow$ ``measurement'',
and attach documents at different folder levels. SciPort also
provides fine grained access control  at folder level for the data
in the database \cite{wang09security}.



The self-describing and rich structure of XML makes it possible to
represent arbitrary complex biomedical research data. By modeling
the data as XML documents, we can take advantage of native XML
database technologies to manage biomedical data, thus we can avoid
complex data model and query translation between XML and RDBMS. This
is especially beneficial since users can define arbitrary structured
and nested data formats for their data.  Moreover, powerful queries
can be supported directly on XML databases with the standard XML
query language XQuery \cite{xquery}.


A salient characteristic of SciPort is that the system is highly
adaptable.  SciPort provides a Web-based schema authoring tool to
easily create complex hierarchical metadata schema  model without
any need of programming.  The organizational hierarchy is also
customizable with its hierarchy authoring tool. This allows users to
configure their own biomedical data repository without requiring
expensive and time-consuming services or software development
effort.



\section{Sharing Distributed Biomedical Data} \label{sec:sharing}


While investigators can share their data simply through giving login
information to collaborators, a more systematic approach for sharing
data across research consortia or networks can provide more
flexibility and benefits. SciPort comes with comprehensive data
sharing capabilities: i) Convenience: data sharing is performed by a
single action and data can be selectively shared; ii) Ownership of
data: researchers own and manage their data by their own;  iii)
Flexible sharing control: data sharing can also be revoked by
researchers at any time;  iv) Up-to-date of shared data. As data are
updated or removed, corresponding shared data also need to be
synchronized accordingly to stay current;  v) Consistently
aggregated shared data from distributed sites; and vi) Lightweight.
Sharing is manipulated through metadata, no copy of large volume
data is needed.


These sharing capabilities are implemented through a lightweight,
central server based approach, as discussed next.

\subsection{Sharing Data through a Central Server}


SciPort provides a distributed architecture to share and integrate
data through a Central Server (Figure \ref{fig:integration}). In
this architecture, each research site will have its own Local
SciPort Server which itself functions as an independent Server for
data collection and management. In addition, there will be an
additional Central  Server upon which Local Servers are able to
selectively publish their data (structured documents) (Figure
\ref{fig:docpublish}). Images/files, which are often the major
source of data volume, are still stored on corresponding Local
Servers but are linked from the published documents on the Central
Server. Once a user on the Central Server begins to download a
document from the Central Server, actual data files are downloaded
from the corresponding Local Server that holds the data.

Thus, the Central Server provides a global view of shared data across all
distributed sites, and can also be used as a hub for sharing schemas among
multiple sites. Since data are shared through the metadata (SciPort
Documents), the integration is lightweight. Users on the Central Server will
only have read access to the data.

Figure \ref{fig:integration} illustrates an example SciPort sharing
architecture formed by four Local Servers at four universities: UCI,
UCSF, Dartmouth and Penn. Each Local Server is used for data
collection and management of clinical trial data at its local
institution. Since these clinical trials are under the same research
consortium, they would share their data together by publishing their
data (documents) to the Central Server located at UCI. Members at
NTR research consortia are granted read access on such shared data
through the Central Server. Once the user identifies a data set from
the Central Server and wants to download the data, the user will be
redirected to the corresponding hosting Local Server to download the
data to the client.

\begin{figure*}[t]%2
\caption{The Central Server Based Architecture for Data
Sharing}\label{fig:integration}
%\centerline{\epsffile{E:/Inderscience/LATEX-FILES/IJMSO/F2.eps}}
\end{figure*}

\begin{figure*}[t]%3
\caption{An Example of Publishing an Existing Document}
\label{fig:docpublish}
%\centerline{\epsffile{E:/Inderscience/LATEX-FILES/IJMSO/F3.eps}}
\end{figure*}

\subsection{Data Synchronization}

Shared data may become outdated as new results or analysis are
generated on the data. SciPort provides automated data
synchronization on shared data through synchronization enforcement
in the following operations: i) Create. When a document is created,
the author or publisher has the option to publish this document
(Figure \ref{fig:editpublish}). Once the document is   published, a
``published'' status is added to the document. A user can   also set
up an automatic publishing flag so all new documents will be
automatically published; ii) Update. When an update is performed on
a published document, the document will automatically be republished
to the Central Server; iii) Delete. When a published document is
deleted, it will also be automatically removed from the Central
Server;  iv) Unpublish. A user can stop sharing a document by
selecting the ``unpublish'' operation.  Unpublished documents will
be removed from the Central Server.

\begin{figure*}[t]%4
\caption{An Example of Publishing a New Document}
\label{fig:editpublish}
%\centerline{\epsffile{E:/Inderscience/LATEX-FILES/IJMSO/F4.eps}}
\end{figure*}

\subsection{Security and Trust between Local Servers and the Central Server}

\textbf{Server Verification.} The trust between Local Servers and
the Central Server is implemented through security tokens. For a
Local Server to be accepted into the network, it will be granted a
security token to access the Central Server services. The token will
be imported at the setup step. When a Local Server tries to connect
to the Central Server, the Central Server verifies if the token
matches.


\textbf{Single Sign-on and Security.} One issue for sharing data
from distributed databases is that it is not feasible for Central
Server users to login to every distributed database. When a user
publishes a document, the user already grants the read access of the
document (including the files linked to the document) to the users
on the Central Server, thus another authentication is unnecessary.
Therefore, users on the Central Server should be able to
automatically access shared data from a Local Server in a
transparent way. To support this, the Local Server Document Access
Control Manager has to make sure that the remote download requests
really come from Central Server users who are currently logging on.
We develop a single sign-on method to guarantee the security of the
data sharing, by verifying if an incoming data request to a Local
Server comes from the Central Server and if the current request user
is currently logged in on the Central Server.



\subsection{Sharing Data in Multiple Data Networks}

Data can  be shared not only in a single data network through one
Central Server, but also in multiple data networks through multiple
Central Servers. One organization may want to share the same data in
multiple networks, as demonstrated in an example (Figure
\ref{fig:multiserverintegration}).  There are two networks,  one
centered at UCI and another centered at Stanford. UCI is
collaborating with both networks and needs to share data with both
networks.  UCI will be granted as a partner site and its Local
Server will be configured for both networks. A document can then be
published to any of the Central Servers or both.  This sharing
architecture makes it possible for very flexible data sharing across
different research networks.

\begin{figure*}[t]%5
\caption{Sharing Data in Multiple Data Networks}
\label{fig:multiserverintegration}
%\centerline{\epsffile{E:/Inderscience/LATEX-FILES/IJMSO/F5.eps}}
\end{figure*}

Besides data sharing, SciPort also provides sharing of schemas
between distributed research sites (Figure
\ref{fig:collaborativearchitecture}), discussed in Section
\ref{sec:sharingschema}.

\begin{figure*}[t]%6
\caption{Types of Information Sharing in SciPort}
\label{fig:collaborativearchitecture}
%\centerline{\epsffile{E:/Inderscience/LATEX-FILES/IJMSO/F6.eps}}
\end{figure*}

\subsection{The Benefits of Our Hybrid Data Sharing Architecture}

The architecture of SciPort is a hybrid of centralized approach and
distributed approach.  Centralized approach can provide high
visibility, easy retrieval, easy aggregation within the repository,
but suffers from limited data types or flexibility of customizing
new data types. Distributed approach allows users to flexibly
manage, customize and control their data and data sharing,  but
often suffers from low visibility, difficulty on retrieval,
interpretation, and aggregation, and lacks of data consistency. Our
hybrid approach benefits from both. The Central Server provides high
data visibility, easy retrieval and aggregation of all shared data
from distributed sites.   SciPort allows users to conveniently
customize their data types through creating and updating schemas.
Users can flexibly manage their data sharing through simple
publishing or unpulishing operations.  Data consistency is
maintained through collaborative schema sharing and management
(discussed in next two Sections). The hybrid approach is also
lightweight, as it does not store directly original data such as raw
data or medical images -- which can have high data volumes, but only
metadata or structured documents that describe and abstract the
data.


\section{Sharing Schemas} \label{sec:sharingschema}

Schemas are used to define structures and constraints of documents.
The former includes a mix of (possibly nested) object types defined
in the data model, and the latter includes i) number of instance
constraints for file and group types, ii) minimal and maximal value
constraints; and iii) controlled values.  Schemas are an essential
component  since they are used for i) data validation; ii) document
authoring form generation; iii) data presentation -- templates are
defined based on schemas; and iv) search form generation.


Sharing schemas are critical for sharing data, since the Central
Server glues data together using shared schemas to present and
search data. How to keep data from multiple Local Servers coherent
is also dependent on at what level and how schemas can be shared
consistently.


\subsection{Publishing Schemas}

Schemas can be shared by publishing them to the Central Server. From
a  Local Server {\em Schema Management} menu, schemas created on the
Local Server can be selectively published to target Central Servers,
as shown in the example in Figure \ref{fig:schemapublish}. When a
new document is being published to the Central Server, the
availability of the corresponding schema on the Central Server will
be checked. If the schema is not present, then the schema will also
be published together with the document.


The {\em owner of a schema} is defined as the Local Server on which
the schema is first created.  A schema is identified by its owner
and schema ID.  SciPort also provides comprehensive access control
management \cite{wang09security}, and two roles are related to
schema management: i) {\em organizer} role with privileges to author
and update schemas and ii) {\em publisher} role with  privileges to
publish documents and schemas.


Once a schema is published, it can be shared through the Central
Server. Other Local Servers can reuse schemas by importing schemas
from the Central Server, as shown in Figure \ref{fig:schemaimport}.
A schema can be unpublished from a Central Server by a Local Server,
thus the schema is not available on the Central Server for further
sharing.  Users on the Central Server with an ``organizer'' role can
remove a schema from the Central Server if no document on Central
Server is depending on this schema. This can be used to clean up
non-used schemas.

\begin{figure*}[t]%7
\caption{An Example of Publishing Schemas from a Local Server}
\label{fig:schemapublish}
%\centerline{\epsffile{E:/Inderscience/LATEX-FILES/IJMSO/F7.eps}}
\end{figure*}

\begin{figure*}[t]%8
\caption{An Example of Importing Schemas to a Local Server}
\label{fig:schemaimport}
%\centerline{\epsffile{E:/Inderscience/LATEX-FILES/IJMSO/F8.eps}}
\end{figure*}

\subsection{Three Scenarios of Schemas Sharing}

Based on the use cases, there are three typical scenarios of schema sharing:

\textbf{Static Schema} A schema is fixed and will not be changed.
For example, some common standard based schemas are not likely to
change. Once a schema is authored and changed to the final status,
it can be published to the Central Server to be shared by every
Local Server (Figure \ref{fig:fixeddistributedschema}). This is the
simplest scenario, as schemas can be created once and shared
directly through the Central Server.

\textbf{Uniform Evolving Schema} A Schema can be changed and a uniform
version is shared by all Local Servers, thus data consistency is maintained
across all sites.  The schema is owned by its original creating Local Server,
and the owner can make certain changes.


\textbf{Multiform Evolving Schema} A ``seed'' schema is first
created and shared as public -- every Local Server becomes the owner
and can update the schema. The Central Server maintains a version of
the schema that conditionally merges updates from Local Servers,
thus all documents published on the Central Server will be
compatible under this schema.

\begin{figure*}[t]%9
\caption{Static Schemas Sharing} \label{fig:fixeddistributedschema}
%\centerline{\epsffile{E:/Inderscience/LATEX-FILES/IJMSO/F9.eps}}
\end{figure*}

\begin{figure*}[t]%10
\caption{Uniform Schema Sharing} \label{fig:distributedschema}
%\centerline{\epsffile{E:/Inderscience/LATEX-FILES/IJMSO/F10.eps}}
\end{figure*}

\begin{figure*}[t]%11
\caption{Multiform Schema Sharing}
\label{fig:publicdistributedschema}
%\centerline{\epsffile{E:/Inderscience/LATEX-FILES/IJMSO/F11.eps}}
\end{figure*}

Next, we will discuss how to manage schemas and their evolution for the last
two scenarios (static schema management is straightforward and ignored here.)


\section{Collaborative Management of Schemas for Consistent Data Sharing}
\label{sec:collaborativeschema}

Since schemas can be shared and used across multiple distributed
data sources, one challenging issue is how to manage schema
evolution while keeping shared documents consistent and compatible
with their schemas in a distributed environment. To solve this
problem, we first define the following favorable rules for schema
management and sharing in a distributed environment:
\begin{itemize}

\item {\em Minimal administration}. Human based manual schema
management can be difficult, especially for schemas that may be
updated by different sites.  To minimize human effort for managing
schemas across distributed sites, it would be ideal if the data
sharing system can facilitate the management of schemas, for
example, relying on an information exchange hub  based on the
Central Server.

\item {\em Data consistency}.  Schema evolution has to be backward compatible otherwise the integrity of documents will be broken.

\item {\em Control of schemas}. To prevent arbitrary update of
schemas, by default, only the owner (the user who creates the
initial schema on a Local Server) of a schema can update a schema.

\item {\em Current of Schemas}: The Central Server always has the
up-to-date version of a schema if that schema is shared by the
owner, to guarantee no outdated schemas are shared across sites.

\item {\em Sharing Maximization}: Since shared schemas on the Central
Server can be unpublished by local database users and removed from
the Central Server, to promote sharing, only the last time publisher
of this schema can unpublish a schema.

\end{itemize}

These rules will help to automate schema management in a distributed
environment, minimize conflicts of updates, maintain coherent shared
data on the Central Server, and reduce the effort from humans.  To
achieve this, we enforce the rules for schema operations on both
Local Servers and Central Servers. Schema operations on Local
Servers include create, update, delete, import, publish, and
unpublish, and schema operations on Central Servers include remove.

\subsection{Uniform Schema Management}


In this scenario, a Schema can be changed and a uniform version is
shared by all Local Servers, thus data consistency is maintained
across all sites and the Central Server.   Next we discuss how
schema management rules are enforced in each schema operation.

\textbf{Create and Delete}

A schema can be created on a Local Server if the user has the
organizer role. This Local Server will become the owner of this
schema. A schema can be  deleted  if the user has the organizer
role, and there are no documents using this schema on this Local
Server.  If the Local Server is the owner of the schema, and the
schema is never published,  deleting a schema will eliminate the
schema forever. If the schema was once published, it may still be
alive and used on other Local Servers.

\textbf{Update}

Incompatible update of a schema is the one that can lead to
inconsistency between the new schema version and existing documents.
Incompatible updates are not allowed unless the following conditions
are met:  i) only the owner of a schema can make updates to the
schema; ii) there are no existing documents created based on  this
schema on this Local Server; and iii) the schema was never
published, i.e., there will no other Local Servers using this schema
to create documents.
%A publishing status is associated with each schema.


Compatible update of a schema will not lead to inconsistency between
the new schema version and existing documents. The following
conditions are required for compatible update.   i){\em Ownership.}
The Local Server is the owner of the schema, and the user is the
organizer on this Local Server; ii){\em Field Containment.}  All the
fields in the last schema are present in the  new schema and belong
to the same category or group, and new categories and fields added.
iii){\em Type Compatibility.} All the fields in the new schema have
the same type   or a compatible type, i.e., a more general type.
iv){\em Relaxed Value Range Constraints.} No new value constraints
are permitted  for  existing fields which do not have any
constraint. Value constraints can be updated with more relaxed
ranges. Constraints on new fields are permitted. v){\em Relaxed
Controlled Values.} For field with controlled values, the extent is
enlarged with more options. vi){\em Relaxed Constraints on Number of
Instances.} For group field or file field, there can be a constraint
on the minimal and/or maximal number of instances.  No instance
number constraint is permitted on existing fields, and instance
number constraint can be updated with more relaxed range. Change of
a field's order within its sibling is not considered incompatible.


Once a schema is updated on the Local Server, it will be automatically
republished to the Central Server (if any) onto which the schema has been
published. This will ensure the Central Server always maintains up-to-date
versions of schemas.

\textbf{Publish and Unpublish}

A schema can be published to one or multiple Central Servers if the
user has a publisher role and there is no newer version of this
schema on the Central Server.  There are three scenarios of schema
publishing: i) Schemas can be manually  published from the schema
publishing interfaces; ii) The schema of a document is automatically
published when a document is published. When a document is
published, the Local Server will check the Central Server if the
schema is available or up-to-date. Otherwise the schema is
republished; iii) If a schema was published and is then updated with
compatibility, the new version schema will be automatically
republished on the Central  Server and replace the last version.
This will keep the schemas on the  Central Server up-to-date.


A schema can be unpublished by a Local Server with the following conditions:
i) the user has the ``publish'' role on the Local Server; ii) there is no
document associated with it on the Central Server, and iii) the Local Server
is the last publisher of the schema. The last condition is necessary,
otherwise if the schema is used at multiple Local Servers, every Local Server
can easily stop the publishing which can be against the sharing goal of the
last publisher.


\textbf{Import}

A Local Server can import a shared schema from the Central Server if
the user has the organizer role. When there is a new version of a
schema on the Central Server, the new version schema will be
automatically detected by a Local Server when there is a document
being published from that Local Server.

\textbf{Remove from Central Server}

Users on the Central Server with the organizer role can remove a
schema from the Central Server if no document on Central Server is
using this schema. This can be used to cleanup non-used schemas.

\textbf{Update, or create?}  As a schema  keeps on evolving, the
number of updates on a schema can be many. When it reaches certain
threshold, the latest schema may be very different from the original
schema, the data between the two schemas may be very different, and
the data consistency does not make much sense any more.  In this
case, it may be desirable to create a new schema.  SciPort provides
a functionality to view change history of a schema, and users can
create a new schema from an existing schema in the Schema Editor
tool.


\textbf{An Example of Uniform Schema Management}

A user with organizer privilege on a Local Server L1 creates a
schema S(V1) (Figure \ref{fig:distributedschema}). The user may
later find the schema not accurate, and makes changes to the schema.
Since no document has been created and the schema has never been
shared yet, the user may make arbitrary changes, including
incompatible updates. Once the user has a stable usable version of
the schema, the user begins to author documents based on this
schema, and publishes documents to the Central Server C. Schema
S(V1) will be automatically published to the Central Server. After
some time, the user may need more information for their data or
adjust existing fields, and need to update schema S. Since there are
existing documents using the schema, and the schema was also
published, the user can make only compatible update to schema S(V1).
(If compatible update is not sufficient, the user has to create a
new schema.) Once schema S(V1) is updated as S(V2), it will be
automatically propagated to the Central Server to replace the last
version S(V1). A user with organizer privilege on a Local Server L2
imports schema S(V1) after S(V1) is published on the Central Server,
and documents then are created on L2 on this schema. Later S(V2)
replaces S(V1) on the Central Server, which is detected when a
document of schema S(V1) is published to C. The user at L2 will be
prompted to synchronize the schema, and S(V1) is replaced by S(V2)
on L2.



\subsection{Distributed Multiform Schema Sharing} \label{sec:multiformschema}

Uniform schema sharing provides data compatibility through a single
uniform version of schemas across all servers, and maintains the
ownership of schemas and provides controlled schema evolution. There
can also be cases that multiple sites need to adapt certain schema
to their own needs, and a uniform schema may not be feasible. To
provide flexible schema evolution at each Local Server and support
data compatibility on the Central Server, we provide a multiform
schema sharing approach.

As shown in Figure \ref{fig:publicdistributedschema}, a ``seed'' schema {\em
S} will be first created and made as public owned, and then published onto the
Central Server. Each Local Server will be able to import this schema, and
adjust the schema for its local use ({\em S$_{L1}$}, {\em S$_{L2}$}, etc.)
Schema updates from each site will be conditionally merged on the Central
Server as {\em S$_M$}. The goal of the merging  is to keep the schema version
on the Central Server mostly relaxed, through the following merging
conditions:

\begin{itemize}
  \item If the changes are incremental structural changes, they will be merged. These include
  adding of new fields or new categories;
  \item If the constraints on the Central Server are more permissive
   than the new or updated constraints that the modification suggests,
   then the suggested changes are ignored by the Central Server;
  \item Structural removal changes such as field removal or category removal
  will be ignored and not merged.
\end{itemize}


In this way, each Local Server will maintain its local schema evolution while
the Central Server will maintain a merged schema for shared data
compatibility.




\section{Collaborative Semantically Tagging of Data}\label{sec:tag}

Recently tagging has become a popular method to enable users to add
keywords to Internet resources, thus to improve search and personal
organization \cite{marlow06tag}. In SciPort, data are hierarchically
organized based on a tree based organizational structure. In
practice, it can be very helpful that users can provide additional
classification of data, by assigning tags from a controlled
vocabulary to documents. Each document can be flexibly annotated
with one or more semantic tags, and tags themselves can be shared by
users at both Local Servers and the Central Server.

Semantically tagging adds additional semantics to documents, which
not only provides semantics based query support, but also enhances
semantic interoperability of shared data from multiple distributed
sites.


\subsection{Tagging from Controlled Vocabulary}

Free tagging is used by most Web-based tagging systems, where users
can define arbitrary tags. One issue for free tagging is that since
there is no common vocabulary among these tags, there can be
semantic mismatch between tags.

Instead, SciPort chooses to annotate data using semantic tags coming
from predefined ontology or controlled vocabulary, such as NCI
Enterprise Vocabulary Services (EVS) \cite{ncievs}. By standardizing
tags using such controlled vocabulary, the system can then provide a
controlled  set of semantic tags. Thus data can be categorized into
multiple semantic groups, which makes it possible to express queries
based on common semantics. For instance, when a user authors a
document of a study on breast cancer,  he assigns  a tag
``Stage\_II\_Breast\_Cancer'' (ID: 18077), and another tag
``Cancer\_Risk'' (ID: 7768). A problem arises that expensive lookup
through vocabulary may simply prevent users from using it.

Next we show that with a collaborative tag management system, we can
provide automatic tag lookup in a controlled vocabulary repository
by caching previously retrieved tags in the Central Server with Ajax
technology.

\begin{figure*}[t] \center%12
\caption{Collaborative Tag Management} \label{collaborativeTags}
%\centerline{\epsffile{E:/Inderscience/LATEX-FILES/IJMSO/F12.eps}}
\end{figure*}

\begin{figure*}[t] \center%13
\caption{An Example of Automatic Tag Lookup} \label{taglookup}
%\centerline{\epsffile{E:/Inderscience/LATEX-FILES/IJMSO/F13.eps}}
\end{figure*}

\subsection{Collaborative Semantic Tag Management} \label{sec:coltag}

Caching tags is to exploit locality inherent to the subset of the
vocabulary that is used within a group of researchers. Since a
collaborative research consortium often focuses on solving a single
significant problem, the vocabulary is quite smaller than the
standardized vocabulary. For instance, the NCI Thesaurus
\cite{ncithesaurus} is 77MB in size.  In SciPort, we provide a
cached vocabulary repository on the Central Server, thus
previous-retrieved tags from the standardized vocabulary is shared
among all users.   Cached vocabulary on the Central Server makes it
very efficient to search for a tag in the vocabulary: instead of
searching for a tag at a remote large vocabulary all the time,
previous used tags in the tag cache repository can be searched
first.

Figure \ref{collaborativeTags} shows the architecture of
collaborative tag management. At each Local Server, there is a tag
repository that manages all tags on this Local Server; there is a
remote NCI controlled vocabulary database from which users can
dynamically search and retrieve tags. Once a tag is defined on the
Local Server, it will  be automatically cached at the Central Server
tag repository. When a user wants to associate a tag to its data, he
can dynamically search and select the tag, through automatic tag
lookup, as discussed next.

\subsection{Semantic Annotation of Documents with Tags}
By taking advantage of the Ajax technology, we provide automatic tag
lookup while a user wants to add a tag. As the user is typing in a
keyword on a web page to search for a tag, there will be an
automatic tag lookup from three resources: the local tag repository,
the cached/shared tag repository on the Central Server, and the NCI
controlled vocabulary. Implemented through Ajax, the lookup is
performed in the background, and retrieved tags are displayed in the
order of local tags, cached tags, and remote tags.  The lookup will
dynamically load a dropdown list of tags from which the user can
pick up (Figure \ref{taglookup}). By dynamically sending
asynchronous tag queries to tag repositories, users can immediately
select a desired tag to label his data, instead of opening multiple
browser windows to do separate searches.  With this technology, we
are able to support using tags from a controlled vocabulary or
shared repository by providing a convenient interface.


\subsection{Semantically Querying and Browsing Data through Semantic Tags}

Once documents are annotated with semantic tags, it is now possible
for users to semantically browse data by clicking on a semantic tag,
or use tagged fields as additional constraints for specifying
queries.  For example, a user might want to  search all documents
tagged with ``Stage\_II\_Breast\_Cancer'', and a type of ``Breast
Cancer'' can generate a drop down list of semantic tags and the user
can quickly use the right tag to specify queries.  Semantic
annotation can also help to classify data. Since semantic tags
coming from an ontology are hierarchically organized in the
ontology,  it is possible to generate an ontology tree based view of
all documents based on annotated ontology concepts on documents.

\begin{figure*}[t] \center %14
\caption{A Screenshot of a SciPort Local Server Deployed at
University of California, Irvine} \label{sciportfrontpage}
%\centerline{\epsffile{E:/Inderscience/LATEX-FILES/IJMSO/F14.eps}}
\end{figure*}


\section{Implementation}

\subsection{Software}

SciPort is built with J2EE and XML, running on Apache Tomcat servers
and Oracle Berkeley DB XML database server (open source) or IBM DB2
XML database. The system is OS neutral and has been tested on
Windows, Linux, and MacOS. It uses standard protocols, including
XML, XSLT, XPath and XQuery, and Web Services. SciPort is a
lightweight application and can be easily deployed and customized
by users.

SciPort is a rich Web-based application, thus it is possible for
users to use it at any place and at any time. SciPort supports major
Web browsers including Internet Explorer, Firefox, Opera and Safari.
Taking advantage of Web 2.0 technologies such as Ajax, SciPort
provides rich application capabilities with smooth user experience.

A salient characteristic of SciPort is that the system is highly
adaptable. Through customizable metadata schemas, hierarchy
organization, and sharing architecture, SciPort can be easily setup
for managing biomedical research data and providing a data sharing
network.

Figure \ref{sciportfrontpage} shows the screenshot of the front page
of a Local Server deployed at University of California, Irvine,
while the Central Server is located at Siemens Corporate Research
located at Princeton, New Jersey.

\subsection{Deployment}

SciPort was initially adopted by the Network for Translational
Research on Optical Imaging (NTROI) for collecting and sharing data
across several institutions (University of California, Irvine,
University of Pennsylvania, Datmouth, Stanford and University of
California, San Francisco. It is again adopted by the second cycle
of Network for Translational Research (Washington University,
University of Texas Health Science Center, Houston, University of
Michigan, and Stanford University).  SciPort is also used for
sharing pathology images and their annotations for a large scale
prostate cancer multi-modality study at University of Pennsylvania,
the State University of New Jersey, Rutgers, and Siemens Corporate
Research.  SciPort was also adopted for supporting Siemens
Healthcare for collecting and sharing large scale biomedical
research data from university and hospital research partners.

\section{Discussion}

The design and development of SciPort has been an iterative process
and driven by numerous discussions with biomedical researchers and
users. During the process, we have learned many lessons.


\textbf{Usability.}  A critical requirement for software tools from
biomedical researchers is the usability of the system. This includes
intuitive user interfaces and workflows for data authoring, sharing
and querying, and the ease of system setup. When the initial version
of SciPort was released, the pilot users complained about multiple
clicks needed to navigate through multiple web pages, and showed
strong desires of simpler interfaces and fast response. Especially,
a single click data publishing is desired, otherwise users gradually
get frustrated to share their data.   The Web based user interfaces
then evolved from static web pages (jumping from one page to
another) based interface to Ajax based interface, and then another
Flex based interface is developed to provide even smoother user
experience. Our conclusion is that smooth user experience of data
sharing tools could much improve the interest for researchers to
share their data.

\textbf{Generalization and Customization.} Although SciPort is
currently a software platform and is not tied to specific
applications, the initial version of SciPort (Version 1.0) was
designed and developed for one single project with static data
forms. Soon we realized that even for a single project, the schemas
were dynamic, and new data forms were occasionally required. And
researchers from other projects were using totally different data
forms. Besides, the data are heterogenous, ranging from structural
data to medical images, to  files, etc. This motivated us to
redesign SciPort as a platform that could provide easy and flexible
customization. This is challenging as schema evolution is very
difficult to support in traditional relational database management
systems based on tables. The emergence of native XML database
technologies makes it possible to support customizable schemas, as
native XML DBMSs have a significant advantage on supporting schema
evolution. Our data models, queries and data management are fully
based on XML.

\textbf{Integration with Other Systems.} In the past, we also worked
on extending SciPort so data managed in SciPort could be integrated
into other infrastructures such as caBIG. We developed a
SciPort-NBIA \cite{nbia} bridge to support such integration.


\section{Related Work}

Computer supported collaborative work has been increasingly used to
support research collaborations. A review of collaborative systems
is presented in \cite{bafoutsou2002281}, and a review of taxonomy of
collaboratory types is discussed \cite{zimmerman07taxonomy}. SciPort
belongs to the Community Data Systems based on this classification.
Lee et al review collaboratory concepts relevant for collaborative
biomedical research \cite{lee09} and analyze the major challenges.
In \cite{myneni10col}, a collaborative information management system
is discussed.  A context-based sharing system is proposed in
\cite{chin04context} to support more data-centric collaboration than
tools oriented ones, and this motivates our metadata oriented
approach for data modeling and sharing in SciPort.


With the increasing collaboration of scientific research,
collaborative cyberinfrastructures have been researched and
developed.  Grid-based systems(such as caBIG{\small \textregistered}
--cancer Biomedical Informatics Grid \cite{cabig}, Biomedical
Informatics Research Network (BIRN) \cite{birn}) provide
infrastructures to integrate existing computing and data resources.
They rely on a top down common data structures. The iPlant
Collaborative (iPC) \cite{pscic}  is a cyberinfrastructure project
recently funded by US NSF. iPlant has a more focus on the human side
of the infrastructure. myGrid \cite{mygrid} is a suite of tools for
e-Science, with a focus on workflows.   A review on
cyberinfrastructure systems for the biological sciences  is
discussed in \cite{stein08}.  Grid based systems are more used on
sharing computing and storage resources, P2P is more used on sharing
data \cite{foster03p2p2grid}.  MIRC \cite{mirc} is an example of P2P
based data system for authoring and sharing radiology  teaching
files.

While Grid based systems are more used on sharing computing and storage
resources, P2P is more used on sharing data \cite{foster03p2p2grid}.  MIRC
\cite{mirc} is a popular pure-P2P based system for authoring and sharing
teaching files.

A publish and subscribe architecture for distributed metadata management is
discussed in \cite{keidl02metadata}, which focuses on the synchronization
problems. In \cite{taylor06reconcile}, an approach of bottom-up collaborative
data sharing is proposed, where each group independently manages and extends
their data, and  the groups compare and reconcile their changes eventually
while tolerating disagreement. Our approach takes an approach in between the
bottom-up and top-down approaches, where each group manages their data, but
also achieves as much agreement on schemas as possible through controlled
schema evolution.


In \cite{rader08tag}, influence on tag choices is analyzed from
del.icio.us. Our approach focuses on semantic tagging from a
controlled vocabulary instead of free tags.  In \cite{pike07},
different approaches for representing scientific knowledge is
discussed.


Extensive work has been done in data integration and schema
integration \cite{halevy06integration,doan05semantic}. In
\cite{beynon97}, a collaborative schema integration system is
discussed for database design. Our system takes a proactive approach
where schema and data consistency is enforced during data authoring
and schema authoring.



\section{Conclusion}

Contemporary biomedical research is moving towards
multi-disciplinary, multi-institutional collaboration. These lead to
strong demand for tools and systems to share biomedical data. This
drives the development of SciPort -- a Web-based data sharing
platform for collaborative biomedical research. To support
meaningful abstraction of data, SciPort provides a generic metadata
model for users to conveniently define their own metadata schemas.
SciPort provides an innovative lightweight hybrid data sharing
architecture which combines the benefits of the centralized approach
and the distributed approach. Investigators are able to flexibly
manage their data and schema sharing, and quickly build data sharing
networks. To enable data consistency for data sharing, SciPort
provides comprehensive collaborative distributed schema management.
Through semantic tagging, SciPort further enhances semantic
interoperability of shared data from distributed sites.

%\section*{Acknowledgement}
%The project is funded in part by the National Institutes of Health, under Grant No. 5R01CA136535.


%%%%%%%%%%%%%%%%%%%%%%%%%%%%%%%%%%%%%%%%%%%%%%%%%%%%%%%%%%%%%%%%%%%%%%%%%%%%%%%%%%%%%

%\bibliography{ijmso}
%\bibliographystyle{unsrt}
%\bibliographystyle{alpha}

\begin{thebibliography}{10}

\bibitem[\protect\citeauthoryear{Myneni and Patel}{2010}]{myneni10col}
Myneni, S. and  Patel, V.L. (2010) 'Organization of biomedical data
for collaborative scientific research: a research information
management system', {\it International Journal of Information
Management}, Vol. 30, No. 3, pp.256--264

\bibitem[\protect\citeauthoryear{Bekelman et al.}{2003}]{bekelman03}
Bekelman, J.E., Li, Y. and Gross, C.P. (2003) 'Scope and impact of
financial conflicts of interest in biomedical research: a systematic
review', {\it JAMA}, Vol. 289, No. 19, pp.454--65

\bibitem[\protect\citeauthoryear{Stein}{2008}]{stein08}
Stein, L.D. {2008} 'Towards a cyberinfrastructure for the biological
sciences: progress, visions and challenges', {\it Nature Reviews
Genetics}, Vol. 9, No. 9, pp.678--688.

\bibitem[\protect\citeauthoryear{NIH Statement on Sharing Scientific Research
Data}{http://grants.nih.gov/grants/policy/data\_sharing/}]{nih03datasharing}
NIH Statement on Sharing Scientific Research Data,
http://grants.nih.gov/grants/policy/data\_sharing/

\bibitem[\protect\citeauthoryear{Policy for sharing of data obtained in NIH supported or conducted
genome-wide association studies}{2008}]{nih08policy} Policy for
sharing of data obtained in NIH supported or conducted genome-wide
association studies,
http://grants.nih.gov/grants/guide/notice-files/not-od-07-088.html

\bibitem[\protect\citeauthoryear{Network for Translational Research (NTR): Optical Imaging in
Multimodality
Platforms}{http://imaging.cancer.gov/programsandresources/speci\break
alizedinitiatives/ntroi}]{ntr} Network for Translational Research
(NTR): Optical Imaging in Multimodality Platforms,
http://imaging.cancer.gov/programsandresources/specializedinitiatives/ntroi

\bibitem[\protect\citeauthoryear{Piwowar~et~al.}{2008}]{piwowar08culture}
Piwowar, H., Becich, M., Bilofsky, H. and Crowley, R. (2008) `PLoS
medicine', {\it Sept, No. 9, Towards a Data Sharing Culture:
Recommendations for Leadership from Academic Health Centers},
Vol.~5.

\bibitem[\protect\citeauthoryear{Birnholtz et al.}{2003}]{birnholtz03data}
Birnholtz, J.P. and Bietz, M.J. (2003) `Data at work: supporting
sharing in science and engineering', {\it GROUP}, pp.339--348.

\bibitem[\protect\citeauthoryear{Data Sharing \& Intellectual Capital (DSIC) Workspace}
{https://cabig.nci.nih.gov/working\_groups/DSIC\_SLWG}]{dsic} Data
Sharing \& Intellectual Capital (DSIC) Workspace,
https://cabig.nci.nih.gov/working\_groups/DSIC\_SLWG

\bibitem[\protect\citeauthoryear{Getting Connected with
caBIG}{https://cabig.nci.nih.gov/getting\_connected/}]{cabigconnected}
Getting Connected with caBIG, https://cabig.nci.nih.gov/\break
getting\_connected/

\bibitem[\protect\citeauthoryear{Digital Imaging and Communications in Medicine
(DICOM)}{http://medical.nema.org/}]{dicom} Digital Imaging and
Communications in Medicine (DICOM), http://medical.nema.org/

\bibitem[\protect\citeauthoryear{The Cancer Genome Atlas (TCGA) Data Portal}{http://cancergenome.nih.gov/dataportal}]{tcga}
The Cancer Genome Atlas (TCGA) Data Portal,
http://cancergenome.nih.gov/dataportal

\bibitem[\protect\citeauthoryear{National Biomedical Imaging
Archive}{https://cabig.nci.nih.gov/tools/NCIA}]{nbia} National
Biomedical Imaging Archive, https://cabig.nci.nih.gov/tools/NCIA

\bibitem[\protect\citeauthoryear{caBIG: cancer Biomedical Informatics Grid}{http://caBIG.nci.nih.gov/}]{cabig}
caBIG: cancer Biomedical Informatics Grid, http://caBIG.nci.nih.gov/

\bibitem[\protect\citeauthoryear{Biomedical Informatics Research Network}{
http://www.nbirn.net/}]{birn}
Biomedical Informatics Research Network, http://www.nbirn.net/

\bibitem[\protect\citeauthoryear{Cyberinfrastructure for the Biological Sciences: Plant Science
Cyberinfrastructure Collaborative
(PSCIC)}{http://www.nsf.gov/pubs/2006/nsf06594/nsf06594.htm}]{pscic}
Cyberinfrastructure for the Biological Sciences: Plant Science
Cyberinfrastructure Collaborative (PSCIC),
http://www.nsf.gov/pubs/2006/nsf06594/nsf06594.htm

\bibitem[\protect\citeauthoryear{MIRC}{http://mirc.rsna.org}]{mirc}
MIRC, http://mirc.rsna.org

\bibitem[\protect\citeauthoryear{Foster and
Iamnitchi}{2003}]{foster03p2p2grid} Foster, I. and Iamnitchi, A.
(2003) `On death, taxes, and the convergence of peer-to-peer and
grid computing', {\it IPTPS'03}.

\bibitem[\protect\citeauthoryear{Keidl et al.}{2002}]{keidl02metadata}
Keidl, M., Kreutz, A., Kemper, A. and Kossmann, D. (2002) `A publish
{\&} subscribe architecture for distributed metadata management',
{\it ICDE}.

\bibitem[\protect\citeauthoryear{Taylor and Ives}{2006}]{taylor06reconcile}
Taylor, N.E. and Ives, Z.G. (2006) `Reconciling while tolerating
disagreement in collaborative data sharing', {\it SIGMOD}.

\bibitem[\protect\citeauthoryear{Rader and Wash}{2008}]{rader08tag}
Rader, E.J. and Wash, R. (2008) {\it CSCW}, pp.239-248, Influences
on tag choices in del.icio.us.

\bibitem[\protect\citeauthoryear{Halevy et~al.}{2006}]{halevy06integration}
Halevy, A., Rajaraman, A. and Ordille, J. (2006) {\it VLDB, Data
integration: the teenage years},
http://portal.acm.org/citation.cfm?id=1182635.1164130

\bibitem[\protect\citeauthoryear{Doan and Halevy}{2005}]{doan05semantic}
Doan, A. and Halevy, A.Y. (2005) Semantic Integration Research in
the Database Community: A Brief Survey, {\it AI Magazine}, Vol. 26,
No. 1, pp.83--94.

\bibitem[\protect\citeauthoryear{Beynon-Davies et~al.}{1997}]{beynon97}
Beynon-Davies, P., Bonde, L., McPhee, D. and Jones, C.B. (1997) `A
collaborative schema integration system', {\it Comput. Supported
Coop. Work}, Vol. 6, No. 1, Norwell, MA, USA, pp.1--18, issn =
0925-9724.

\bibitem[\protect\citeauthoryear{Wang and Vergara-Niedermayr}{2008}]{wang08com}
Wang, F. and Vergara-Niedermayr, C. (2008) `Collaboratively Sharing
Scientific Data', {\it CollaborateCom}, pp.805--823.

\bibitem[\protect\citeauthoryear{Chin and Lansing}{2004}]{chin04context} Chin Jr., G. and Lansing,
C.S. (2008) `Capturing and Supporting Contexts for Scientific Data
Sharing via the Biological Sciences Collaboratory', {\it CSCW}, ISBN
1-58113-810-5.

\bibitem[\protect\citeauthoryear{W3C XML Query (XQuery)}{http://www.w3.org/XML/Query}]{xquery}
W3C XML Query (XQuery), http://www.w3.org/XML/Query

\bibitem[\protect\citeauthoryear{Wang et al.}{2009}]{wang09security}
Wang, F., Hussels, P. and Liu, P. (2009) `Securely and flexibly
sharing a biomedical data management system', {\it SPIE}.

\bibitem[\protect\citeauthoryear{Marlow et al.}{2006}]{marlow06tag}
Marlow, C., Naaman, M., Boyd, D. and Davis, M. (2006) `Position
Paper, Tagging, Taxonomy, Flickr, Article, ToRead', {\it
Collaborative Web Tagging Workshop}.

\bibitem[\protect\citeauthoryear{NCI Enterprise Vocabulary Services (EVS)}{http://ncicb.nci.nih.gov/NCICB /infrastructure /cacore\_overview
/vocabulary}]{ncievs} NCI Enterprise Vocabulary Services (EVS),
http://ncicb.nci.nih.gov/NCICB /infrastructure /cacore\_overview
/vocabulary

\bibitem[\protect\citeauthoryear{NCI Enterprise Vocabulary Services (EVS)}{http://ncicb.nci.nih.gov/NCICB /infrastructure /cacore\_overview
/vocabulary}]{ncithesaurus} NCI Enterprise Vocabulary Services
(EVS), http://ncicb.nci.nih.gov/NCICB /infrastructure
/cacore\_overview /vocabulary

\bibitem[\protect\citeauthoryear{Bafoutsou and Mentzas}{2002}]{bafoutsou2002281}
Bafoutsou, G. and Mentzas, G. (2002) `Review and functional
classification of collaborative systems', {\it International Journal
of Information Management}, Vol. 22, No. 4, pp.281--305.

\bibitem[\protect\citeauthoryear{Bos et al.}{2007}]{zimmerman07taxonomy}
Bos, N., Zimmerman, A., Olson, J., Yew, J., Yerkie, J. and Dahl, E.
et al. (2007) `From shared databases to communities of practice: A
taxonomy of collaboratories', {\it Journal of Computer-Mediated
Communication},  Vol. 12, No. 2.

\bibitem[\protect\citeauthoryear{Myneni and Patel}{2010}]{myneni10col}
Myneni, S. and  Patel, V.L. (2010) `Organization of biomedical data
for collaborative scientific research: A research information
management system', {\it International Journal of Information
Management}, Vol. 30, No. 3, June, pp.256--264.

\bibitem[\protect\citeauthoryear{myGrid}{http://www.mygrid.org.uk/}]{misc{mygrid}}
myGrid, http://www.mygrid.org.uk/

\bibitem[\protect\citeauthoryear{Pike and Gahegan}{2007}]{pike07}
Pike, W. and Gahegan, M. (2007) `Beyond ontologies: Toward situated
representations of scientific knowledge', {\it International Journal
of Human-Computer Studies}, Vol. 65, No. 7, pp.674--688.
\end{thebibliography}
\end{document}
