While Functional Requirements establish \textit{what} is computed by an application, Non-Functional Requirements (NFRs) are concerned with \textit{how} the task is preformed.  
NFRs include aspects such as performance, authentication and quality constraints.
These requirements are usually specified by conditions, called Non-functional Properties.
Non-functional properties are also referred to as constraints, quality attributes, quality goals, quality of service requirements and non-behavioral requirements~\cite{Chung91,MylopoulosBook99,Chung2009}.

Most research  efforts on NFRs focus on the evaluation of compliance by the software system as a whole. %~\cite{BBKL78,FePf96,KiDa96,Lyu96,MuIO90}. 
In the case of service-based applications, non-functional requirements are related to the application itself as well as to its component services. 

In~\cite{Babamir2010,Yeom2006} non-functional properties of web services are classified according to three points of view:   
\textit{service level}, \textit{system level} and \textit{business level}.
%In~\cite{Babamir2010} NFRs are denoted as \textit{quality constraints}, which are expressed as logic formulae.
%In~\cite{Yeom2006} authors classify NFRs into \textit{category}, \textit{sub-category} and \textit{property}. Categories include \textit{business}, \textit{service} and \textit{system}.
%Possible \textit{sub-categories} are \textit{security}, \textit{value} or  \textit{interoperability}. The work also defines a \textit{web service quality model}, which considers non-functional properties. 
In~\cite{XiaoCZBOLH08} the authors use the terms  
\textit{non-functional attributes}, \textit{composition mo\-del}  \textit{entity} and \textit{mo\-del entity}  to classify different concepts related to NFRs.
The notion of non-functional attribute is used to describe NFRs of the abstract process model. 
In the lower level, the composition is annotated with non-functional attributes.

D'Ambrogio~\cite{DAmbrogio06} uses the term \textit{quality category} to group similar \textit{quality characteristics}. 
\textit{Quality dimensions} are used to quantify an individual characteristic.
For instance, the quality category \textit{performance} groups characteristics such as
\textit{latency} and \textit{throughput}. 
The development process is based on MDA and the authors also present a WSDL extension for describing the QoS of web services. A catalog of \textit{QoS characteristics} is provided for the web service domain, including properties such as \textit{availability}, \textit{reliability} and \textit{access control}. 

 
Schmeling et al.~\cite{SchmelingCM11} present an approach and a toolkit for specifying and implementing web service compositions with support to several NFRs. 
Their approach defines abstraction levels, where the terms \textit{Non-Functional Concerns}, \textit{Non-Functional Actions} and \textit{Non-Functional Attributes} are used at each level (in decreasing order of abstraction).
A non-func\-tion\-al concern is a general term used to describe non-functional requirements, such as  \textit{security}, \textit{reliability} or \textit{transactional behavior}. 
Each  concern is refined into a set of non-func\-tion\-al  at\-tri\-butes, where  
a non-func\-tion\-al at\-tri\-bute represents some behavior, to be refined into non-functional actions. 
For instance, \textit{encryption} is a non-func\-tion\-al  action  which provides the implementation of the non-functional attribute \textit{confidentiality}, which is part of the \textit{security} non-functional concern. 
%Non-functional actions related to a common concern are grouped into \textit{non-functional activities}. 

Pastrana et al.~\cite{PastranaPK11} use a traditional design-by-contract approach~\cite{Meyer97}.
The authors use the term \textit{contract} to describe non-functional re\-quire\-ments.  
Contracts may present pre-con\-di\-tions, post-conditions and invariants. 
Also, a con\-tract may define \textit{assertions} associated with \textit{qual\-i\-ty properties}. 
Each service may have as many associated \textit{contracts} as needed.

Chollet et al.~\cite{CholletL09} associate (non-func\-tion\-al) \textit{quality properties} to 
(functional) activities. 
They present a security meta-model that takes into account web service compositions. 
In this work the non-functional requirements considered are  \textit{authentication}, \textit{integrity} and \textit{confidentiality}. 
Each NFR is associated with a service activity.


Ceri et al.\cite{CeriDMF07} use the notions of \textit{policy}, \textit{rule}, \textit{condition} and \textit{action model} to specify NFRs.
Agarwal et al.~\cite{AgarwalLS09} associate \textit{service policies} to services. 
Each service may also have \textit{properties}, such as \textit{security} and \textit{reliability}. 
Ovaska et al.~\cite{OvaskaEHPA10} use the terms \textit{quality attribute}, \textit{category}, \textit{conceptual layer} and \textit{importance} to organize and classify NFRs.
Other authors do not define specific terms to refer to NFRs; they use terms such as \textit{attribute}~\cite{ZhangPSP05,BasinDL06,JeongCL09}, 
\textit{property}~\cite{Fabra2011}, 
\textit{factor}~\cite{MohantyRP10,GutierrezRF10}, 
\textit{characteristic}~\cite{DiamadopoulouMPS08}, 
\textit{qual\-i\-ty level} \cite{ModicaTV09}, and
\textit{value}~\cite{ThissenW06,BasinDL06}.


Despite of the different notations found in the literature for classifying NFRs, some non-functional requirements are frequently considered, such as \textit{security}, \textit{performance}, \textit{reliability}, \textit{usability}, and \textit{availability}.
However, distinct hierarchies and models are proposed for NFRs,  according to different points of view.
We have identified a number of approaches~\cite{DAmbrogio06,CholletL09,SchmelingCM11,BasinDL06,Fabra2011,OvaskaEHPA10} that use MDD (Model Driven Development) for designing and developing applications. 

%Fabra \textit{et al.}~\cite{Fabra2011} also describes the importance of  MDD for service-oriented applications. 
%The work  presents a development methodology, although this methodology is not centered on NFRs.
The authors in~\cite{ThissenW06,ZhangPSP05} use formal methods to define a service-based development process that takes NFRs into account. 
In~\cite{AgarwalLS09,PastranaPK11} ontologies are used to define and model NFRs, 
as  in~\cite{XiaoCZBOLH08,GutierrezRF10} Business Process Modeling is used for
system specification, including NFRs. 
In the method defined in~\cite{XiaoCZBOLH08}, each task and data item of the application can be 
annotated with functional as well as non-functional attributes (NFAs). 
Functional and non-functional at\-tri\-butes are independently defined.
They are attached to specific tasks later in the development of the application. 
NFAs for data considers \textit{value} and \textit{range}, whereas NFAs for tasks include \textit{cost}, \textit{time}, \textit{resources} and \textit{expressions}.

The proposal in~\cite{ThissenW06} presents steps for  selecting services 
by taking QoS information into account. The proposed steps are: 
\textit{(i)} identification of relevant QoS information; 
\textit{(ii)} identification of basic composition patterns and 
QoS aggregation rules for these patterns; and 
\textit{(iii)} definition of a selection mechanism of services. 
The QoS properties considered are \textit{performance}, \textit{cost}, \textit{reliability} and
\textit{availability}. 
  
Karunamurthy et al.~\cite{Karunamurthy2012787} use the term \textit{non-function parameters} to define NFRs, such as \textit{cost}, \textit{response time}, \textit{availability}, \textit{security}, \textit{reliability} and \textit{reputation}.  
The \textit{Non-Func\-tion\-al Specification Language} (NFSL) is proposed as a domain specific language (DSL) to express \textit{non-function parameters}.

Liu et al.~\cite{Liu20121080} use the term \textit{QoS parameter} to describe non-functional requirements such as \textit{cost}, \textit{execution duration}, \textit{accuracy}, \textit{security}, \textit{integrity}, \textit{availability} and \textit{reliability}.  
In the same way, Tran et al.~\cite{Tran2012531} use the term \textit{QoS policies} to classify similar non-functional requirements.

Li et al.~\cite{Li2013} associate \textit{dimensions} to  \textit{QoS parameters} to classify NFRs.  
For instance, the \textit{time} dimension is associated to the \textit{execution time} and \textit{communication time} parameters; the \textit{spatial} dimension is associated to the \textit{storage capacity} and \textit{message length} parameters; the \textit{reliability} dimension is associated to the \textit{availability} and \textit{reliability} parameters and the \textit{cost} dimension is associated to the \textit{service cost} parameter.
Rumpel et al.~\cite{Rumpel2012}  associate \textit{quality requirements} to  \textit{quality properties}. Quality requirements are intended to be specified as constraints. 

\bigskip
Most works agree on distinguishing three points of view, namely the point of view of the organization (\textit{Business} view), of the individual service providers (\textit{Service} view) and of the composition designer (\textit{System} view).
The Business view is concerned with the business logic (\textit{i.e.}, an abstract level of tasks, defined by the guidelines and constraints imposed by the organization).
Service and System views guides the implementation of the software solution: The Service level is related to the building blocks of the application.
It may use web services provided by third party sources.
The System level deals with the coordination of services, to implement the business logic.
Our proposal, shown in the next section, follows these ideas.