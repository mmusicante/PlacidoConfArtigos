
%Service oriented computing is at the origin of an evolution in the field of software development.
%Service oriented methods advocates for the construction of software systems formed by the composition of heterogeneous, loosely coupled modules.
%These modules (or services) communicate in order to achieve a common purpose.


%(Moreover, IT systems need to evolve according to the business needs.)
%Thus, organizations are seeking for mechanisms to bridge the gap between the actually developed systems and their business needs~\cite{bell}.


In Service-Oriented Computing~\cite{Papazoglou2007}, pre-ex\-isting services are
combined to build an application business logic.
The selection of services is usually guided by the \textit{functional} requirements of the application being developed~\cite{2,decastro1,PapazoglouH06}\footnote{Functional properties of a computer system are characterized by the effect produced by the system when given a defined input.}.
An important challenge of service-o\-rien\-ted development is  to ensure the alignment between the functional requirements imposed by the business logic and the functions actually being developed.

Functional properties are not the only  aspect in the software development process.
Non-functional properties, such as data privacy, exception handling, atomicity  and, data persistence, need to be addressed  to fit in the application.
%Adding non-functional properties and respecting services constraints while composing services is a complex task that implies programming  protocols for instance authentication protocols to call a service, and atomicity (exception handling and recovery) for ensuring a true synchronization of the results produced by the service methods calls.

Even if service-oriented computing benefits from reuse, this reuse is usually guided only by functional requirements.
%
%Furthermore,  often they are  not fully considered in the specification process of existing services' oriented development methods. Rather,
%They   are either ensured by the underlying execution platform, or they are programmed through ad-hoc protocols once the application has been programmed.
Ideally, non-functional requirements should be considered in every phase of the software development.
Yet, they are partially or rarely methodologically derived from the specification, being usually added once the code has been im\-ple\-men\-ted.
In consequence, the development process does not fully preserve the compliance and re\-use expectations provided by the service oriented computing methods.


%Yet, non-functional properties of  services, often expressed as requirements and constraints in general purpose methodologies, are not usually considered from the early phases of the (service) software process.
%Most methods integrate them only after the application has been implemented.
%This leads to service based applications that are partialy specified and, thereby, partialy compliant with the requirements of the application.

% song title disseminated in the walls of the user's Facebook and Twitter accounts.



%The adoption of non-functional specifications from the early states of development can help the developer to produce applications that  can deal with the application context.

The literature stresses the need for meth\-od\-ol\-o\-gies and techniques for service oriented analysis and design
%since they are the cornerstone in the development of meaningful service based applications
~\cite{Papazoglou2007}.
Existing approaches maintain that the convergence of model-driven soft\-ware development, service orientation,   and  busi\-ness processes improvement are key for developing accurate  software~\cite{watson}.
Model Driven Development (MDD)  for software systems is mainly characterized by the use of models as a product~\cite{Selic03}.
These models are successively refined from abstract specifications into actual computer programs. A recent literature review study concludes there are a very scarce attention to non-functional requirements in the development of service-oriented applications using model-driven methodologies~\cite{Ameller201542}.

Our work proposes, $\pi$-\textit{Service Oriented Development Method} (\pisodm) to support non-functional aspects of service-oriented applications, taking into account both functional and non-functional requirements early in software development.
Our proposal is aligned with the MDD guidelines and proposes models, practices and techniques for the development of service-based applications.  The technique proposes  the use of \textit{models} to specify a software system at different levels of abstraction.
Models are organized according to the guidelines of the Model Driven Archi\-tec\-ture (MDA)~\cite{miller} approach.
The goals of  \pisodm are to:
%SOD-M does not provide support for the specification of non-functional requirements, such as
%security, reliability, and efficiency.
\begin{trivlist}
\item \textit{(i)} Improve the development process by providing an abstract view of the application and aiding to ensure the conformance to its specification.
\item \textit{(ii)} Reduce the programming effort through the semi-automatic generation of  models for the application, in order to produce concrete implementations from higher-level models.
\end{trivlist}

The applicability of our proposal was tested with an industrial case study concerning risk assessment for financial companies as implemented by the ORCA System\footnote{The ORCA System is a trademark of GCP Global (www.gcpglobal.com).}. In this work we show the application of our method \pisodm to develop a service based application called \FlyingPig that provides risk assessment as a service.The models presented were generated as a result of interacting with software developers at GCP Global and trying to attend to all the non-functional requirements present in the application domain.

This paper is organized as follows.
Sec\-tion \ref{sec:relworks} summarizes the general principles of existing works for addressing NFP and associating them to service compositions.
Sections~\ref{sec:motivation} and~\ref{sec:mmrules} introduce respectively the meta-models of the \pisodm and transformation rules.
% Section~\ref{sec:implementation} describes the \pisodm environment that provides tools for building service based applications according to the \pisodm methodology.
Section~\ref{sec:flyingPig} describes the \FlyingPig case study that we develop for validating our method and discusses lessons learned.
Finally, Section~\ref{sec:conclusions} concludes the paper and gives research perspectives.





