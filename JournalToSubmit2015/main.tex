\documentclass{ws-ijseke}
\usepackage{xspace}



\newcommand{\pisodm}[0]{$\pi$SOD-M\xspace}
\def\FlyingPig{\textsl{FlyingPig}\xspace}

\newcounter{numberInTrivlist}

\newenvironment{numtrivlist}{\begin{list}{\rm \arabic{numberInTrivlist})}
                                         {\usecounter{numberInTrivlist}
                                          \setlength{\leftmargin}{0pt}
                                          \setlength{\rightmargin}{0pt}
                                          \setlength{\itemindent}{12pt}
                                          \setlength{\listparindent}{0pt}}}
                            {\end{list}}

\newenvironment{itemizedTrivlist}{\begin{list}{\rm ~\hspace{2mm} $\bullet$\ }
                                         {\setlength{\leftmargin}{0pt}
                                          \setlength{\rightmargin}{0pt}
                                          \setlength{\itemindent}{12pt}
                                          \setlength{\listparindent}{0pt}}}
                            {\end{list}}

\begin{document}

\markboth{K.Belhajjame \textit{et al.}}
{\pisodm: Building Service-Oriented Applications in the Presence of Non-Functional Requirements}

%%%%%%%%%%%%%%%%%%%%% Publisher's Area please ignore %%%%%%%%%%%%%%%
%
\catchline{}{}{}{}{}
%
%%%%%%%%%%%%%%%%%%%%%%%%%%%%%%%%%%%%%%%%%%%%%%%%%%%%%%%%%%%%%%%%%%%%

\title{\pisodm: Building Service-Oriented Applications in the Presence of Non-Functional Requirements}

\author{Khalid Belhajjame}
\address{Universit\'e de Paris - Dauphine -- Paris, France\\
\email{kbelhajj@googlemail.com}
}

\author{Valeria~de~Castro} 
\address{Universidad Rey Juan Carlos -- M\'{o}stoles, Spain\\
\email{valeria.decastro@urjc.es}
}

\author{Umberto~Souza~da~Costa} 
\address{Universidade Federal do Rio Grande do Norte -- Natal, Brazil\\
\email{umberto@dimap.ufrn.br}
}

\author{Javier~A.~Espinosa-Oviedo} 
\address{Universidad de las Am\'ericas-Puebla, LAFMIA -- Cholula, Mexico\\
\email{javiera.espinosa@gmail.com}
}

\author{Martin~A.~Musicante} 
\address{Universidade Federal do Rio Grande do Norte -- Natal, Brazil\\
\email{mam@dimap.ufrn.br}
}

\author{Pl\'acido~A.~Souza~Neto} 
\address{Instituto Federal de Educa\c{c}\~{a}o, Ci\^{e}ncia e Tecnologia do Rio Grande do Norte -- Natal, Brazil\\
\email{placido.neto@ifrn.edu.br}
}

\author{Genoveva~Vargas-Solar} 
\address{CNRS, LIG-LAFMIA, Saint Martin d'H\`eres, France\\
\email{genoveva.vargas@imag.fr}
}

\author{Jos\'e-Luis~Zechinelli-Martini}
\address{Universidad de las Am\'ericas-Puebla, LAFMIA -- Cholula, Mexico\\
\email{joseluis.zechinelli@udlap.mx}
}

\maketitle

\begin{history}
\received{(Day Month Year)}
\revised{(Day Month Year)}
\accepted{(Day Month Year)}
%\comby{(xxxxxxxxxx)}
\end{history}






\begin{abstract}
Specifying non-functional requirements (NFRs) is a complex task, being usually dealt with on the later phases of the software process.
The late inclusion of NFRs in the development may compromise the quality of the deployed application.
This paper presents \pisodm, a method and associated tools that
\textit{(i)}  allows the early specification of non-functional requirements in a principled way: users are abstracted away from low level details;
\textit{(ii)} embraces the MDA philosophy, generating models (code) whenever possible.
The pro\-po\-sed solution has been utilized in the context of an industrial and real case study.
\end{abstract}

\keywords{MDA \and Non-Functional Requirements \and Service-based software process}


%% main text
%*********************************************************************************************************
\section{Introduction}
\label{sec:intro}

Model Driven Development (MDD) is a top-down approach for the development of software systems. 
The main ideas of MDD were originally proposed by the Object Management Group (OMG)~\cite{mda}, as a set of guidelines for the structuring of specifications.
The technique advocates for the use of \textit{models} to specify a software system at different levels of abstraction (called \textit{viewpoints}):

\begin{trivlist}
\item \textbf{Computation Independent Models (CIM):} This level focusses on the
environment of the system, as well as on its business and requirement specifications. 
This viewpoint represents the software system at its highest level of abstraction. 
At this moment of the development, the structure and system processing details are still unknown or undetermined. 
 
\item \textbf{Platform Independent Models (PIM):} This level focusses on the system functionality, hiding the details of any particular platform. 
The specification defines those parts of the system that do not change from one platform to another. 

\item \textbf{Platform Specific Models (PSM):} This level focusses on the functionality, in the context of a particular implementation platform.
Models at this level combine the platform-independent view with the specific aspects of the platform to implement the system.  
\end{trivlist}

Besides the notion of model at each level of abstraction, MDD requires the use of \textit{model transformations} within and between levels.
Intra-level transformations are used to provide a unified representation of concepts of a given level.
Inter-level transformation implement a refinement process between levels.
Transformations may be automatic or semi-automatic.

MDD techniques has been successfully used for the development of hardware and software systems~\cite{MDDvariosAqui}. 
In particular, we are interested in the application of MDD to the design and implementation of web service applications.

Service oriented computing~\cite{Papazoglou2007} is at the origin of an evolution in the field of software development~\cite{??}. 
An important challenge of service oriented development is  to ensure the alignment between the requirements imposed by the business logic and the IT systems actually developed.
(Moreover, IT systems need to evolve according to the business needs.)
Thus, organizations are seeking for mechanisms to bridge the gap between the systems developed and business needs~\cite{bell}. 
The literature stresses the need for methodologies and techniques for service oriented analysis and design, claiming that they are the cornerstone  in the development of meaningful service based applications~\cite{5}.  
In this context, we argue that the convergence of model-driven software development, service orientation and better techniques for documenting and improving business processes are key to make real the idea of rapid, accurate development of software that serves, rather than dictates the needs of its users~\cite{watson}. 

In Service-Oriented Computing, pre-existing services are
combined to produce applications and provide the business logic. 
The selection of services is usually guided by the \textit{functional} requirements of the application being developed~\cite{1,2,decastro1,PapazoglouH06}. 
(Functional properties of a computer system are characterized by the effect produced by the system when given a defined input.)
Functional properties are not the only crucial aspect in the software development process. 
Other properties need to be addressed to fit in the application with its context.
These other aspects are called Non-Functional Properties.

Non-functional aspects of the services, often expressed as requirements and constraints in general purpose methodologies, are not usually considered from the beginning of the (service) software process.
Most methods consider them only after the application has been implemented, in order to ensure some level of reliability (e.g., data privacy, exception handling, atomicity, data persistence). 
This leads to service based applications that are partly specified and, thereby, partly compliant with the requirements of the application.
Ideally, non-functional requirements should be considered along with all the stages of the software development. 
The adoption of non-functional specifications from the early states of development
can help the developer to produce applications that are capable of dealing with
the application context.

In this work, we are interested in the extension of the \textit{Service Oriented Development Method} (SOD-M)~\cite{decastro1}, to support non-functional aspects, from the early stages of software development.
SOD-M is aligned with the MDD directives and proposes models, practices and techniques for the development of service-based applications.
SOD-M does not provide support for the specification of non-functional requirements, such as
security, reliability, and efficiency. 

The main goals of our work are:
\begin{trivlist}
\item \textit{(i)} To define a NFR model including a set of concepts need for the modeling of NFR in service-oriented applications.
\item \textit{(ii)} To propose a methodology for supporting the construction of service-oriented applications, taking into account both functional and non-functional requirements;
\item \textit{(iii)} To improve the construction process by providing an abstract view of the application and ensure the conformance to its specification;
\item \textit{(iv)} To reduce the programming effort through the semi-automatic generation of  models for the application, to produce concrete implementations from high abstraction models;
\end{trivlist}

The rest of the paper is organized as follows. 
Section\dots


\begin{center}
\textsc{\underline{Version from CAISE submission}}
\end{center}


Service oriented computing is at the origin of an evolution in the field of software development. 
An important challenge of service oriented development is  to ensure the alignment between IT systems and the business logic.
%dealing thereby with the promise that IT systems can  evolve  according to the business needs. 
Thus, organizations are  seeking for mechanisms to deal with the gap between the systems developed and business needs \cite{bell}. The literature stresses the need for methodologies and techniques for service oriented analysis and design, claiming that they are the cornerstone  in the development of meaningful services' based applications \cite{5}.  In this context, some authors argue that the convergence of model-driven software development, service orientation and better techniques for documenting and improving business processes are the key to make real the idea of rapid, accurate development of software that serves, rather than dictates, software users' goals \cite{watson}. 

Service oriented development methodologies providing models, best practices, and reference architectures to build services' based applications mainly address  functional aspects \cite{1,2,decastro1,PapazoglouH06}.  Non-functional aspects concerning services' and application's "semantics", often expressed as requirements and constraints in general purpose methodologies, are not fully considered or they are added once the application has been implemented in order to ensure some level of reliability (e.g., data privacy, exception handling, atomicity, data persistence). This leads to services' based applications that are partially specified and that are thereby partially compliant with application requirements.

The objective of this work   is to model non-functional constraints and associate them to  services' based applications  early during the services' composition modeling phase. Therefore this paper presents $\pi$-SOD-M, a model-driven method  that extends the SOD-M  \cite{decastro1} for building reliable  services' based information systems (SIS). 
%SOD-M defines a process  starting with the  identification of business services through business modeling, and, by means of models' transformations it allows to obtain a services' composition model \cite{decastro1} and the executable code that implements it. 

Our work proposes to extend the SOD-M \cite{decastro1} method with  (i)  the notion of {\em A-Policy} \cite{Espinosa-Oviedo2011a} for representing non-functional constraints associated to services' based applications.  
%{\em A-policies} are used to express constraints which can be applied to all the services' composition  or to a particular service used for implementing it. They represent both systems' cross-cutting aspects (e.g., exception handling expressing what to do when a service is not available) and use constraints imposed by the services  (e.g., the fact that a service requires imposes an authentication protocol for executing a method). 
Our work  also (ii) defines the $\pi$-{\sc Pews}  meta-model \cite{Placido2010LTPD} providing guidelines for expressing the composition and the {\em A-policies}. Finally, our work (iii) defines model to model transformation rules for generating the $\pi$-{\sc Pews} model of a reliable services' composition starting from the extended services' composition model; and, model to text transformations for generating the corresponding implementation. As will be shown within our environment implementing these meta models and rules, one may represent both systems' cross-cutting aspects (e.g., exception handling for describing what to do when a service is not available, recovery, persistence aspects) and constraints associated to services, that must be respected for using them (e.g., the fact that a service requires an authentication protocol for executing a method). 

The remainder of the paper is organized as follows. Section \ref{sec:motivation} gives an overview of our approach. It describes a motivation example that integrates and synchronizes well-known social networks services namely Facebook, Twitter and, Spotify. Sections \ref{sec:piscm}, \ref{sec:pewsmetamodel}, and \ref{sec:mmrules} describe respectively the three key elements of our proposal, namelly the $\pi$-SCM and $\pi$-{\sc Pews} meta-models and the transformation rules that support the semi-automatic generation of reliable services' compositions.
%describes $\pi$-SOD-M method that enables the representation and association of {\em A-policies} to services' composition  thereby making them reliable. 
%
Section \ref{sec:implementation} describes implementation and validation issues.
Section \ref{sec:related} analyses related work concerning policy/contract based programming and, services' composition platforms. Section \ref{sec:conclusions} concludes the paper and discusses future work.









\section{Related Works}
\label{sec:relworks}
While Functional Requirements establish \textit{what} is computed by an application, Non-Functional Requirements (NFRs) are concerned with \textit{how} the task is preformed.  
NFRs include aspects such as performance, authentication and quality constraints.
Non-functional properties are also referred to as constraints, quality attributes��, quality goals��, quality of service requirements and ��non-behavioural requirements�� \cite{StGr05}. 
In the case of service-based applications, non-functional requirements concern the application itself as well as its component services. 
%The majority of the works on NFPs focus on measuring in which extent a software system  fulfills the NFPs that it should satisfy \cite{BBKL78,FePf96,KiDa96,Lyu96,MuIO90}. 
Most research on NFRs focus on the evaluation of compliance by the software system. 

In~\cite{Babamir2010,Yeom2006} non-functional properties of web services are classified according to three points of view, namely,  
\textit{service level}, \textit{system level} and \textit{business level}.
In~\cite{Babamir2010} NFRs are denoted as \textit{quality constraints}, which are expressed as logic formulae.
In~\cite{Yeom2006} authors classify NFRs into \textit{category}, \textit{sub-category} and \textit{property}. Categories include \textit{business}, \textit{service} and \textit{system}.
Possible \textit{sub-categories} are \textit{security}, \textit{value} or  \textit{interoperability}. The work also defines a \textit{web service quality model}, which considers non-functional properties. 

In~\cite{XiaoCZBOLH08} the authors use the terms  
\textit{non-functional attributes}, \textit{composition mo\-del entity} and \textit{mo\-del entity}  to classify different concepts related to NFRs.
The notion of non-functional attribute is used to describe NFRs of the abstract process model. In the lower level, the composition is annotated with non-functional attributes.

D'Ambrogio~\cite{DAmbrogio06} uses the term \textit{quality category} to group similar \textit{quality characteristics}. 
\textit{Quality dimensions} are used to quantify an individual characteristic.
For instance, the quality category \textit{performance} groups characteristics such as
\textit{latency} and \textit{throughput}. 
The development process is based on MDA and the authors also present a WSDL extension for describing the QoS of web services. A catalog of \textit{QoS characteristics} is provided for the web service domain, including properties as \textit{availability}, \textit{reliability} and \textit{access control}. 

 
Schmeling et al.~\cite{SchmelingCM11} present an approach and a toolset for specifying and implementing web service compositions with support to several NFRs. The term \textit{non-functional concern} (NFC) is used to denote  NFRs. 
%Two aspects are considered: specification and realization.
\textit{Non-functional concern} is a general term used to describe non-functional requirements. 
For instance, \textit{security}, \textit{reliability}, \textit{transactional behavior} are non-functional requirements. 
A \textit{non-functional action} represents some behavior that implements \textit{non-functional attributes}. 
An example of \textit{non-functional action} is \textit{encryption}, which provides the implementation of the \textit{non-functional attribute} \textit{confidentiality}. 
Non-functional actions related to a common concern are grouped into \textit{non-functional activities}. 

Pastrana et al.~\cite{PastranaPK11} use the term \textit{contract} to describe non-functional requirements. 
\textit{Contracts} may have pre-conditions, post-conditions and invariants. 
Each contract defines \textit{assertions} associated with \textit{quality properties}. 
Each service may have as many associated \textit{contracts} as needed.

Chollet et al.~\cite{CholletL09} associate (non-functional) \textit{quality properties} to 
(functional) activities. The authors present a security meta-model for web service
composition. The NFRs considered are \textit{authentication}, \textit{integrity} and \textit{confidentiality}. 
Each NFR is associated with a service activity.


Ceri et al.\cite{CeriDMF07} uses the notions of \textit{policy}, \textit{rule}, \textit{condition} and \textit{action model} to specify NFRs.
Agarwal et al.~\cite{AgarwalLS09} associate \textit{service policies} to services. 
Each service may also have \textit{properties}, such as \textit{security} and \textit{reliability}. 
Ovaska et al.~\cite{OvaskaEHPA10} use the terms \textit{quality attribute}, \textit{category}, \textit{conceptual layer} and \textit{importance} to organize and classify NFRs.
Other authors do not define specific terms to refer to NFRs. 
They use terms such as \textit{attribute}~\cite{ZhangPSP05,BasinDL06,JeongCL09}, 
\textit{property}~\cite{Fabra2011}, 
\textit{factor}~\cite{MohantyRP10,GutierrezRF10}, 
\textit{characteristic}~\cite{DiamadopoulouMPS08}, 
\textit{quality level}~\cite{ModicaTV09}, and
\textit{value}~\cite{ThissenW06,BasinDL06}.


Despite of the different notations found in the literature for classifying NFRs, some non-functional requirements are frequently considered, such as \textit{security}, \textit{performance}, \textit{reliability}, \textit{usability}, and \textit{availability}.
However, distinct hierarchies and models are proposed for NFRs,  according to different points of view.
We have identified a number of approaches~\cite{DAmbrogio06,CholletL09,SchmelingCM11,BasinDL06,Fabra2011,OvaskaEHPA10} that use MDD (Model Driven Development) for designing and developing applications. 

Fabra \textit{et al.}~\cite{Fabra2011} also describes the importance of  MDD for service-oriented applications. This work  presents a complete development methodology, although this methodology is not centered on NFRs.
The authors in~\cite{ThissenW06,ZhangPSP05} use formal methods to define a service-based development process that takes NFRs into account. 
In~\cite{AgarwalLS09,PastranaPK11} ontologies are used to define and model NFRs, 
whereas in~\cite{XiaoCZBOLH08,GutierrezRF10} Business Process Modeling (BPM) is used for
system specification, including NFRs. 
The majority of the authors concentrate on the  modeling of service compositions, although a significant number of approaches is focused on the definition of NFR models.


In the method defined in~\cite{XiaoCZBOLH08}, tasks in the process model can be 
annotated with \textit{non-functional attributes} (NFAs). 
NFAs are defined apart  and are concerned with data items or tasks. 
NFAs for data considers \textit{value} and \textit{range}, whereas NFAs for tasks include \textit{cost}, \textit{time}, \textit{resources} and \textit{expressions}.

The proposal in~\cite{ThissenW06} presents steps for  selecting services 
by taking QoS information into account. The proposed steps are: 
\textit{(i)} identification of relevant QoS information; 
\textit{(ii)} identification of basic composition patterns and 
QoS aggregation rules for these patterns; and 
\textit{(iii)} definition of a selection mechanism of services. 
The QoS properties considered are \textit{performance}, \textit{cost}, \textit{reliability} and
\textit{availability}. 
  
Karunamurthy et al.~\cite{Karunamurthy2012787} use the term \textit{non-function parameters} to define NFRs, such as \textit{cost}, \textit{response time}, \textit{availability}, \textit{security}, \textit{reliability} and \textit{reputation}.  
The \textit{Non-Func\-tion\-al Specification Language} (NFSL) is proposed as a domain specific language (DSL) to express \textit{non-function parameters}.

Liu et al.~\cite{Liu20121080} use the term \textit{QoS parameter} to describe non-functional requirements such as \textit{cost}, \textit{execution duration}, \textit{accuracy}, \textit{security}, \textit{integrity}, \textit{availability} and \textit{reliability}.  
In the same way, Tran et al.~\cite{Tran2012531} use the term \textit{QoS policies} to classify similar non-functional requirements.

Li et al.~\cite{Li2013} associate \textit{dimensions} to  \textit{QoS parameters} to classify NFRs.  
For instance, the \textit{time} dimension is associated to the \textit{execution time} and \textit{communication time} parameters; the \textit{spatial} dimension is associated to the \textit{storage capacity} and \textit{message length} parameters; the \textit{reliability} dimension is associated to the \textit{availability} and \textit{reliability} parameters and the \textit{cost} dimension is associated to the \textit{service cost} parameter.
Rumpel et al.~\cite{Rumpel2012}  associate \textit{quality requirements} to  \textit{quality properties}. Quality requirements are intended to be specified as constraints. 

\bigskip
Most works agree on distinguishing three points of view, namely the point of view of the organization (or Business view), of the individual service providers (or Service view) and of the composition designer (or System view).
The Business view is concerned with the business logic (\textit{i.e.}, an abstract level of tasks, defined by the guidelines and constraints imposed by the organization).
Service and System views are concerned with the implementation of the software solution: The Service level is concerned with the building blocks of the application.
It may use web services provided by third party sources.
The System level is concerned with the coordination of services, to implement the business logic.


\section{Modeling reliable service compositions with \pisodm}\label{sec:motivation}
%(Aquí lo dejaría con la estructura que teníamos para Caise, presentamos el framework y posteriormente comentamos modelos y transformaciones)


\section{Transformation rules}\label{sec:mmrules}


%..--..--..--..--..--..--..--..--..--..--..--..--..--..--..--..--..--..--..--..--..--..--..--..--..--..--..--..--..--..--..--..
%\subsubsection{$\pi$-{\sc Pews}  meta-model}\label{sec:pewsmetamodel}
%..--..--..--..--..--..--..--..--..--..--..--..--..--..--..--..--..--..--..--..--..--..--..--..--..--..--..--..--..--..--..--..

\pisodm  proposes intra-level transformation rules between $\pi$-use case to $\pi$-service process models, $\pi$-service process to $\pi$-service composition  models, as well as inter-level transformation rules between $\pi$-service composition to $\pi$-PEWS models. 
These rules are used to build a semi-automatic tool.
The transformations from CIM to PIM level models is not automatized, due to the informal nature of CIM level representations.
 
% _ . _ . _ . _ . _ . _ . _ . _ . _ . _ . _ . _ . _ . _ . _ . _ . _ . _ .
\subsection{From $\pi$-UseCase to the $\pi$-ServiceProcess}
% _ . _ . _ . _ . _ . _ . _ . _ . _ . _ . _ . _ . _ . _ . _ . _ . _ . _ .

The refinement of a (composite) $\pi$-Use Case model into a $\pi$-Service Process model is driven by the principle of expressing a set of $\pi$-use cases (i.e., use cases with constraints)   in terms of  a business process.
The resulting model consists of actions related by a control flow and contracts specifying NFPs. 

As defined by the $\pi$-ServiceProcess  meta-model (Figure~\ref{fig:CIM:serviceprocessmetamodel}) an {\sc Activity Service} consists of a composition of  entities of type {\sc Action}. 
Thus, every {\sf $\pi$-use case} is transformed into an {\sf Action} of the target $\pi$-Service Process model.  
Every {\sf Extend} relationship identified in a $\pi$-Use Case model is
transformed into a  {\sf Fork node}\footnote{Notice that Fork nodes are used
both to represent alternative and parallel execution. {\color{red} \sc Placido,
please check this!} {\color{blue} This is consequence of the use cases' extends
and include relationships. To transform an use case into an action, it is
necessary check the dependence relation. The include implies a parallel fork,
and a extends implies in a alternative fork. I included the references for this definition.
} }\cite{valeriaThesis,placidoPhDThesis2012}.
If the {\sf Extend} relationship concerns just one {\sf use case}, it is transformed into an {\sf Action} inside one flow of the fork node. 
Otherwise, several  {\sf use cases} are transformed into different {\sf Actions} that belong to different flows departing from the fork node.   

%\begin{figure}
%\centering{
%\includegraphics[width=0.96\textwidth]{figs/35}}
%\caption{ $\pi$-UseCase to $\pi$-ServiceProcess transformation rules}
%\label{fig:transformationsUseCase-ServiceProcessRules}
%\end{figure}

A {\sf Constraint} associated to a {\sf Use Case}  is transformed into an  {\sf Assertion}.  
The set of resulting assertions  are grouped into a {\sf Contract}.
Constraints are transformed according to their type:
{\sc Business} constraints and {\sc Value} constraints with the {\sf  isExceptionalBehaviour} attribute set to false are transformed into {\sf Assertion}s;
{\sc Value } constraints with the {\sf  isExceptionalBehaviour} attribute set to true are transformed into {\sf Exceptional behaviour}s.

In order to transform constraints of type {\sf Value Constraint}, the designer must specify thresholds to be associated to the assertions of a contract.
By default, value constraints are transformed into pre-conditions and business constraints are transformed into post-conditions. 
  
%A {\sf Requirement} and a {\sf Composite use case} in a  $\pi$-UseCase model are transformed respectively into a {\sf Service activity} and an {\sf Action} (see figures  \ref{fig:transformationsUseCase-ServiceProcessRules}-c  and d). 
%
%
%\begin{figure}
%\centering{
%\includegraphics[width=0.80\textwidth]{figs/37}}
%\caption{ Extended Transformation Examples}
%\label{fig:transformaton-examples}
%\end{figure}

%The transformations for {\sc Extend} and {\sc Include} dependencies  are not as simple as the previous transformations (figures \ref{fig:transformaton-examples}-a). 

The relationships of type  {\sc Extend} and {\sc Include}  determine the way the business process is expressed as a workflow.  
The generated workflow is composed by {\sf Fork} and {\sf Join} nodes,  {\sc Control flow} constructors, as well as entities of type {\sc Action}.

%\begin{figure}
%\centering{
%\includegraphics[width=0.96\textwidth]{figs/36}}
%\caption{ $\pi$-UseCase to $\pi$-ServiceProcess Model Transformation Rules (2)}
%\label{fig:transformationsUseCase-ServiceProcessRules}
%\end{figure}

{\sf Include} use case entities are transformed into an {\sf Action} sequence.
A {\sc Use case} element is transformed into an {\sf Action}. 
A set of $n$ {\sc Use cases} is transformed into an  $n-1$ {\sf Object flow} elements. 

The details of these transformations are not included here (due to space restrictions).
They can be found in~\cite{SouzaNeto:2012}.

\begin{example}[To Publish Music \textit{(cont)}]\label{ex:toPublicMusicT1}
The rules presented above have been applied to the model in Figure~\ref{fig:piUseCaseModel}, in order to obtain the $\pi$-Service Process model for our running example (Figure~\ref{fig:CIM:serviceprocess}).

The ``listen music'' use case is transformed into a Service Action. 
This Action  represents a Spotify service function that can be invoked to play the music. 
For the ``publish music'' use case,  constraints are transformed into a set of assertions that are grouped into a Contract ({\sf ``publishMusicContract''}) associated to the Action {\sf ``publishMusic''}. 
The ``download music'' use case  includes the payment process to buy the music. 
Thus, these use cases  are transformed into {\sf Actions}, and a {\sf Service Activity} that aggregates these {\sf Actions}.   
This \textsf{Service Activity} is transformed into a sequence flow on the $\pi$-service process model.
%(as shown in figure \ref{fig:transformation-example-include}). 
The same rule is applied for the ``publish music'' use case, which has two extended use cases, to publish on Twitter and Facebook.
 \end{example}

%\begin{figure}
%\centering{
%\includegraphics[width=0.76\textwidth]{figs/38}}
%\caption{ Include Transformation Example}
%\label{fig:transformation-example-include}
%\end{figure}

% _ . _ . _ . _ . _ . _ . _ . _ . _ . _ . _ . _ . _ . _ . _ . _ . _ . _ .
\subsection{From $\pi$-ServiceProcess to $\pi$-ServiceComposition}
% _ . _ . _ . _ . _ . _ . _ . _ . _ . _ . _ . _ . _ . _ . _ . _ . _ . _ .

The  principle of the transformation  of a  $\pi$-Service Process model into a $\pi$-Service Composition model is to group {\sf Contracts} into {\sf A-Policies} and {\sf Actions} into {\sf Service Activities}.   

Each {\sf Assertion} of a {\sf Contract} is transformed into a {\sf Rule}. 
{\sf Rules}  concerning the same NFP  are grouped into {\sf A-Policies}. 
Each {\sf Assertion} of a {\sf Contract} is transformed into a {\sf Rule:Condition} attribute. 
If the {\sf Assertion} has a value type, the name and the attributes are transformed into {\sf Variables} in the target model.  
The {\sf Assertion: aProperty} attribute can have different transformations, according to: 
\textit{(i)} a {\sf Precondition} is transformed into {\sf Pre};
\textit{(ii)} a {\sf Post-Condition} is transformed into a {\sf  Post};
\textit{(iii)} a {\sf TimeRestriction} is transformed into {\sf Time}.

%
%\begin{figure}
%\centering{
%\includegraphics[width=0.96\textwidth]{figs/ExceptionalRules}}
%\caption{$\pi$-ServiceProcess to $\pi$-ServiceComposition Model Transformation Rules}
%\label{fig:ServiceProcess-ServiceComposition-Rules}
%\end{figure}
{\color{red}
A {\sf Package} in the $\pi$-use case model is transformed   into a {\sf Business Collaborator}. 

Placido: Is this correct? Why it is here? We consider the $\pi$-use case in the previous section!
}{\color{blue} Yes, it is correct. I defined it in the thesis because in the
previous SOD-M a Business Collaborator comes from 'nothing'. So, as we needed to
refine most concepts from use case, I decided separete the use cases in
packages and then tranform it in a Collaborator. As a use case is transformed
into an action, and all action is from a BC, I related both concepts. There is a
layer gap ($\pi$-ServiceProcess), because in this layer we define only the
actions.  

We are defining it here because the concept is necessary for the
transformation. Packages is part of UML in the use case model, and I used it for define the colaborators.}

{\sf Actions} and {\sf Service activity} of a $\pi$-Service Process model are transformed into their homonym concepts of the $\pi$-Service Composition model. 

{\color{blue} An {\sf Exceptional behaviour} entity is transformed into an {\sf
Action} in a $\pi$-Serv\-ice\-Com\-po\-si\-tion model and every {\sf
Non-functional attribute} associated to an {\sf Contract} (it comes from the $\pi$UseCase model) element 
in a $\pi$-ServiceProcess model becomes a {\sf Non-functional Requirement}
associated to the corresponding element {\sf A-Policy} in a
$\pi$-ServiceComposition model}.\footnote{\color{red} Not found in the
metamodel! --Martin and Umberto \color{blue}  ANSWER:  If you analyze the model $\pi$-UseCase, a constraint may be associated with one or more NFA (Non-functional attribute). The constraints
associated with the same use case, are grouped into a contract. Each contract is
related to a actions ($\pi$-ServiceProcess). Thus, the contracts (set of
constraints) have NFAs associated (from piUseCase). The transformation between
the first 2 levels (PIM $\pi$-UseCase to $\pi$-ServiceProcess) give us this.
NFAs associated to an element (Contract) in a $\pi$-ServiceProcess model are
refined (grouped) into a Non-functional requirement associated to the corresponding
element A-Policy in the $\pi$-ServiceComposition model. I have to update the
meta-models (thesis version) because they were outdated.} Finally, {\sf Actions}
are grouped by  a {\sf Business collaborator}.\footnote{\color{red} How? Where
in the metamodel? --Martin and Umberto \color{blue} ANSWER: Figure
\ref{fig:e-scomposition-metamodel} presents the relationship between Action and BusinessCollaborator}

%\begin{figure}
%\centering{
%\includegraphics[width=1\textwidth]{figs/ServiceProcess-ServiceComposition}}
%\caption{ Transformation Rules: From $\pi$-ServiceProcess to $\pi$-ServiceComposition}
%\label{fig:ServiceProcess-ServiceComposition}
%\end{figure}






%As the $\pi$-ServiceComposition model refines the $\pi$-ServiceProcess concepts at PIM level, a service previously defined as actions (source model) is refined as composition of those actions (target model) that are necessary to represent a business service, identifying who are the partners involved in the realization ({\sc Business collaborators}). 
%In addition, $\pi$-SOD-M defines a platform specific model based on web services composition. This model is explicitly indicates those actions which are (or will be, if not yet implemented) supported by web services.

\begin{example}[To Publish Music \textit{(cont)}]\label{ex:toPublicMusicT5}
Considering the example scenario, the model in Figure~\ref{fig:PiServiceCompositionModel} was obtained by applying the rules above to the model depicted by Figure~\ref{fig:CIM:serviceprocess}.  

The "securityLoginPolicy" consists of a set of {\sf Rules} that were transformed from the {\sf Assertions} in $\pi$-service process model. 
The information about Facebook and Spotify (both of them {\sf Business
Collaborators}) come from entity of type {\sc Package} in the $\pi$-use case
model.\footnote{\color{red} Como assim? - Placido, explain this, please! Why
have we bypassed one level here? \color{blue} ANSWER: It was explained before.}
\end{example}

% _ . _ . _ . _ . _ . _ . _ . _ . _ . _ . _ . _ . _ . _ . _ . _ . _ . _ .
\subsection{From $\pi$-ServiceComposition to $\pi$-PEWS}
% _ . _ . _ . _ . _ . _ . _ . _ . _ . _ . _ . _ . _ . _ . _ . _ . _ . _ .

This section describes the PIM to PSM transformations from a $\pi$-Service composition model to a $\pi$-PEWS model. 
We distinguish two groups of rules: \textit{(i)} those transforming service composition entities into workflows; and \textit{(ii)} those that transform  A-Policies into Contracts.

Single actions (represented by {\sf Action} and {\sf
Action:name}\footnote{\color{red} Where is the name defined? \color{blue} ANSWER: I updated the metamodel with the name atribute, however
in general, I ommit the atributes for the metamodel presentation, but in the
implementation the atributes are there.}) are transformed into individual service {\sf Operations}.
Complex actions (represented by {\sf ServiceActivity}  and  {\sf
ServiceActivity:name}\footnote{\color{red} Where is the name defined?
\color{blue} ANSWER: I updated the metamodel with the name atribute, however
in general, I ommit the atributes for the metamodel presentation, but in the
implementation the atributes are there.}) are transformed into (named) composite
operations that defines a (named) workflow of the application.
Composition patterns expressed using the operators {\sc\em merge, decision, fork and join} are transformed into their corresponding workflows of the $\pi$-PEWS model.

%\begin{figure}
%\centering{
%\includegraphics[width=0.96\textwidth]{figs/Table7}}
%\caption{Transformation Rules: From $\pi$-ServiceComposition to $\pi$-PEWS}
%\label{fig:ServiceComposition-Pews-Rules}
%\end{figure}

The A-Policies defined for the entities of a $\pi$-service composition model are transformed into {\sf A-policy}
\footnote{\color{red} Please rename this for CONTRACT on the pi-pews metamodel!
\color{blue} ANSWER: The latest version of the $\pi$-PEWS grammar I defined
policy.} entities, named according to the names expressed in the source model.
The transformation of the rules expressed in a $\pi$-service composition is guided by the event types associated to these rules. 
The variables associated to an A-Policy expressed in a $\pi$-service composition model as {\sf $<$Variable:name, Variable:type$>$} are transformed into entities  {\sf Variable} with attributes {\sf Name} and {\sf Type} directly specified from the elements {\sf Variable:name} and {\sf Variable:type} of a $\pi$-service composition model.

Events of type {\sf Pre}, {\sf Post} and {\sf Time} generate, respectively, {\sf Preconditions}, {\sf Postcondition} and {\sf TimeRestrictions}.

\begin{example}[To Publish Music \textit{(cont)}]\label{ex:toPublicMusicT6}
Figure \ref{fig:piSOD-M} shows the $\pi$-PEWS code resulting from the $\pi$- service composition model  of our scenario example.
\end{example}


\subsection{Implementation}
We have provided tools for aiding the user to define and transform the models for all the levels, except for the CIM into PIM level, which should be manually performed.

Our tool is implemented as a series of Eclipse plug-ins: 
\begin{itemize}
\item 	We  used the Eclipse Modeling Framework (EMF)\footnote {The EMF project is a modeling framework and code generation facility for building tools and other applications based on a structured data model.}   for implementing the  $\pi$-Service Composition and $\pi$-{\sc Pews}  meta-models. 
Then, from these meta-models, we  developed plug-ins to support their graphical representation.

\item	 We used  ATL\footnote{http://eclipse.org/atl/. An ATL program is basically a set of rules that define how source model elements are matched and navigated to create and initialize the elements of the target models.}
for implementing the  mappings between models.

\item 	We  used Acceleo\footnote{http://www.acceleo.org/pages/home/en} for implementing  the code generation plug-in for generating executable code. 
It takes a $\pi$-PEWS model and generates the code to be executed by the {\em
A-Policy} based service composition execution environment\footnote{\color{red}
Placido: Policy or contract??. \color{blue} ANSWER: The latest version of the
$\pi$-PEWS grammar I defined \textbf{policy}. Particularly I prefer \textbf{policy},
because for what we're proposing, one same policy can be applied to different
services and groups a set of contracts.}.
\end{itemize}



\section{Applying \pisodm: The \FlyingPig\ case study}
\label{sec:flyingPig}

We validated our methodology by developing a use-case concerning risk assessment for financial companies.
Our application is inspired on the ORCA System\footnote{The ORCA System is a trademark of GCP Global (www.gcpglobal.com).}.
Risk assessment is implemented by an interactive business process based on the exchange of a series of questionnaires intended to evaluate the risks implied the client's business practice.
For instance, the conditions and protocols used to perform confidential transactions, the physical security for accessing reserved areas such as computing server installations.
The information gathered by the questionnaires is used to determine whether there are risky practices within the business processes of the company, as well as to propose amends to these practices.
The ultimate goal of the risk assessment is to determine a degree of compliance to existing standards.
By analysing the questionnaires, ORCA detects risky practices, proposes solutions and triggers further assessment processes to ensure that the solutions have been implemented.

Our goal is to model a service based application (called \FlyingPig), for providing risk assessment as a service.
In order to provide this functionality, \FlyingPig\ would benefit from ORCA's legacy services: storage, assessment and data visualization functions.

In the following, we describe the results of applying $\Pi$-SODM to develop the \FlyingPig\ risk assessment system.
The models presented next were generated as a result of interacting with software developers at GCP Global.

\begin{figure}
\centering
\includegraphics[width=0.7\textwidth]{figs/3ValueModel.pdf}
\hspace*{5cm}\includegraphics[width=0.4\textwidth]{figs/3ValueKey.pdf}
\caption{E3value model for \FlyingPig.\label{fig:E3valuemodel}}
\end{figure}


\subsection{Computation-Independent Models (CIM)}

$\Pi$-SODM uses two models at the CIM level (see Section~\ref{sec:modelingWithPISODM}): The E3value~\cite{e3value} model and BPMN~\cite{BPMN}.
The former is a simple model just to identify the transference of value information between components of the system.
The BPMN model establishes which are the actors and main tasks of the application.

Figure~\ref{fig:E3valuemodel} shows the value model for the \FlyingPig\ application.
It is a business model that graphically represents a business case as a set of value exchanges ($\triangleright$ and $\triangleleft$) and value activities (rounded boxes) performed by business actors (squared boxes).

In our use-case, we identify two business actors: \textsl{ORCA} and \textsl{Broker}. 
Brokers are responsible for channelling requests for risk assessment of one or several companies. 
ORCA have two value activities which are services that provide an economical benefit: to \textsl{Identify Amendments} and the possibility to \textsl{Assess Risk Situation}. 
The values exchanged between ORCA and the brokers are: \textsl{Amendments} and \textsl{Evaluation Reports} which are value objects ([ \!\dots]) for the companies that need to have a risk assessment, as well as  \textsl{Questionnaire and Evidences} and the risk assessment \textsl{Fee} which are value objects for ORCA System.

The e3value defines \textit{dependency paths}, showing the value exchanges, which are triggered by the occurrence of an end-consumer need (in our case, the need of a risk assessment). 
A dependency path has a direction and consists of a sequence of linked dependency nodes.
A dependency path starts with a \textit{start stimulus} node and ends with an \textit{end stimulus} node (see Legend on Figure~\ref{fig:E3valuemodel}). 
Dependency paths may also contain \textsl{OR} and \textsl{AND} elements (both for initiate and join alternative and parallel paths).

The dependency path in Figure~\ref{fig:E3valuemodel} initiates with the need of assessment by a particular company. 
Once this need occurs, the value exchanges between ORCA and Broker are triggered. 
The client company will provide orca wit information (answers to a questionary), evidence (to support the information) and a fee (monetary value).
ORCA will provide amendments (recommendations to change practices) and an evaluation report. 

\begin{figure}[t]
\centering
\includegraphics[width=1.0\textwidth]{figs/BPMN_GCP.pdf}

{\color{red} Javier: Please change \underline{Asses} by \underline{Assess}. --M}
\caption{BPMN model for \FlyingPig.\label{fig:BPMNmodel}}
\end{figure}

The BPMN model is devised to better understand the process in which the value exchanges occur.
Figure~\ref{fig:BPMNmodel} shows the BPMN model\footnote{Details on BPMN (Business Process Management Notation) can be found in http://www.bpmn.org/.} for the \FlyingPig\ scenario. 
The model includes two pools representing the \textsl{ORCA} system and the \textsl{Brokers}. 
Brokers have two lanes, the client \textsl{Company} and a \textsl{User}. 
The user is a contact member of the company, who will coordinate the assessment process. 
This process will involve other members of the company as well.

The risk assessment process starts after a request from a company.
(This agrees with the value model, in which the start stimulus triggers the whole process.)
The request leads to the definition of a group of users that will answer questionnaires for evaluating risk.
Questionnaires are considered tasks that users will have to perform. 
Other tasks include amending a ``risky situation'' as well as producing evidence to show that a specific risk has been eliminated\footnote{Risky situations include from physical facts such as not having easy access to handicapped persons or having an unsecured access to the premises of the company, to more intangible ones, such as the use of an less-than-optimal protocol to access data on the company's computer server.}.

Once tasks are completed, they are stored and analysed to generate a list of un-compliant situations, associated to their corresponding \textit{calls for amendment}, in case there are any, or a report specifying a compliance level, incidents and a risk map.
During the process of analysing a questionnaire, the answers of some questions can trigger the generation of other questionnaires or amendments, that will become new tasks.  

Business processes have also associated rules and constraints that define their non functional requirements.
NFR represents the ``semantics'' and the conditions in which the tasks must be done.
In our example we have some constraints.

{\color{red}
Placido, Valeria: What can we say about non-functional requirements at the CIM level?
}



\begin{itemize}
\item {\color{magenta} Placido and Umberto: Could you please complete these items? --G \& M.}
\end{itemize}




\subsection{Platform-Independent Models (PIM)}

In Section~\ref{sec:modelingWithPISODM} we defined three models at the PIM level.
These models are built next, for the \FlyingPig\ scenario.

\begin{figure}[t]
\centering
\includegraphics[width=0.9\textwidth]{figs/UseCaseGeneral.png}

{\color{red} \raggedright
$\star$ ``Create a responsible'' should be changed for ``Designate user in charge''.

$\star$ ``requires concurrency for more than 200 users'' should be changed by ``Maximum number of users should be greater than 200''.
}
\caption{$\pi$-UseCase model for \FlyingPig.\label{fig:piUseCaseModel}}
\end{figure}


\paragraph{\underline{$\pi$-UseCase Model for \FlyingPig}}~

The $\pi$-UseCase model shown in Figure~\ref{fig:piUseCaseModel} describes the features and constraints for the \FlyingPig\ application. 
In this model, three actors are identified: \textit{Company}, \textit{User} and \textit{Channel-Broker}\footnote{\color{red} Can we change this for Broker ? --M.}. 
They are represented as stick figures.
In the context of \FlyingPig, Company is the actor asking for risk evaluation.
A Channel-Broker is the responsible for channelling the the evaluation process, assigning users to be in charge of tasks as well as delegating tasks. 
A User, in this model is an actor who answers questionnaires (with base on the actual facts about the Company).
The User also produces evidence to support facts and performs the necessary amendments to improve the results of the risk assessment.

Each actor is associated to one or more use cases (depicted as white ovals in Figure~\ref{fig:piUseCaseModel}). 
Use cases describe the main functionalities of the system.
The $\pi$-UseCase model for \FlyingPig\ defines six use cases. 

In our model, each use case may be associated to one or more (non-functional) constraints (depicted as coloured ovals in Figure~\ref{fig:piUseCaseModel}). 
The model defines three types of constraints: \textit{value} , \textit{business} or \textit{exceptional behavior}. 
Each constraint is identified by the word $<<$\textsf{constraint}$>>$ followed by its type.

In the case of \FlyingPig, the model counts seven constraints:
\begin{numtrivlist}
\item An acknowledgement is due less than 30 seconds after registering a task or demand for assessment. 
\item The system's infrastructure should be prepared to deal with, at least, 200 users. 
\item If the number of requests exceeds 200, \FlyingPig\ should make a load balance of the requests. 
\item The Channel-Broker must have their privileges verified \textit{before} the execution of the actions associated to the \textsf{designate user in charge} use case.
\item Users must have their privileges verified \textit{before} the execution of the actions associated to the \textsf{answer questionnaire and add evidences} use case.
\item All questionnaires need to be fully answered, in order to consider that a task is completed.
\item There is a time limit (in days) for each amendment required by the system.
\end{numtrivlist}

{\color{red} Anything else here? --M}

\begin{figure}
\centering
\includegraphics[width=1.0\textwidth]{figs/ServiceProcessGeneral.png}
\caption{$\pi$-ServiceProcess model for \FlyingPig.\label{fig:PiServiceProcessModel}}
\end{figure}

\paragraph{\underline{$\pi$-ServiceProcess Model for \FlyingPig}}~

The $\pi$-ServiceProcess model (Figure~\ref{fig:PiServiceProcessModel}) presents the workflow for \FlyingPig.
The actions in this model were obtained by applying the use case transformation rules described in Section~\ref{sec:pewsmetamodel}.

The Company, Broker-Channel and User actors are transformed into lanes that represent the business collaborators.
Use cases are transformed into \textit{actions} and are represented by white boxes.
The restrictions associated to each use case are transformed into \textit{assertions} (represented by coloured boxes) and may be decorated with pre- and post-conditions. 
The set of assertions related to a particular action form a \textit{contract} for this action. 

This model refines the concepts defined in the $\pi$-UseCase model and it is more suited to better describe the assertions, by grouping them into contracts.

\begin{figure}
\centering
\includegraphics[width=1.0\textwidth]{figs/ServiceCompositionGeneral.png}
\caption{$\pi$-ServiceComposition model for \FlyingPig.\label{fig:PiServiceCompositionModel}}
\end{figure}

\paragraph{\underline{$\pi$-ServiceProcess Model for \FlyingPig}}~

This model presented in Figure~\ref{fig:PiServiceCompositionModel} describes the interactions of the actions described in pi-ServiceProcess model with the external services. Services are provided through the ORCA system and the access is accomplished by FlyingPig interface. FlyingPig is the point of interaction between the actions described in the precess with the services offered by ORCA. FlyingPig provides an interface with five actions, they are: receive a request, generate new interface, create questionnaire, notify responsable e receive answers.. This interface makes the direct calls for the ORCA' services. The services offered by ORCA are: create space company, generate the questionnaire, analyze answer, data store, call for amendment and create report. These services are called according to the process execution.

With respect to non-functional requirements, the assertions grouped into contract in the pi-ServiceProcess model are transformed into policies in the pi-ServiceComposition model. The policies identified in this case study are: security, performance and conformability policies. Each policy is associated with specific the services and it is verified at the service execution time regarding the specific rule which the sercive is associated with..



\subsection{PSM}

{\color{magenta} Placido and Umberto: Could you please insert the $\pi$-PEWS model here? --G \& M.}


%%%%%%%%%%%%%%%%%%%%%%%%%%%%%%%%%%%%%%%%%%%%%%%%%%5
%% By Placido...
%%%%%%%%%%%%%%%%%%%%%%%%%%%%%%%%%%%%%%%%%%%%%%%%%%

Pi-UseCase Model description for FlyingPig

The piUseCase model shown in Figure  X models the use cases (features) and constrainsts for the FlyingPig application. 3 actors are identified in this model, the are: (I) Company, (II) User and the (III) Channel-Broker. The Company is the whom ask for risk evaluation, the (II) Channel-Broker is the responsible for managing a defining the evaluation flow, beyond create responsible for evaluating and delegate the tasks. This model also represents the (III) User actor which answers questionnaires for the assessment, including evidence and proceeding with necessary corrections when asked for this type of task.

Each actor is associated with use cases. Use cases represent the actions that the actors are responsible. Altogether, this model defines 6 (six) use cases. The use cases related to the Company's actor is (i) request assessment. The use cases related to Broker-Channel's actor are: (ii) create a responsable and (iii) delegate task; and the use cases related User's actor are: (iv) answer questionnaire and add evidences, (v) register task, and (vi) perform amendment.

In this model, each use case may be related with one or more constraints. The constraints can be of three types: value , business or exceptional behavior . This model as a whole has 7 constraints . The use case (i) request assessment has 2 constraints related with, the are from the value type. One is related to ( i.a) response time (confirmation) that should be up to 30 seconds , and the other ( i.b ) describes that the system should support a large number of requests from different companies . Moreover ( i.c ) if the number of requests exceeds 200 , FlyingPig should make a load balance of the requests. The use case create a responsable has a value constraint ( ii.d ) which defines that the Channel-Broker must have privileges verified before the execution to perform this action. This use case also ( ii.c ) is also related with the load balancing verification. The use case ( iii ) delegate task has no associated constraint . The use case answer questionnaire and add evidences have 2 constraints , they are: the User which performs this action ( iv.e ) must have privileges to answer the questionnaires and ( iv.f ) the time limit must be less than 15 days. The use case register task has 3 constraints associated . Two of them are the same constraints related with the request assessment use case, where (v.a) need a response time confirmation that should be up to 30 seconds, and the other ( v.b ) describes what the system should support a large number of requests for this action, once Users from various companies will be answering the questionnaire and performing tasks. The other constraint requires that ( v.g ) throughout the survey all the questions must be properly answered . Finally , the use case ( vi.f ) perform amendment must be executed in a set time limit in days, the same constraint related with the use case questionnaire answer and add evidences. Figure A presents the relation between the use case and its constraints.






TABLE A - RELATION BETWEEN USE CASE AND CONSTRAINT
use cases / constraints	(i) request assessment	(ii) create e responsable 	(iii) delegate task	(iv) answer questionnaire and add evidences	(v) register task	(vi) perform amendment
(a) confirmation must be received in up to 30sec	X				X	
(b) requires concurrency for more than 200 users	X				X	
(c) load balance if there is more than 200 requests	X	X				
(d) requires Broker’ privileges		X				
(e) requires User' privileges				X		
(f) all tasks must be done within a given time limit (e.g. maximum of 15 days)				X		X
(g) requires questionnaire completion					X	


Pi-ServiceProcess Model description for FlyingPig 

The pi-ServiceProcess model (Figure Y) represents the application execution flow of FlyingPig. In this model the action are the result of the use case transformation identified in pi-UseCase model according to their respective actors. The Company, Broker-Channel and User actors are transformed into rays that represent the business collaborators.

The restrictions associated with each use case are transformed into assertions. The set of assertions related to a particular action form the contract for this action. The process of FlyingPig executing begins with the action ( i ) request assessment performed by the Company. This action has one pre-condition regarding the response time and a post-condition concerning the load balance control for the server when it reaches 200 requests. Once the request is made by the Company for risk analysis, the Broker ( ii ) creates responsible for its analysis and ( iii ) delegate the necessary risk assessment. These actions are executed in sequence and has a pre-condition concerning the Broker's privilege. This can only create responsible and delegate the tasks if their authentication datas are correct and if and there was less than 3 requests. The tasks to be performed by the User can be ( iv ) answer questionnaire and add evidences; or ( v) perform amendments. Both the assertion has one pre-condition concerning the time limit for form submission. In this example, we use the limit of 15 days for form submission. If the User is performing the task of answer questionnaire and add evidences, this will have to enter with their authentication data. As postcondition, it is checked if all the answers were answered . The action of perform amendments has no additional verification. Finally, the user registers the tasks performed. This action has response time verification.

This model refines the concepts defined in the pi-UseCase model in order to better describe the assertions and groups them into contracts.


Pi-ServiceComposition Model description for FlyingPig 

This model (presented in Figure Z) describes the interactions of the actions described in pi-ServiceProcess model with the external services. Services are provided through the ORCA system and the access is accomplished by FlyingPig interface. FlyingPig is the point of interaction between the actions described in the precess with the services offered by ORCA. FlyingPig provides an interface with five actions, they are: receive a request, generate new interface, create questionnaire, notify responsable e receive answers.. This interface makes the direct calls for the ORCA' services. The services offered by ORCA are: create space company, generate the questionnaire, analyze answer, data store, call for amendment and create report. These services are called according to the process execution.

With respect to non-functional requirements, the assertions grouped into contract in the pi-ServiceProcess model are transformed into policies in the pi-ServiceComposition model. The policies identified in this case study are: security, performance and conformability policies. Each policy is associated with specific the services and it is verified at the service execution time regarding the specific rule which the sercive is associated with..





%%%%%%%%%%%%%%%%%%%%%%%%%%%%%%%%%%%%%%%%%%%%%%%%%%%%%
\subsection{Lessons Learned}

Through the example we underlined that every application implements functional aspects that describe its application logic.
Recall that an application logic refers to routines that perform the activities to reach the application objective.
Also there are non functional properties derived from NFR. They refer to strategies to be considered for the application execution like: security, isolation, adaptability, atomicity, and more.
These non functional properties must be ensured at execution time, and they are not completely defined within the application logic.

The challenge is to define them and to associate them with the application logic considering that different to existing solutions that suppose that it is possible to access the execution stat of all the components  of an application and that the application has complete control on them, in the case for service oriented applications  the components are autonomous services
API does not necessarily export information about methods dependency (e.g., in the REST protocol);
they do not share their state (stateless).

Given a set of services with their exported methods known in advance or provided by a  service directory, building services' based applications can be  a simple task that implies expressing an application logic as a services' composition. The challenge being  ensuring the compliance between the specification and the resulting application. Software engineering methods (e.g., \cite{1,2,decastro1,PapazoglouH06}) today can help to ensure this compliance, particularly when information systems include several sometimes complex business processes calling Web services or legacy applications exported as services.


\section{Conclusions and future work}\label{sec:conclusions}
We presented \pisodm, an model-driven method for designing and developing reliable service-based applications.
\pisodm extends a previously defined method (called SOD-M) to include Non-Functional Requirements.
These requirements are taken into account from the early stages of the software development process.
Non-functional constraints are related to business rules associated to the behavior of the application and, in the case of service-based applications, they are also concerned with constraints imposed by the services.

Our method includes two CIM-level models, three PIM-level models and one PSM-level model.
We implemented the meta-models on the Eclipse platform and we validated the approach by using an industrially inspired use case.

Our case study demonstrates the applicability of \pisodm.
This case study was developed together with our industrial partner, GCP Global.
The Company is using \pisodm for the development of their product.
The case study presented here is a simplified version of their application. 


\bibliographystyle{plain}
\bibliography{biblio}

\end{document}


