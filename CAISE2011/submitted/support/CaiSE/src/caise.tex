%%This is a very basic article template.
%%There is just one section and two subsections.
\documentclass{article}

\begin{document}


\section{Related Works}  

%Some works related with service-oriented
%development have presented that the impact of the service-oriented
%computing paradigm, the quality garantees on software development and the
%way in which systems can be constructed have been growing up, however 

There are
 few methodologies and approaches that address the explicit modeling  of non functional
properties for service based applications.   Software process methodologies for
building  services based applications have been proposed in \cite{PapazoglouH06,Papazoglou03,cdl2006,MilanovicM06,FeuerlichtM05,Ramollari_asurvey,somet2005}, and they focus mainly on the modeling and construction process of services based business processes that represent the application logic of information systems. 

 \textit{Design by Contract} \cite{HL05TACoS} is an approach for specifying web services and verifying them through runtime checkers before they are deployed. A contract adds behavioral information to a service specification, that is, it specifies the conditions in which methods exported by a service can be called. Contracts are expressed using the language \textit{jmlrac} \cite{LeavensCCRC02} . 
 

 The \textit{Contract Definition Language} (CDL) \cite{cdl2006} is a XML-based
description language, for defining contracts for services. There are an associated architecture framework, design standards and a
methodology \cite{MilanovicM05,Milanovic05,Milanovic06,MilanovicM06}, for developing applications using services. 
A services' based application  specification is generated after several 
B \cite{AbrialLNSS91} machines refinements that describe the services and their
compositions.

\cite{PapazoglouH06} proposes a methodology 
based on a SOA extension. This work defines a service
oriented business process development methodology with phases for business process
development. The whole life-cycle is based on six phases: planning, analysis and design,
construction and testing, provisioning, deployment, and execution
and monitoring. 

IBM proposes a methodology for the development of SOA
solutions, called SOMA \cite{soma}. SOMA defines a life-cycle with seven
phases: business modelling and transformation, solution management, identification,
specification, realization, implementation and deployment monitoring
and management.

Sommerville \cite{sommerville08} describes some key points for building services' based applications based on business process models that define the activities and information
exchanged in a business processes. Activities in business process
can be performed by services so that the model of business process
represents a composition of services. It classifies services as  as public utilities, business services or
coordination services. Software development that uses services involves creating
programs for composing and configuring services to create new
composite services. The service engineering process involves identifying services
candidates, service interface and
implementation definition, testing and deployment of each service.

Unlike the works presented, the main contribution of our proposal is the
methodology description together with model representations in three levels (CIM, PIM and PSM) 
 for the design and development of distributed applications that can be reused
 and that are reliable. 
 


\section{Statements}  
\label{sec:challenges}   

The research questions that guide our research are
intimately related with the spirit of classic software engineering, specifically
the definition of a method for non functional properties-based distributed
applications development. We also have in mind the application of the concept related with
model driven development. There is a need to develop a proper model for these
emerging technologies to reduce developing costs and to produce flexible and adaptable
services based on quality properties. Therefore, we must ask the following:

\begin{enumerate}
  \item What the elements that make the development of web
  service applications different from traditional software development?
  \item Is it possible to define a methodology for web service application
  development that is based on non functional properties requirements?
  \item Is it possible to define a language to specify composition constraints
  and contract restriction for web service?
  \item How to verify web services contract restrictions at runtime?
  \item It is possible to summarize all the development through models?
\end{enumerate}
 
Therefore, we will defend the following concepts:
 
``\textit{Developing applications that use compositions of services from a
methodology for this purpose and which is based on non functional properties
assurance requirements, can provide a better development and outcome. Allied to this, a
method based on MDD / MDA can provide a better reuse of applications and models.
Finally, a specification language that expresses web service composition and its
constraints can help further the development.}''

\bibliography{caise}  
\bibliographystyle{plain} 

\end{document}
