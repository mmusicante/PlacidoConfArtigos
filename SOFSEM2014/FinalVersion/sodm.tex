
The Service-Oriented Development Method (SOD-M)~\cite{decastro1}
adopts  the MDD approach to build service-based applications. SOD-M considers two points of view:
\textit{(i)} \textit{business}, focusing on the characteristics and requirements
of the organization, and \textit{(ii)} \textit{system requirements}, focusing on
features and processes to be implemented in order application requirements. In
this way, SOD-M  simplifies the design of service-oriented applications, as
well as their implementation using current technologies.

SOD-M provides a framework with models and standards to express functionalities
of applications at a high-level of abstraction. SOD-M meta-models are organized 
into three levels: CIM (\textit{Computational Independent Models}), 
PIM (\textit{Platform Independent Models}) and PSM (\textit{Platform Specific Models}).
Two models are defined at the CIM level: \textit{value model} 
and \textit{BPMN model}. 
The PIM level models the entire structure of the application flow,
while, the PSM level provides transformations towards more specific platforms.
The PIM-level models are: \textit{use case}, \textit{extended use case}, \textit{service process} and
\textit{service composition}. The PSM level models are: \textit{web service interface}, \textit{extended composition service} and \textit{business logic}. 
 
 %%%BEGIN{Text Included by Placido Neto for Camera-Ready}%%%%%%
The \textit{value model} is a business model that describes a business
case as a set of values and value activities shared by business actors. The
\textit{BPMN model} (business process model) is used to describe the business
process related to the environment which the system will run. These two models
represent the independent aspects of computing. The \textit{use case model} is used to represent
the business services to be implemented by the system, while the \textit{extended use case model} is a
behavioral model, to represent the system features as a way to implement
the business services. The \textit{service process model} describes the set of
activities that must be performed on the system to implement a business service.
Finally, the \textit{service composition model} represents the full flow of
business system. This model is an extension of the service process model,
however, in more detail. These four models represent the platform independent
aspects.
%%%END{Text Included by Placido Neto for Camera-Ready}%%%%%%

The SOD-M approach includes transformations between models:
\textit{CIM-to-PIM, PIM-to-PIM} and \textit{PIM-to-PSM} transformations. Given
an abstract model at the CIM level, it is possible to apply transformations for
generating a model of the PSM level. In this context, it is necessary to
follow the process activities described by the methodology. 
%
These three SOD-M levels have no support for describing non-functional requirements.The following section introduces $\pi$-SODM the extension proposed for considering these requirements.



