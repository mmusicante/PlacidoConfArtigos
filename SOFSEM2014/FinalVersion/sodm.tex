
The Service-Oriented Development Method (SOD-M)~\cite{decastro1}
adopts  the MDD approach to build service-based applications. SOD-M considers two points of view:
\textit{(i)} \textit{business}, focusing on the characteristics and requirements
of the organization; and 
\textit{(ii)} \textit{system requirements}, focusing on
features and processes to be implemented in order to build service-based applications in accordance to the business requirements. In
this way, SOD-M  simplifies the design of service-oriented applications, as
well as their implementation using current technologies.

SOD-M provides a framework with models and standards to express functionalities
of applications at a high-level of abstraction. The SOD-M meta-models are organized 
into three levels: CIM (\textit{Computational Independent Models}), 
PIM (\textit{Platform Independent Models}) and PSM (\textit{Platform Specific Models}).
Two models are defined at the CIM level: \textit{value model} 
and \textit{BPMN model}. 
The PIM level models the entire structure of the application, as
the PSM level provides transformations towards more specific platforms.
The PIM-level models are: \textit{use case}, \textit{extended use case}, \textit{service process} and
\textit{service composition}. The PSM-level models are: \textit{web service interface}, \textit{extended composition service} and \textit{business logic}. 
 
 %%%BEGIN{Text Included by Placido Neto for Camera-Ready}%%%%%%
At the CIM level, the \textit{value model} describes a business
scenario as a set of values and activities shared by business actors. 
The \textit{BPMN model} describes a business process and the corresponding environment. 
%These two models represent the computing-independent aspects of the application logic. 
At the PIM level, the \textit{use case model} represents
a business service, as the \textit{extended use case model} contains 
behavioral descriptions of features to be implemented. 
The \textit{service process model} describes a set of
activities to be performed in order to implement a business service.
Finally, the \textit{service composition model} represents the complete flow of a business system. 
This model is an extension of the service process model. 
%These four models represent the platform independent aspects.
%%%END{Text Included by Placido Neto for Camera-Ready}%%%%%%

The SOD-M approach includes transformations between models:
\textit{CIM-to-PIM, PIM-to-PIM} and \textit{PIM-to-PSM} transformations. Given
an abstract model at the CIM level, it is possible to apply transformations for
generating a model of the PSM level. In this context, it is necessary to
follow the process activities described by the methodology. 
%
These three SOD-M levels have no support for describing non-functional requirements. The following section introduces $\pi$SOD-M, the extension that we propose for suporting these requirements.



