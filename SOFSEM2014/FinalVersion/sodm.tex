
The Service-Oriented Development Method (SOD-M)~\cite{decastro1}
adopts  the MDD approach to build service-based applications. SOD-M considers two points of view:
\textit{(i)} \textit{business}, focusing on the characteristics and requirements
of the organization, and \textit{(ii)} \textit{system requirements}, focusing on
features and processes to be implemented in order application requirements. In
this way, SOD-M  simplifies the design of service-oriented applications, as
well as their implementation using current technologies.

SOD-M provides a framework with models and standards to express functionalities
of applications at a high-level of abstraction. SOD-M meta-models are organized 
into three levels: CIM (\textit{Computational Independent Models}), 
PIM (\textit{Platform Independent Models}) and PSM (\textit{Platform Specific Models}).
Two models are defined at the CIM level: \textit{value model} 
and \textit{BPMN model}. 
The PIM level models the entire structure of the application flow,
while, the PSM level provides transformations towards more specific platforms.
The PIM-level models are: \textit{use case}, \textit{extended use case}, \textit{service process} and
\textit{service composition}. The PSM level models are: \textit{web service interface}, \textit{extended composition service} and \textit{business logic}. 
 

The SOD-M approach includes transformations between models:
\textit{CIM-to-PIM, PIM-to-PIM} and \textit{PIM-to-PSM} transformations. Given
an abstract model at the CIM level, it is possible to apply transformations for
generating a model of the PSM level. In this context, it is necessary to
follow the process activities described by the methodology. 

These three SOD-M levels have no support for describing non-functional requirements.The following section introduces $\pi$-SODM the extension proposed for considering these requirements.



