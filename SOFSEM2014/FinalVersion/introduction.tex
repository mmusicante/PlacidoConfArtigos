Model Driven Development (MDD)~\cite{Favre06arigorous} is a top-down approach for  designing and developing software systems 
%The main ideas of MDD were originaly 
proposed by the Object Management Group
(OMG)\footnote{\texttt{http://www.omg.org/mda}.}. 
MDD provides a set of
guidelines for  structuring  specifications using
 \textit{models} that specify a software system at different levels of abstraction or \textit{viewpoints}: 
%MDD defines three viewpoints:

\begin{trivlist}
\item \textit{Computation Independent Models (CIM):} This viewpoint represents the software system at its highest level of abstraction. It focusses on the system
environment, and on its business and requirement specifications. 
At this moment of the development, the structure and system processing details are still unknown or undetermined. 
 
\item \textit{Platform Independent Models (PIM):} This level focusses on the system functionality, hiding the details of any particular platform. 
%The specification defines those parts of the system that are independent of the development platform. 

\item \textit{Platform Specific Models (PSM):} This level focusses on the functionality, in the context of a particular implementation platform.
Models at this level combine the platform-independent view with the specific aspects of the platform to implement the system.  
\end{trivlist}
%
Besides the notion of model at each level of abstraction, MDD requires the use of \textit{model transformations} between levels.
These transformations may be automatic or semi-automatic and implement the refinement process between levels. 

%\bigskip 
MDD has been applied for developing service oriented applications.
In Service-Oriented Computing~\cite{Papazoglou2007}, pre-existing services are
combined to produce applications and provide the business logic. The selection of services is usually guided by the functional requirements of the application being developed. 
Some methodologies and techniques have been proposed to help the software
developer in the specification of functional requirements of the business logic,
such as the Service-Oriented Development Method
(SOD-M)~\cite{decastro1}. 
SOD-M is based on MDD and proposes
models, practices and techniques to aid in software development. SOD-M does not
 suport  the specification of non-functional requirements, such as
security, reliability, and efficiency. Ideally, non-functional requirements
would be considered along with all the stages of the software development. The
adoption of non-functional specifications from the early states of development
can help the developer to produce applications that are capable of dealing with
the application context.

%\bigskip

Non-functional properties of service-oriented applications have been
addressed in academic works and standards~\cite{ws-co,ws-tra,wsci}.
Dealing with these kind of properties involves the use of specific technologies
in different layers of the SOC architecture, for instance during the description
of service APIs (such as WSDL\cite{wsdl} or REST~\cite{rest}) or to express
service coordinations (like WS-BPEL~\cite{bpel03}).

%Existing work addressing non-functional properties for service-oriented applications can be classified under four approaches:
%\begin{trivlist}
%\item[-] Those coming from the Business Process Domain and from the Web
%service standards that propose ad-hoc protocols. 
%Examples of these include the WS*-Family of the W3C\cite{ws-co,ws-tra,btp}.
%\item[-] Those adopting a classic middleware approach were non-functional
%properties are provided as middleware services, like the ones presented
%in~\cite{BeVaC00,RohmBSS02,NepalFGJKS05,Bonita}.
%\item[-] Those providing languages and enabling the specification of protocols
%used for expressing service
%compositions~\cite{LakhalKY05,LakhalKY05b,RouachedGABG06,FauvetDDB05}.
%\item[-] Those adopting separation of concerns,
%like~\cite{Milanovic06,Espinosa-OviedoVZC09,SchmelingCM11,PastranaPK11}.
%\end{trivlist}

Protocols and models implementing non-functional properties assume the existence of a global control of the artifacts implementing the application.
They also assume that each service exports its interface.
So, the challenge of supporting non-functional properties is related to
\textit{(i)} the specification of the business rules of the application; and 
\textit{(ii)} dealing with the technical characteristics of the infrastructure where the application is executed.

%In this context, there is a need of 
%\textit{(i)} flexible protocols and models accepting a best effort approach for ensuring non-functional properties.
%\textit{(ii)} A methodology for specifying the application logic and its associated non-functional properties, starting at the early phases of the development process.
%
%In this work, we proppose to address the structured engineering of service-oriented applications in the presence of non-functional properties where: 
%\textit{(i)} The designer must make the diference between requirements that concern the application logic and the non-functional requirements;
%\textit{(ii)} General concepts must be provided to support the representation of  different non-functional requirements;
%\textit{(iii)} The non-functional requirements, specified in an abstract way, must provide enough information to be translated into the specification of technical aspects implementing concrete non-functional properties to be verified at runtime.


This paper presents $\pi$SOD-M  a methodology for supporting the construction of service-oriented applications, taking into account both functional and non-functional requirements. The goal of the methdology are to:
\textit{(i)} improve the construction process by providing an abstract view of the application and ensure the conformance to its specification;
\textit{(ii)} reduce the programming effort through the semi-automatic generation of  models for the application, to produce concrete implementations from high abstraction models.


%\bigskip
Accordingly, the remainder of the paper is organized as follows:
Sections~\ref{sec:sodm} and~\ref{sec:pisodm} present, respectively, the SOD-M method of service software process and $\pi$SOD-M our proposed extension to deal with non-functional requirements.
A proof of concept  is developed in Section~\ref{sec:poc}. Section~\ref{sec:related} describes related works.
Section~\ref{sec:conclusions} concludes the paper  and gives final remarks.
