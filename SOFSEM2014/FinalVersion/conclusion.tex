Over the last years, a number of approaches have been proposed for the
development of web services. These approaches range from the proposal of new 
languages for web service descriptions~\cite{bpel03,SBS04} 
to techniques to support phases of the development cycle of this kind of
software~\cite{lipari2007,BianculliGSBG07}. In general, these
approaches concentrate on specific problems, like supporting transactions or
QoS, in order to improve the security and reliability of service-based
applications. Some proposals address service composition: workflow
definition~\cite{AalstHKB03,MuP06} or semantic equivalence between
services~\cite{BHM06}. The proposed solutions come from many communities, 
including those of Theoretical Computer Science~\cite{VA05,GGP08}, 
Software Engineering~\cite{Aal03,MendesPDB09}, Programming
Languages~\cite{MPC08,bpel03} and Databases~\cite{ABM01}.

Despite the variety of tecniques proposed, there is not yet a consensus on a software
process methodology for web services. Some methodologies address the service-based 
development towards a standard or a new way to develop reliable applications. 
SOD-M and SOMF \cite{somf} are MDD approaches for web services; 
S-Cube \cite{scube2010book} is focused on the representation of business processes and 
service-based development; SOMA \cite{soma} is a methodology described by IBM
for SOA solutions; Extended SOA \cite{PapazoglouH06} merges RUP and
BPM\cite{bpm} concepts for service modeling; DEVISE \cite{DEVISE} is a
methodology for building service-based infrastructure for collaborative
enterprises. Other proposals include, the WIED model \cite{TongrungrojanaL04}, 
that acts as a bridge between business modeling and design models, and traditional 
approaches for software engineering \cite{sommerville08} applied to SOC. 

\bigskip

This paper presented the $\pi$SOD-M software process for specifying and designing service based applications in the presence of some non-functional constraints. 
Our proposal harness the SOD-M method with constraints, policies and contracts in order to consider non-functional constraints.
We implemented the proposed meta-models on the Eclipse platform and we illustrated the approach by developping a simple application.

$\pi$SOD-M is being used in an academic environment.
So far, the preliminary results indicate that $\pi$SOD-M approach is useful for the development of complex web service applications. 
We are now working on the definition of a PCM-level meta-model to generate BPEL programs (instead of $\pi$-PEWS). 
